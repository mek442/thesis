%% LyX 2.0.3 created this file.  For more info, see http://www.lyx.org/.
%% Do not edit unless you really know what you are doing.
\documentclass[12pt,english]{report}
\usepackage[T1]{fontenc}
\usepackage[latin9]{inputenc}
\usepackage{float}
\usepackage{calc}
\usepackage{amsthm}
\usepackage{amsmath}
\usepackage{setspace}
\usepackage[hidelinks]{hyperref}
\usepackage{longtable,lscape}
\usepackage{multirow}
\usepackage{colortbl}
\usepackage{mathrsfs}
\usepackage{listings}
\usepackage{xspace}
\usepackage{nccmath}


\makeatletter

%%%%%%%%%%%%%%%%%%%%%%%%%%%%%% LyX specific LaTeX commands.
\providecommand{\LyX}{L\kern-.1667em\lower.25em\hbox{Y}\kern-.125emX\@}
%% Because html converters don't know tabularnewline
\providecommand{\tabularnewline}{\\}
\floatstyle{ruled}


%%%%%%%%%%%%%%%%%%%%%%%%%%%%%% Textclass specific LaTeX commands.
\usepackage{UTSAthesis}      
\usepackage{times}            
\usepackage{latexsym}
\usepackage{fixltx2e}

\newenvironment{ruledcenter}{%
  \begin{center}
  \rule{\textwidth}{1mm} } {%
  \rule{\textwidth}{1mm} 
  \end{center}}%


  \theoremstyle{definition}
  \newtheorem{defn}{\protect\definitionname}
\theoremstyle{plain}
\newtheorem{thm}{\protect\theoremname}

\usepackage{fixltx2e}
\usepackage{times}
\usepackage{graphicx}
\usepackage{epsf}
\usepackage{verbatim}
\usepackage{cite}
\usepackage{url}
\usepackage{color}
\usepackage{alltt}
\usepackage[linesnumbered,ruled,vlined]{algorithm2e}
\usepackage{longtable,lscape}
\usepackage{multirow}
\usepackage{colortbl}
\usepackage{mathrsfs}
\usepackage{listings}




\usepackage{enumerate}
\usepackage[all]{xy}
\usepackage{algorithmic}
\usepackage{float}
\usepackage{threeparttable}
\usepackage{mathrsfs}
\usepackage{balance}
\usepackage{amsmath}
\usepackage{indentfirst}
\usepackage[table]{xcolor}
\usepackage{amssymb}
\usepackage{hyperref}
\usepackage{times}
\usepackage{graphicx}
%\usepackage{epsf}
\usepackage{verbatim}
%\usepackage{psfig}
%\usepackage{cite}
\usepackage{url}
\usepackage{color}
\usepackage{alltt}

\newcommand{\Framework}{}


\newcommand{\Add}{\CodeIn{add}}
\newcommand{\AVTree}{\CodeIn{AVTree}}
\newcommand{\Assignment}[3]{$\langle$ \Object{#1}, \Object{#2}, \Object{#3} $\rangle$}
\newcommand{\BinaryTreeRemove}{\CodeIn{BinaryTree\_remove}}
\newcommand{\BinaryTree}{\CodeIn{BinaryTree}}
\newcommand{\Caption}{\caption}
\newcommand{\Char}[1]{`#1'}
\newcommand{\CheckRep}{\CodeIn{checkRep}}
\newcommand{\ClassC}{\CodeIn{C}}
\newcommand{\CodeIn}[1]{{\small\texttt{#1}}}
\newcommand{\CodeOutSize}{\scriptsize}
\newcommand{\Comment}[1]{}
\newcommand{\Ensures}{\CodeIn{ensures}}
\newcommand{\ExtractMax}{\CodeIn{extractMax}}
\newcommand{\FAL}{field-ordering}
\newcommand{\FALs}{field-orderings}
\newcommand{\Fact}{observation}
\newcommand{\Get}{\CodeIn{get}}
\newcommand{\HashSet}{\CodeIn{HashSet}}
\newcommand{\HeapArray}{\CodeIn{HeapArray}}
\newcommand{\Intro}[1]{\emph{#1}}
\newcommand{\Invariant}{\CodeIn{invariant}}
\newcommand{\JUC}{\CodeIn{java.\-util.\-Collections}}
\newcommand{\JUS}{\CodeIn{java.\-util.\-Set}}
\newcommand{\JUTM}{\CodeIn{java.\-util.\-TreeMap}}
\newcommand{\JUTS}{\CodeIn{java.\-util.\-TreeSet}}
\newcommand{\JUV}{\CodeIn{java.\-util.\-Vector}}
\newcommand{\JMLPlusJUnit}{JML+JUnit}
\newcommand{\Korat}{Korat}
\newcommand{\Left}{\CodeIn{left}}
\newcommand{\Lookup}{\CodeIn{lookup}}
\newcommand{\MethM}{\CodeIn{m}}
\newcommand{\Node}[1]{\CodeIn{N}$_#1$}
\newcommand{\Null}{\CodeIn{null}}
\newcommand{\Object}[1]{\CodeIn{o}\ensuremath{_#1}}
\newcommand{\PostM}{\MethM$_{post}$}
\newcommand{\PreM}{\MethM$_{pre}$}
\newcommand{\Put}{\CodeIn{put}}
\newcommand{\Remove}{\CodeIn{remove}}
\newcommand{\RepOk}{\CodeIn{repOk}}
\newcommand{\Requires}{\CodeIn{requires}}
\newcommand{\Reverse}{\CodeIn{reverse}}
\newcommand{\Right}{\CodeIn{right}}
\newcommand{\Root}{\CodeIn{root}}
\newcommand{\Set}{\CodeIn{set}}
\newcommand{\State}[1]{2^{#1}}
\newcommand{\TestEra}{TestEra}
\newcommand{\TreeMap}{\CodeIn{TreeMap}}

\newenvironment{CodeOut}{\begin{scriptsize}}{\end{scriptsize}}
\newenvironment{SmallOut}{\begin{small}}{\end{small}}

\newcommand{\pairwiseEquals}{PairwiseEquals}
\newcommand{\monitorEquals}{MonitorEquals}
%\newcommand{\monitorWField}{WholeStateW}
\newcommand{\traverseField}{WholeState}
\newcommand{\monitorSMSeq}{ModifyingSeq}
\newcommand{\monitorSeq}{WholeSeq}

\newcommand{\IntStack}{\CodeIn{IntStack}}
\newcommand{\UBStack}{\CodeIn{UBStack}}
\newcommand{\BSet}{\CodeIn{BSet}}
\newcommand{\BBag}{\CodeIn{BBag}}
\newcommand{\ShoppingCart}{\CodeIn{ShoppingCart}}
\newcommand{\BankAccount}{\CodeIn{BankAccount}}
\newcommand{\BinarySearchTree}{\CodeIn{BinarySearchTree}}
\newcommand{\LinkedList}{\CodeIn{LinkedList}}

\newcommand{\Book}{\CodeIn{Book}}
\newcommand{\Library}{\CodeIn{Library}}

\newcommand{\Jtest}{Jtest}
\newcommand{\JCrasher}{JCrasher}
\newcommand{\Daikon}{Daikon}
\newcommand{\JUnit}{JUnit}

\newcommand{\trie}{trie}

\newcommand{\Perl}{Perl}


\newcommand{\SubjectCount}{11}
\newcommand{\DSSubjectCount}{two}

\newcommand{\Equals}{\CodeIn{equals}}
\newcommand{\Pairwise}{PairwiseEquals}
\newcommand{\Subgraph}{MonitorEquals}
\newcommand{\Concrete}{WholeState}
\newcommand{\ModSeq}{ModifyingSeq}
\newcommand{\Seq}{WholeSeq}
\newcommand{\Aeq}{equality}

\newcommand{\Meaning}[1]{\ensuremath{[\![}#1\ensuremath{]\!]}}
\newcommand{\Pair}[2]{\ensuremath{\langle #1, #2 \rangle}}
\newcommand{\Triple}[3]{\ensuremath{\langle #1, #2, #3 \rangle}}
\newcommand{\SetSuch}[2]{\ensuremath{\{ #1 | #2 \}}}
%\newtheorem{definition}{Definition}
%\newtheorem{theorem}[definition]{Theorem}
%\newcommand{\Equiv}[2]{\ensuremath{#1 \EquivSTRel{} #2}}
%\newcommand{\EquivME}{\Equiv}
%\newcommand{\EquivST}{\Equiv}
%\newcommand{\EquivSTRel}{\ensuremath{\cong}}
%\newcommand{\Redundant}[2]{\ensuremath{#1 \lhd #2}}
%\newcommand{\VB}{\ensuremath{\mid}}
%\newcommand{\MES}{method-entry state}

%\newcommand{\Small}[1]{{\small{#1}}}

%\newcommand{\CenterCell}[1]{\multicolumn{1}{c|}{#1}}


\newcommand{\pure}{\CodeIn{pure}}
\newcommand{\correct}{\CodeIn{correct}}
\newcommand{\visitor}{\CodeIn{Visitor}}
\newcommand{\savethis}{\CodeIn{savethis}}



\newcommand*{\listingsfont}{\fontfamily{[Scale=0.7]{Courier}}\selectfont}


\usepackage{color}
\definecolor{sh_comment}{rgb}{0.12, 0.38, 0.18 } %adjusted, in Eclipse: {0.25, 0.42, 0.30 } = #3F6A4D
\definecolor{sh_keyword}{rgb}{0.37, 0.08, 0.25}  % #5F1441
\definecolor{sh_string}{rgb}{0.06, 0.10, 0.98} % #101AF9


\lstset {
 frame=bt,
 rulesepcolor=\color{black},
 showspaces=false,showtabs=false,tabsize=1,
 numberstyle=\tiny,numbers=left,
 basicstyle= \listingsfont,
 stringstyle=\color{sh_string},
 keywordstyle = \color{sh_keyword}\bfseries,
 commentstyle=\color{sh_comment}\itshape,
 captionpos=b,
 xleftmargin=0.5cm, xrightmargin=0.5cm,
 lineskip=-0.3em,
 escapebegin={\lstsmallmath}, escapeend={\lstsmallmathend}
}


\usepackage{listings}
\usepackage{multirow}
\usepackage{varwidth}
\usepackage{array}
\usepackage{float}
\usepackage{balance}

\floatstyle{ruled}
\newfloat{example}{thp}{lop}
\floatname{example}{Example}

\newcolumntype{R}[1]{>{\raggedleft\arraybackslash}p{#1}}
\newcommand\NameEntry[2]{%
	\multirow{#1}*{%
		\begin{varwidth}{5em}% --- or minipage, if you prefer a fixed width
			 #2%
		\end{varwidth}}}

%%%%%%%%%%%%%%%%%%%%%%%%%%%%%% User specified LaTeX commands.
\usepackage{epsfig}          
\usepackage{setspace}         
\usepackage{lscape}
\usepackage{graphicx}
\usepackage{cite}


\usepackage{xspace}
\usepackage{enumerate}
\usepackage{array} 
\usepackage{color}
\usepackage{cases}
\usepackage{amsfonts}


\@ifundefined{showcaptionsetup}{}{%
 \PassOptionsToPackage{caption=false}{subfig}}
\usepackage{subfig}
\makeatother

\usepackage{babel}
  \providecommand{\definitionname}{Definition}
\providecommand{\theoremname}{Theorem}



\begin{document}


\supervisor{Xiaoyin Wang, Ph.D.}

\committeeB{Jianwei Niu, Ph.D.}

\committeeD{Palden Lama, Ph.D.}

\committeeC{Wei Wang, Ph.D.}

\committeeE{Lide Duan, Ph.D.}




\informationitems{Doctor of Philosophy in Computer Science}{Ph. D.}{B. Sc.}{Department of Computer Science}{College of Sciences}{March}{ 2018 }


\thesiscopyright{Copyright 2018 Shaikh Nahid Mostafa \\
All rights reserved. }


\dedication{\emph{I would like to dedicate this thesis/dissertation 
to my parents.}}


\title{\textbf{Study and Prioritization of Dependency-Oriented Test Code for Efficient Regression Testing}}


\author{Shaikh Mostafa}
\maketitle
\begin{acknowledgements}
First of all, I would like to thank my adviser, Xiaoyin Wang. If it were not for him, I would
neither have started nor finished my PhD. I very much enjoyed spending time and working with him.
I have to thank him for numerous nights of assisting on the project or conversation through email.
His advises regarding my research as well as professional career have been invaluable.
Although I could never do enough to return all that he has done for me, I hope to assist
the junior co-workers with the same passion in the future.\\ 


I would like to convey my deepest gratitude to Tao Xie for being a tremendous mentor of the research project PerfRanker. Furthermore,
I would like to thank Jianwei Niu, Palden Lama, Wei Wang, and Lide Duan
for their generous help during my graduate studies. They served on my thesis
committee and helped me improve presentation of this material.\\


My internship mentors also provided great support and lovely environments: John
Micco (Google) and Abhayendra Shing (Google) to carry out a research project on google
infrastructure. I was honored to work with exceptional master's student Rodney Rodriguez.
I would like to thank other colleagues and friends, including Foyzul Hasan and Xue Qin.\\

Last but not least, I would like to thank my sister, my brother, my parents
and my wife for their never-ending love, care, and support.

\vspace{4in}

\begin{singlespace}
\emph{This Doctoral Dissertation Thesis
was produced in accordance with guidelines which permit the inclusion
as part of the Doctoral Dissertation
the text of an original paper, or papers, submitted for publication.
The Doctoral Dissertation must
still conform to all other requirements explained in the Guide for
the Preparation of  Doctoral Dissertation 
at The University of Texas at San Antonio. It must include a comprehensive
abstract, a full introduction and literature review, and a final overall
conclusion. Additional material (procedural and design data as well
as descriptions of equipment) must be provided in sufficient detail
to allow a clear and precise judgement to be made of the importance
and originality of the research reported. }

\emph{It is acceptable for this
Doctoral Dissertation to include as chapters authentic copies of papers
already published, provided these meet type size, margin, and legibility
requirements. In such cases, connecting texts, which provide logical
bridges between different manuscripts, are mandatory. Where the student
is not the sole author of a manuscript, the student is required to
make an explicit statement in the introductory material to that manuscript
describing the students contribution to the work and acknowledging
the contribution of the other author(s). The signatures of the Supervising
Committee which precede all other material in  Doctoral Dissertation attest to the accuracy of this statement.}\end{singlespace}
\end{acknowledgements}
\begin{abstract}
In modern software process, software testing is performed along with software development to detect errors as early as possible and to guarantee that changes made in software did not affect the system negatively. However, during the development phase, the test suite is updated and tends to increase in size. Due to the resource and time constraints for re-executing large test suites, it is important to develop techniques to reduce the effort of regression testing.\\ 
 
Unit testing testing is the core of test driven development where software testers need to test a class or a component without integration with some of its dependencies. Typical reasons for excluding dependencies in testing high cost of invoking some dependencies (e.g., slow network or database operations, commercial third-party web services), and the potential interference of bugs in the dependencies. In practice, mock objects have been used in software testing to simulate such missing dependencies. However, due to exclusion of dependencies, mock-objects-based testing is not suitable for performance regression and backward incompatibility regression testing.\\
 
A small extent of performance degradation may result in severe consequence and running performance test needs more time and resources. Also, it can be hard for a developer to understand performance impact with few runs. Furthermore, A code changes touches several test cases is very common during the evolution of software. Due to the resource and time constraints for re-executing large test suites, it is important to develop techniques to reduce the effort of regression testing. Our proposed method focus on the performance test suite prioritization via performance impact analysis of change.\\
 
Nowadays, due to the frequent technological innovation and market changes, software libraries are evolving very quickly. To make sure that existing client software applications are not broken after a library update, backward compatibility has always been one of the most important requirements during the evolution of software platforms and libraries. Previous studies on this topic mainly focus on API signature changes between consecutive versions of software libraries, but behavioral changes of APIs with untouched signatures are actually more dangerous and are causing most real world bugs because they cannot be easily detected. Our study categorizes behavioral backward incompatibilities according to incompatible behaviors and invocation conditions. We propose compare those detected in regression testing with those causing real-world bugs and prioritization test case based on backward incompatibilities in dependencies.\\
 
\end{abstract}

\pageone{}


\chapter{Introduction}


\section{Motivation}

In every aspect of our lives, e.g., ranging from communication to social networks to entertainment to business to transportation to health uses of software is predominant. Therefore, software correctness is of utmost importance. The world has witnessed the high cost of bugs far too many times. Prior studies in the area of software testing estimate that bugs cost global economy more than \$300 billion per year. Despite the risk of introducing new bugs while making changes, software constantly evolves due to never-ending requirements. When Agile software development models were first envisioned, a core tenet was to iterate more quickly on software changes and determine the correct path via exploration--essentially, striving to "fail fast" and iterate to correctness as a fundamental project goal. \\



Thus, software developers have to check, at each project revision, not only correctness of newly added functionality, but also that the recent project changes did not break any previously working functionality. Software testing is the most common approach in industry to check correctness of software. Software developers usually write tests for newly implemented functionality and include these tests in a test suite (i.e., a set of tests for the entire project). To check that project changes did not break previously working functionality, developers practice regression testing - running test suite at each project revision.  Clearly, more automation and better tools for software testing can lower the cost of software development, increase the reliability of software, and reduce the negative economic impact of defective software. So, Continuous Integration process which automatically compile, build, and test every new version of code committed to the central team repository, ensures that the entire team is alerted any time the central code repository contains broken code.\\


Although regression testing is important, it is costly because it frequently runs a large number of tests. Some research studies [38, 48, 67, 109, 117] estimate that regression testing can take up to 80\% of the testing budget and up to 50\% of the software maintenance cost. The cost of regression testing increases as software grows. For example, Google reported that In 2014, approximately 15 million lines of code were changed in approximately 250,000 files in the Google repository on a weekly basis. Google's codebase is shared by more than 25,000 Google software developers from dozens of offices in countries around the world. On a typical workday, they commit 16,000 changes to the codebase, and another 24,000 changes are committed by automated systems. 
Their regression-testing system, TAP [65,146,149], has had a linear increase in both the number of project changes per day and the average test-suite execution time per change, leading to a quadratic increase in the total test-suite execution time per day. As a result, the increase is challenging to keep up with even for a company with an abundance of computing resources. Other companies and open-source projects also reported long regression testing time.\\



Due to the resource and time constraints for re-executing large test suites, it is important to develop techniques to reduce the effort of regression testing.  Unit testing testing is the core of test driven development where software testers need to test a class or a component without integration with some of its dependencies. Typical reasons for excluding dependencies in testing high cost of invoking some dependencies (e.g., slow network or database operations, commercial third-party web services), and the potential interference of bugs in the dependencies. In practice, mock objects have been used in software testing to simulate such missing dependencies. However, due to exclusion of dependencies, mock-objects-based testing is not suitable for performance regression and backward incompatibility regression testing.\\
 
A small extent of performance degradation may result in severe consequence motivate us that running performance test needs more time and resources. Also, it can be hard for a developer to understand performance impact with few runs. Furthermore, A code changes touches several test cases is very common during the evolution of software. Due to the resource and time constraints for re-executing large test suites, it is important to develop techniques to reduce the effort of regression testing. Our proposed method focus on the performance test suite prioritization via performance impact analysis of change.\\

 
Nowadays, due to the frequent technological innovation and market changes, software libraries are evolving very quickly. To make sure that existing client software applications are not broken after a library update, backward compatibility has always been one of the most important requirements during the evolution of software platforms and libraries. Previous studies on this topic mainly focus on API signature changes between consecutive versions of software libraries, but behavioral changes of APIs with untouched signatures are actually more dangerous and are causing most real world bugs because they cannot be easily detected. Our study categorizes behavioral backward incompatibilities according to incompatible behaviors and invocation conditions. We compared those detected in regression testing with those causing real-world bugs and prioritization test case based on backward incompatibilities in dependencies.

\section{Thesis Statement}
Our thesis is three-pronged:\\
\\
\medskip\vspace{+0.05cm}
\noindent\begin{tabular}{|p{16cm}|}
	\hline
	\textit{\textbf{(1)} Developer should have better understanding when mock and when real object}\\
	\textit{\textbf{(2)} There is a need for an automated performance test suite prioritization technique for collection-intensive software that works in practice}\\
	\textit{\textbf{(3)} It is important to study and categorization of behavioral backward incompatibilities}\\
	\hline
\end{tabular}
\medskip


\section{Contributions}
To confirm the thesis statement, this dissertation makes the following contributions:
\begin{itemize}
\item The dissertation presents the first empirical study on status and the usage of mocking frameworks in software testing. The study shows that mocking frameworks are widely used in practice, and a small portion of dependencies are mocked. Software testers tend to mock source code classes than libraries, while library classes also
take a substantial proportion (40\%) in all mocked classes and developer most likely to mock network, database and time consuming services API.

\item The main goal of this dissertation project is to investigate and prioritize software regression testing and how we can improve the efficiency and cost reduction
of software regression testing. Particular, given a large number of existing performance test cases, how should we rank them such that commit of code changes are pushed to repository in a given time. The dissertation introduces a novel technique for performance RTS, named PerfRanker and our evaluation results show that, compared with the best of the three other baseline approaches, our approach achieves an average improvement of 17.6 percentage points on APFD-P and 27.4 percentage points on DCG. Furthermore, for Apache Commons Math and Xalan, our approach is able to rank top 1 affected test case within top 8\% and top 16\% test cases, and top 3 affected test cases within top 37\% and 30\% test cases, respectively.

\item The dissertation presents study and categorization behavioral backward incompatibilities according to incompatible behaviors and invocation conditions beyond api signature. 
The study shows that behavioral backward incompatibilities are prevalent among Java software libraries, and caused most of real-world backward-incompatibility bugs. Furthermore, many of the behavioral backward incompatibilities are intentional, but are rarely well documented.
\end{itemize}

\section{Organization}
\vspace{-0.5cm}
This dissertation is organized as follows. Chapter 2 introduces the background and related work. Chapter 3 describes our empirical study on the usage of mocking frameworks in software testing. Chapter 4 presents performance test case prioritization and experimental results for them. Chapter 5 describe our study categorization of behavioral backward incompatibilities according to incompatible behaviors and invocation conditions. In chapter 6 and 7 we discusses the lesson learned and future work direction.




\chapter{Background and Related Work}
The purpose of this section is to provide the background of this study and a review of
related literature. First, the background is introduced followed by a related work section
about Mocking, Regression, Performance Regression and Backward incompatibility.



\section{Mocking}
Mocking objects are used in software testing to simulate software dependencies so that the testing process can be accelerated and the testing scope can be limited to the component under test (instead of going beyond the interface of dependencies and invoke potential bugs relevant to dependencies). To
simulate real dependencies, mock objects typically have the same interface as the objects they mimic. Therefore, the client object remains unaware of whether it is using a real object or a mock object.

There have been a number of existing research efforts on enhancing mocking frameworks or leveraging mock objects to improve automatic software testing. Freeman et al.~\cite{IEEEhowto:kopka}~\cite{Freeman} presented the basic process and concepts of using mocks objects in unit testing, as well as a mocking framework jMock. Galler et al.~\cite{Galler} proposed an approach to automatically generation of mocking objects satisfying certain preconditions to serve as test inputs in automatic test-case generation for Java unit testing. Taneja et al.~\cite{Taneja} proposed to automatically generate mock objects to simulate the behavior of database systems in the automatic test-case generation of database systems. Coelho et al.~\cite{Coelho} proposed an approach to generate mock agents in the unit testing of multi-agent software systems. Due to the necessity of mock objects in unit testing, researchers also proposed approaches to automatically generating mock objects along with unit test cases ~\cite{woda}~\cite{Pasternak}. For the existing test cases which are not using mock objects, Saff and Ernst~\cite{Saff} proposed an approach to automatically refactor such test cases by adding mock objects. Marri et al.~\cite{Marri} carried out an experiment to study the benefit of using mock objects to simulate file systems. All these research efforts are about automatic generation of mock objects and how to leverage mock objects in specific testing problems, while our work focuses on the current usage status of mocking frameworks and mock objects in real world software projects. As far as we know, this paper is the first research effort to study the usage status of mocking frameworks and mock objects in software practice.

\section{ Regression}
Regression testing is the activity of testing software after it has been modified to gain
confidence that the newly introduced changes do not obstruct the behavior of the existing,
unchanged parts of the software. In order to gain confidence that program changes did not introduce any errors, regression test suites are executed recurringly. The number of test cases
can greatly influence the execution time of a test suite. There are a number of challenges related to regression testing, such as identification of obsolete test cases, regression test selection, prioritization
and minimization and test suite augmentation [19]. Yoo and Harman [44] conducted a
survey on regression testing minimization, selection and prioritization, constituting nearly
200 papers. It encompasses the main research results around regression testing, addressing
the problems of identifying obsolete, reusable and re-testable test cases (selection),
eliminating redundant test cases (minimization) and ordering test cases to maximize early
fault detection (prioritization).



\section{Performance Reression}
The default approach for regression testing is to retest all test cases after releasing a new version, which is an expensive proposition.
To solve this problem, there are good collection of industry case studies and research effort on performance regression testing in software systems. For example, automating regression benchmarking \cite{KALIBERA}, a model-based performance testing framework for workloads\cite{BARNA11}, genetic algorithm to expose performance regression\cite{LUO16}, learning-based performance testing framework for test input data\cite{GRECHANIK12}, symbolic execution to generate load test\cite{ZHANG11} and probabilistic symbolic execution \cite{Chen2016} focus on building better performance regression testing infrastructure and test cases. Other important works are performance bug detection(\cite{JIN12, JOVIC11, KILLIAN10, YAN12}), performance debugging(\cite{SHEN05,HAN12,LEUNG07,AGUILERA03}), automatic fixing performance problem \cite{Nistor15} and 
performance regression testing result analysis(\cite{FOO11,FOO10}) focus on detecting performance regression bugs or provide forensic diagnosis. But Our work 
assumes the existence performance testing infrastructure and test cases, and improves the testing efficiency by prioritizing test suite.


Functional regression is a well research area to reduce testing cost by test case selection based on test case property and
code modification(\cite{ROTHERMEL97,CHEN94}), test suite reduction by removing redundancy in test suite (\cite{HARROLD93,BLACK04,ZHONG06}) and test cases prioritization orders test case execution in a way to hope to detect functional fault faster(\cite{ELBAUM00,KIM02,LI07,ROTHERMEL99}). Different from these work, our goal is to reduce performance regression
testing overhead via test suite prioritization based on change impact analysis whether an operation is expensive or lies in hot path.
Most relevant work on the performance risk implication of code change~\cite{huang2014performance}. However, this work relies on static analysis and focuses on specific types of performance regressions. Furthermore, prioritizing commits is not enough to address performance regression because 
a code changes touches several test cases is very common during the evolution of software. In our study, we found that some commits touches more than 100 test cases.



\section{Backward incompatibility }
Nowadays, as software products become larger and more complicated, software libraries have become a necessary part of almost any software. Since software libraries and their client software are typically maintained by different developers, the asynchronous evolution of software libraries and client software may result in incompatibilities. To avoid incompatibilities, for decades, ``backward compatibility'' has been a well known requirement in the evolution of software libraries. Each API method in an existing version of software library should exactly maintain its behavior in the following versions.


Researchers have noticed that software libraries are evolving frequently for a long time, so a number of studies have been conducted on the evolution of software libraries. Raemaekers et al.~\cite{Raemaekers:APIStability} proposed a measurement of software-library stability which considers API method difference and code difference, and studied the stability of 140 industrial Java systems based on the measurement. McDonnell et al.~\cite{McDonnell:APIStability} studied the stability of Android APIs (in terms of added and removed classes and methods), and the time lag between the release of API changes and the corresponding adaptation at the client software side. Espinha et al.~\cite{Espinha:WebAPI} interviewed 6 web client software developers and conducted an empirical study on four widely used web services to understand their API evolution trends, including the frequency of API changes, and the time given client developers to upgrade to the new version of services. Bavota et al.~\cite{Bavota:upgrade,Bavota:upgradeJ}, studied the evolution of software dependency upgrades in the apache software ecosystem. Robbes et al.~\cite{Robbes} studied the reaction of developers to deprecated APIs in SmallTalk ecosystem. The existing research efforts mainly focus on signature-level API changes (Raemaekers et al.'s work further considers the amount of code difference) to measure API changes and stability. By contrast, our study focuses on behavioral changes of software libraries, which are more difficult to be detected and may cause more severe consequences.






\chapter{PerfRanker: Prioritization of Performance Regression Tests for Collection-Intensive Software}


\section{Introduction}
\label{sec:intro}
During software evolution, frequent code changes, often including problematic changes, may degrade software performance. For example, a study~\cite{huang2014performance} found that upgrading from MySQL 4.1 to 5.0 caused the loading time of the same web page to increase from 1 second to 20 seconds in a production e-commerce website. Even small performance degradation may result in severe consequence. For example, Google could lose 20\% traffic due to an increase of 500ms latency~\cite{Google}. Amazon could have 1\% decrease in sales due to a 100ms delay in page rendering~\cite{Stevefamov}. \\

Developers can apply systematic, continuous performance regression testing to reveal such performance regressions in early stages~\cite{foxref,poliniref,Ejref,MITCHELL,KALIBERA}. But due to its high overhead, performance regression testing is expensive to conduct frequently. Actually, the typical execution cost of popular performance benchmarks varies from tens of minutes to tens of hours~\cite{huang2014performance}, so it is impractical to run all performance test cases for each code commit. Recently, PerfScope~\cite{huang2014performance} was proposed to predict whether a code commit may significantly affect software performance and thus require performance testing. Specifically, PerfScope extracts various features from the original version and the code commit, and trains a classification model for prediction. Although PerfScope helps reduce code commits for performance regression testing, its empirical evaluation shows that a non-trivial proportion of code commits still require performance testing; thus, there is still a strong need of reducing the cost of conducting performance regression testing on a code commit, even after applying PerfScope in practice.\\

%Such problematic changes are referred to as performance regressions in this paper. and GCC from 4.3 to 4.5, Mozilla developers experienced an up to 19\% performance regression, which forced them to consider a complete switchover.Performance regression testing is a major approach to detecting performance regressions, This is widely advocated in academia, open source community and industry .for web servers is 3 minute to 1 hr, for databases is 10 minutes to 3 hrs, for compilers is 1 hr to 20 hrs and for operating systems is 2 hrs to 24 hrs . With the high revision frequency and high testing cost, 


%
%it does not consider the performance impact of code commits on individual test cases, and thus results in two limitations. First,
%

To address such strong need, developers shall prioritize performance test cases on a code commit for three main reasons. First, there can be high cost to execute all performance test cases on a code commit for large systems in practice. Second, as reported in a previous industrial study~\cite{TSEPerform} and our study in Section~\ref{sec:motivation}, various random factors may affect the observed execution time, so it typically requires a large number of repetitive executions to confirm a performance regression. Therefore, with prioritized test cases, developers can better distribute testing resources  (i.e., do more executions on test cases likely to trigger performance regressions). 
Third, a code commit may accelerate some test cases while slowing down others. It is often important for the developers to understand the performance of their software under different scenarios, while a coarse-grained commit-level technique is not helpful on this requirement. \\



%research effort by addressed this problem with PerfScope, or regressional performance testing. Specifically, for each new code commit, PerfScope applied to the code change various program analysis (e.g., hot path analysis, bound analysis, and data dependency analysis) to  to 

%For example, our study shows that it averagely requires 150 executions to confirm a 10\% performance difference, which is typically considered significant, and dynamic techniques bringing in more than 10\% overhead are typically not considered for deployment-time usage~\cite{}. 


%Mozilla’s Talos performance regression detection system~\cite{firefoxTalos} runs performance tests every time a change is pushed to the Firefox source repository~\cite{firefoxperf}. In our empirical evaluation, we observe that one code commit may affect more than 100 test cases. 


 


%There is a useful work on the performance risk implication of code change. But their approach may not accurately assess the risk of performance regression issues because of generic nature of static modeling and lacking profiling information. Furthermore, prioritizing commits is not enough to address performance regression because  a code changes touches several test cases is very common during the evolution of software. In our study, we found that some commits touches more than 100 test cases. So the key difference is that we focus  on the prioritizing test cases in regressional performance testing via performance impact analysis of code changes that includes one time profiling and static modeling.

To develop an effective test-prioritization solution for performance regression testing, we focus on \textit{collection-intensive software}, an important type of modern software whose execution time is heavily spent on loading, manipulating, and writing collections of data. Collections are widely used in software for scalable data storage and processing, and thus collection-intensive software is very common. Examples include libraries for data structures, text formating and parsing, mathematics, image processing, etc. Also, collection-intensive software is often used as components in complex systems. Moreover, a recent study~\cite{JIN12} shows that a large portion of performance bugs are related to loops, which are often used to iterate through collections. Our statistics show that 89\% and 77\% of loops iterate through collections for our two subjects Xalan and Apache Commons Math, respectively.\\

%improper iteration on collections has been identified as one of the most
%

For collection-intensive software, a straightforward approach to prioritizing performance test cases on a code commit would be measuring collection iterations (e.g., loops) impacted by the code commit and executed by each test case. However, such an approach may not be precise enough to differentiate test cases in the presence of newly added iterations, manipulations, and processing of collections, as well as their effect on existing collection iterations. Consider the simplified code example from Xalan in Listing~\ref{list:example-p3}. The code commit involves a new loop, and its location may or may not be at the hot spot for all or most test cases. Therefore, its impact on different test cases may largely depend on the different iteration counts of the added loop, the side effect of changing variable \CodeIn{list}, and the operations in the loop. Since Loop$_B$ depends on a collection variable \CodeIn{limits}, which further depends on Loop$_A$, we can infer the test-case-specific iteration count of Loop$_B$ from that of Loop$_A$; such iteration count can be acquired by profiling the base version for all test cases. Furthermore, we can infer the effect of adding \CodeIn{list.add(...)} on \CodeIn{list} with the iteration count of Loop$_B$, and update the iteration count of loops dependent on \CodeIn{list}. Moreover, we can enhance the estimation precision by using test-case-specific execution time of operations (e.g., \CodeIn{new Arc(...)}). 


{\fontsize{10}{10}
	\begin{lstlisting}[columns=flexible,language=Java,caption=Collection Loop Correlation, label={list:example-p3}]
	  while(i <= m\_size){ //Loop A
	    limits.add(new Limit(...))
	    ...
	  }
	  ...
	+ Collections.sort(limits);
	+ for (int i = 0; i < limits.size()-1; i++) { //Loop B
	+   list.add(new Arc(limits.get(i), ...));
	+ }
	\end{lstlisting}
}


These observations inspire us for three main insights to effectively model a code commit and its effect on existing collections and their iterations. First, collection sizes (e.g., \CodeIn{limits.size()}) and loop-iteration counts (e.g., Loop$_B$) can often be correlated, so collection sizes can be inferred from loop-iteration numbers and vice versa. Also, collection variables (e.g., \CodeIn{limits}) can be used as bridges to infer iteration counts of new loops (e.g., Loop$_B$) from existing loops (e.g., Loop$_A$). Second, collection manipulations (e.g., \CodeIn{list.add(...)}) are often inside loops, so the size of collections referred by collection variables (e.g. \CodeIn{list}) can be estimated from loop-iteration counts (e.g., Loop$_B$). Third, due to the large number of elements in collections, the average processing time of elements (e.g., \CodeIn{new Arc(...)}) is relatively stable, so a method's average execution time in the new version may be estimated from that in the base version.\\ 

Based on the three insights, we propose PerfRanker, which consists of four automatic steps. First, on a base version, we execute each test case in a profiling mode to collect information about the test execution, including the runtime call graph and the iteration counts of all executed loops. We also perform static analysis to capture the dependency among collection objects and loops. Second, based on the profiling information, we construct a performance model for each test case. Third, given a code commit, we estimate the execution time of each test case on the new version (formed by the code commit) by extending and revising its old performance model. We use profiling information and loop-collection correlations to infer parameters of the new performance model, and refer to this step as \textit{Performance Impact Analysis}. Fourth, we rank all the test cases based on the performance impact on them.\\

%its impact on each test case from two aspects: the execution time of the added and removed code in the revised method, and the execution time of the loops affected by the collection objects which are return values of revised methods. 




%To address the two limitations above, we propose a novel approach to prioritize individual test cases in performance regression testing via performance impact analysis, which estimates the impact of a given code revision on the execution time of a given test case. 



%our algorithm automatically takes the information in each code commit by statically analyzing the added code and removed code from automatically generated diff of each commit. Incorporating the profile and static information into each test case's call graph we determine the run-time cost and the execution frequency of the code affected by code changes. 

We implement our approach and apply it on two sets of code commits collected from popular open source collection-intensive projects: Apache Commons Math and Xalan. To measure the effectiveness of test case prioritization for a code commit in performance regression testing, we use three metrics: (1) \textit{APFD-P} (Average Percentage Fault Detected for Performance), an adapted version of the APFD metric~\cite{AlexeyAPFD} for performance testing, (2) \textit{DCG}~\cite{NDCG}, a general metric for comparing the similarity of two sequences, and (3) \textit{Top-N Percentile}, which calculates the percentage of test cases needed to be executed to cover the top N test cases whose execution time is most affected by the code commit. Our evaluation results show that, compared with the best of the three other baseline approaches, our approach achieves an average improvement of 17.6 percentage points on \textit{APFD-P} and 27.4 percentage points on DCG. Furthermore, for Apache Commons Math and Xalan, our approach is able to rank top 1 affected test case within top 8\% and top 16\% test cases, and top 3 affected test cases within top 37\% and 30\% test cases, respectively. 

%Since we are not aware of previous research efforts on prioritizing test cases for performance regression testing, we developed  for Apache Commons Math, and an improvement of 16.6 percentage points on APFD and 27.2 percentage points on DCG for Xalan, respectively

%calculated impact score is used with adaptive APFD and DCG metric formula to rank the test case which shows that they can significantly reduce the cost of performance regression. 


This paper makes the following major contributions:
  \vspace{-0.1cm}
\begin{itemize}
  \item A novel approach to prioritizing test cases in performance regression testing of collection-intensive software.
  
  
%supported by a novel technique called performance impact analysis which estimates the performance impact of code revisions on test cases.
  
%  \item A motivation study on the number of executions required to acquire stable performance testing results.
    
    
  \item Adaptation of the APFD metric to measure the result of test prioritization for performance regression testing. 
    
  \item An evaluation of our approach on real-world code commits from two popular open source collection-intensive projects. 
  
\end{itemize}
  
  
%The rest parts of this paper are organized as follows. In Section~\ref{sec:motivation}, we present a motivation study showing that multiple test executions are required to confirm a performance regression. In Section~\ref{sec:approach}, we introduce our approach and performance impact analysis in details. We present our evaluation results in Section~\ref{sec:evaluation}, and discuss some related issues in Section~\ref{sec:discuss}. Before we conclude in Section~\ref{sec:conclusion}, we introduce related research efforts in Section~\ref{sec:related}, and indicate some future work in Section~\ref{sec:future}.
\section{Motivation}
\label{sec:motivation}

%As mentioned in the introduction, besides the fact that performance test cases often involves larger inputs and thus consumes more resources and take more time, the performance variance among different executions of a same test case provides us an even stronger motivation for test prioritization in performance regression testing. 

%Theoretically, for each test case that does not involve random process (e.g., generating random numbers, randomly selecting next method invocations), all of its executions should go over exactly the same instruction sequence, and thus take exactly the same time. 

\begin{figure}
	\centering
	\includegraphics[width=0.45\textwidth]{performance/images/standard-deviation.pdf}
	\caption{Relative Standard Deviation vs. Sample Size}	
	\label{fig:motiv-math}	
	
\end{figure}

In this section, we provide preliminary study results to motivate prioritization of performance regression tests, due to high cost of executing the same test case for many times for performance regression testing. In particular, modern mechanisms in hardware and software often bring in random factors impacting performance. Some well-known examples are the randomness in scheduling cores and buses in multi-core systems~\cite{MultiCoreRandom}, in caching policies~\cite{CacheRandom}, and in the garbage-collection process~\cite{GBRandom}, etc. These factors interact with each other and amplify their effect so that the execution time of a test case may vary substantially from time to time. 

To neutralize such randomness, researchers or developers execute a test case multiple times and calculate the average performance~\cite{LiuCGO}. To better understand this requirement, we perform a motivating study (with more details on the project website~\cite{perfranker}) on the two open source projects used in evaluating our approach. In particular, we execute the test cases for 5,000 times, and randomly select samples with different sizes to calculate their standard deviation. In Figure~\ref{fig:motiv-math}, we show how the execution time's relative standard deviation (y axis) changes as the execution times (x axis) increase from 1 to 5,000. The figure shows that the average relative standard deviation with sample size 1 is over 20\% in both projects. In other words, if only one execution is used, the recorded execution time is expected to have more than 20\% difference from the average execution time in the 5,000 executions. Note that 20\% is a very large variance because 10\% performance enhancement is typically considered significant, and techniques incurring over 10\% overhead are often considered too slow for deployment purposes~\cite{DoubleTake}. The figure also shows that 173 and 47 executions are required to achieve relative standard deviation less than 10\% for Apache Commons Math and Xalan, respectively. Executing the whole test suite for many times can be prohibitively expensive, calling for prioritization of performance regression tests, as targeted by our approach. 


%The number of executions to acquire precise performance measurement may vary from scenario to scenario, and there is not a comprehensive study for it. In this paper, to understand how many executions are required to achieve precise result for our evaluation settings (i.e., performance testing of Java Projects on a state-of-the-art multi-core server), we performed a motivation study on the two test suite of our evaluation subjects: Apache Commons Maths and Xalan. 

%\textbf{Study Design.} For each project, we execute the test suite for 5,000 times on a quiet\footnote{``Quiet'' means that no other user process is running on the server.} DELL X630 server with 32 cores and 256GB memory. From the executions, for each test case, we recorded its execution time as a sequence with 5,000 data points\footnote{To avoid potential noises such as test-case loading or IO blocking, we removed the outliers that are beyond 2 times of standard deviation, which is a standard data purification process~\cite{}, and removed 0.9\% data points on average.}. In our study, we define ~\textit{Sample Size} as the number of iterations a test case is executed, and ~\textit{Sample Performance} as the average execution time calculated from the executions. For a small sample size (e.g., 1), the sample performance from different samples tends to have larger standard deviation, and when the sample size gets larger, the standard deviation among samples tends to get smaller, and our goal is to find the sample size that can lead to small enough standard deviation. 
%
%For each sample size $i$, we use all the sub-sequences with length $i$ of the original data-point sequence whose length is $N$\footnote{The value of $N$ is slightly smaller than 5,000 due to data purification.}
%
%Specifically, if the $k^{th}$ data point in the original data-point sequence is denoted as $ex(k)$, the first sample, the $k^{th}$ sample, and the last sample in the set will be:
%
%\begin{multline}
%sample(i, 1) = \{ex(1), ex(2), ...., ex(i)\}\\
%sample(i, k) = \{ex(k), ex(k+1$ $mod$ $N), ...., ex((k+i)$ $mod$ $N)\}\\
%sample(i, N) = \{ex(N), ex(1), ...., ex(i-1)\}\\
%\end{multline}
%
%Therefore, our sample set for sample size $i$ can be viewed as executing the test case for $i$ times from any start point in the original execution sequence. Based on these sample sets, if we denote the sample performance of $sample(i, k)$ as $perf(i, k)$, the relative standard deviation for sample size $i$ can be calculated as follows. 
%
%\begin{equation}
%Var(i) = \frac{Std(\cup{k=1}{N}{\{perf(i, k)\}\}})}{\Sigma{k=1}{N}{ex(k)}/N}
%\end{equation}
%
%In the formula, to normalize the standard deviation for different test cases and projects, we divide it with the average execution time of the test case for all $N$ executions. Therefore, we actually use the average execution time of the 5,000 executions to approximate the ideally precise execution time. It should be noted that, 5,000 is much larger than 10 to 100 which are common used in performance related research efforts in literature~\cite{}~\cite{}, so we believe that the approximation should be reasonably accurate.  For the sample sizes from 1 to 5000, we draw the trends of relative standard deviation (average for all test cases)
%\textbf{Study Results.} 


% 1,580 and 406 executions are required to achieve relative standard deviation less than 5\%. 



%in which horizontal axes represent sample sizes in log scale and vertical axes represent the relative standard deviation. 


%It should be noted that, our study has some limitations such as the purification of execution-time outliers which may exist in the real world testing, and the fact that sample sets are not independent with each other because there are overlaps between neighboring sample sets (e.g., $sample(i, 1)$ and $sample(i, 2))$. However, these two limitations will both reduce the observed relative standard deviation, indicating that we actually provided a lower bound of relative standard deviation over sample sizes. Therefore, as a motivation study, it show that we do need several hundred executions to confirm a performance regression around 10\%. Based on this fact, the prioritization of test cases is very helpful, because we can focus the highly prioritized test cases, and execute them for more times to better confirm perform regressions. 





%We can see that at first iteration average deviation from ideal is 20.79\% and reduced to 5\% with 1580 iteration. Finally the overall variance decreased to 1\% with iteration 4200. Furthermore, deviation decreases sharply from 20\% to 10\% with 173 times of execution whereas reduction from 10\% to 5\% take 1407 iteration. So we can say that large number of iteration of execution of test case are needed to get a stable execution time of a performance test case. A small extent of performance degradation may result in severe consequence motivate us that running performance test needs more time and resources. It is hard for a developer to understand performance impact with few runs.\\


%In the evaluation of techniques for compilers and systems, to reduce some randomness, researchers sometimes choose to turn off some features, such as doing garbage collection at fixed time point. However, 

%{\fontsize{7}{7}
%\begin{multline}
%\begin{gathered}
%Average = Avg(1,2,3,...,5000)\\
%Var(i) = \frac{ Std( Avg(1,2,...,i),...,Avg(5000,1,...,i-1))}{Average}\\
%Iteration(i) = \frac{ \sum \limits_{k=1}^{alltest} Var(i)}{alltest}
%\label{eq:eq-motivation}
%\end{gathered}
%\end{multline}
%}





%We run real test cases and captured average execution time of two popular open source project apache common math and xalan. \textbf{Common math}:  We chose 2703 test cases which are affected by code changes from a collection 3900 test cases and each of test cases  exercise 5000 time and record their execution time. In the formula ~\ref{eq:eq-motivation} function Avg and Std corresponding to average and standard deviation. For iteration-1, it becomes {\fontsize{7}{7}\[ Var(1) = \frac{Std(1,2,3,...,5000)}{Average}\]} which is fraction of standard derivation of 5000 data and average of 5000 data. The value of Var(1) represents how far the deviation is from the ideal case with a single iteration.For iteration-2, it becomes {\fontsize{7}{7} \[Var(2) = \frac{ Std( Avg(1,2), Avg(2,3),...,Avg(5000,1))}{Average}\]} where each data of standard deviation is average of 2 data which means that 2-iteration on the average how much deviate from ideal case. For iteration-5000, it becomes {\fontsize{7}{7}\[Var(5000) = \frac{ Std(Avg(1,2,...,5000),...,Avg(5000,1,...,4999))}{Average}\]} where each data of standard deviation is average of 5000 data which means Var(5000) value equals to zero that we considered as ideal or base case. According to the formula~\ref{eq:eq-motivation}, we generate a graph in the figure ~\ref{fig:common-math-motiv} 

\section{Approach}
\label{sec:approach}

In this section, we introduce our test prioritization approach in detail. In particular, we first present the overview of our approach, major technical challenges, and our performance model. After that, we introduce performance-impact estimation of a code commit based on our performance model, including method-execution-time estimation, and method-invocation-frequency estimation based on collection-loop correlation and iteration-count inference.


\subsection{Overview}

%To prioritize performance test cases, the core difficult

%The objective of PerfChecker is to examine a code commit content and determine the cost of a change, and then rank the performance test case on basis of a change performance impact cost which helps to reduce the cost in performance regression testing. Profiler takes the base version code and statically modifies the code in order to extract profile information. Run all the performance test case and store method execution time, execution frequency and loop counter in profile database.Profile database also store JDK library method's execution summary. 

\begin{figure}
\centering
	
	\includegraphics[width=3.5in, height=3.5in]{performance/images/Workflow2.pdf}
	\caption{Workflow of Our Approach}	
	\label{fig:approach-workflow}
		
\end{figure}

Figure~\ref{fig:approach-workflow} shows the workflow of our approach. The input to our approach includes the project code base, its code commits, and performance test cases. The output of our approach is an ordered list of performance test cases. The first step of our approach is to  profile the base version of the project under test. During the profiling, for each test case, we record the dynamic call graph as its original performance model. We also record average execution time and frequency of methods, and iteration counts of all loops. At the same time, we statically analyze the base version to gather dependencies between loops and collection variables, as well as aliases among collection variables. When a new code commit comes, we conduct performance impact analysis to estimate its performance impact on all test cases, and prioritize test cases accordingly. 


%Another component is Alias analyzer which statically analyze the source code loop controlling variable, array and collection with their alias and store their information in database. Diff generator produce diff of each committed code into the repository. Parser parses information regarding the changed files, lines and types (add, delete) from the generated diff file content. Filter prunes out insignificant change such as stylish change or renaming. If the changes are significant, the commit will be considered to fed to the performance impact analyzer. Performance impact analyzer analyse the cost of the change by constructing CFG of the modified method and integrating the profile information from the database into the call graph of the test case. At the final stage it produce a sorted list of test case according the cost performance impact score. 



 
 
%The challenge is how to estimate the cost about performance impact of each of test case after a change without actually running the software.We first briefly describe the input-output of our approach and major component. 

%Although it is straightforward to prioritize test cases based on the code commit's performance impact on them, 

\textbf{Technical Challenges.} Performance impact analysis is the core of our approach. Although we focus on collection-intensive software, it is still challenging to conduct performance impact analysis, facing three major technical challenges:

\begin{itemize}

\item Challenge 1. A code commit may include any type and scope of code changes, from one-line revision, to feature addition and interface revision. Therefore, there is a strong need of a unified and formal presentation for code commits.  

\item Challenge 2. A code commit may contain newly added code, especially new loops. No execution information of such code is available, but given that loops can have high impact on performance, there is a strong need of estimating the code commit's execution time and frequency.
 
\item Challenge 3. Even if the execution time of changed code in a code commit has little impact on performance, the code commit may include changes on collection variables, eventually affecting the performance of unchanged code. 

%In our paper, we refer such performance changes as \textit{side performance impact}. 

\end{itemize}

To address Challenge 1, we present a code commit as three sets of methods: added methods, revised methods, and removed methods. As our performance model is based on the dynamic call graph, any code commit can be mapped to a series of operations for method addition, removal, and replacement in the performance model. To address Challenge 2, we leverage the recorded profiling information of the base version as much as possible. Specifically, if an existing method is invoked in the newly added code, we can use the recorded execution time of the existing method as its execution-time estimation for this new invocation. Furthermore, as discussed in Section~\ref{sec:intro}, we use collection variables as bridges to estimate iteration counts of new loops from those of existing loops. To address Challenge 3, we track all the element-addition and element-removal operations of collection variables in the newly added code, and estimate the size change of collection variables from the iteration count of their enclosing loops. This new size is used to update the iteration counts of loops depending on the changed collection variables. 

%Note that, the execution time of added and removed methods can affect existing test executions only through revised methods, where invocations to them are added and removed. %For each revised method, based on the performance model in Section~\ref{subsec:model}
%we estimate the execution time of the pre-commit version and post-commit version, respectively, and estimate the performance impact of a revised method as the execution-time difference between its pre-commit version and post-commit version. Then, the directly performance impact of a code commit can be calculated as the performance impact sum of all revised methods, together with side performance impact. 



%and if a newly added loop depends on an existing collection variable, 

%we can use the recorded size of the collection variable to estimate the number of iterations of the loop. The detailed process of estimating direct performance impact is presented in Section~\ref{subsec:direct}. 



%their new value ranges in the new version according to the iteration number of the writing operations. Then, for each collection variable $v$, if there is any loop $l$ depending on $v$, we calculate the side performance impact of $l$ with $v$'s new value range and sum up side performance impact of all such loops. The detailed process of estimating side performance impact is presented in Section~\ref{subsec:side}.

Since we focus on collection-intensive software, we consider only loops whose iteration number depends on collection variables, e.g., variables of array type, and other collection types defined in Java Utility Collections. Note that there are also some loops whose iteration number depends on simple integers, such as a loop to sum up a numbers from $i$ to $j$, but such loops are not common in collection-intensive software, and our evaluation results show that our approach is effective on both data-processing software (Xalan) an mathematics software (Apache Commons Maths). 

%The rationale behind this design decision are (1) according to literatures~\cite{}~\cite{}, most performance issues are related to loops; and (2) most loops in programs are for manipulating iterative data structures (i.e., collection variables).  


%This is a limitation of our performance impact analysis, b ut 
  
\subsection{Performance Model}
\label{subsec:model}
\begin{figure}[t]
\centering
  \includegraphics[width=4in, height=3in]{performance/images/performance-model.pdf}
  
   \caption{An Example Performance Model}	
    \label{fig:performance-model}
   
 \end{figure}

In this subsection, we introduce our performance model to break  down execution time of a test case to all the methods invoked by the test case. The basic intuition behind our model is that the execution time of a method invocation $M$ is the execution-time sum of all method invocations directly invoked in $M$, together with the execution time of instructions in $M$. Since most basic operations in Java programs are performed by JDK library methods (e.g., a string concatenation), the latter part is typically trivial compared with the former part, so our performance model ignores instructions in $M$ itself, but focuses only on methods that $M$ invokes. 

%Specifically, for each test case, our performance model is based on the dynamic call graph of the test case. So the nodes of the graph are method bodies, and the edges are invocations. In the graph, we record two attributes: invocation frequency, and average execution time. For example, on the directed edge from method $A$ to method $B$, we record the average number of invocations from $A$ to $B$ in each invocation of $A$, and on the node $B$, we record the average execution time of $B$.  

We illustrate our model in Figure~\ref{fig:performance-model}, where each node represents a method and each directed edge represents an invocation relation. Here each node annotated with label $t_{avg}$, which represents the average execution time of a method. Each edge in the graph is labeled with $f_{AB}$, which represents the invocation frequency of method B from method A. Given a code commit, the performance model of the post-commit version can be acquired by adding and removing nodes and edges to the original performance model. For example, in Figure~\ref{fig:performance-model}, method $D$ is a revised method and it now calls (1) method $G$, which it originally calls, (2) $E$, which is an existing method in the base version, and (3) $H$, which is a newly added method. With the average execution time and invocation frequency of all invoked methods in $D$, we are able to calculate an execution-time estimation for revised method $D$. The new execution time at $D$ can be propagated upward to its ancestors, until the main method is reached and a new estimation of the whole program's execution time can be made. 


% $m\textsubscript{K}$ are changed and they are called from multiple context $m\textsubscript{B}$,$m\textsubscript{G}$  and $m\textsubscript{H}$. So the total cost is calculated from the modification cost of method $m\textsubscript{J}$  and $m\textsubscript{K}$ by multiplying the frequency  $f\textsubscript{BJ}$,$f\textsubscript{GJ}$,$f\textsubscript{HK}$ and $f\textsubscript{GK}$. Finally, The cost is propagated from child to parent in the graph.

\subsection{Performance Impact Analysis}
\label{subsec:direct}

The basic idea of performance impact analysis is to calculate the  execution-time change of each revised method $M$ in the code commit. Then, through propagating the execution-time change to $M$'s predecessors in the performance model, we can calculate the execution-time change of the whole test case at the root node. 

%We focus only on revised methods because added and removed methods may affect software performance only through revised methods (i.e., by adding or removing method invocations).

We realize this idea in three steps. First, for each revised method, we extend the performance model to either add it and/or some of its callees (and transitive callees). Second, we estimate the execution-time change of each method in the new performance model. Third, we estimate the invocation frequency on the edges of the new performance model. We next introduce the three steps in details.



\subsubsection{Model Extension}
\label{subsub:findInvoke}

For each revised method, we add its direct and transitive callees (e.g., methods $E$ through $K$ in Figure~\ref{fig:performance-model} for the revised method $m_D$) into the performance model, if they do not already exist in the model. In this recursive process, we terminate the extension of a method node if it is an unrevised method existing in the base version, a JDK library method, or a method whose source code is not available. Since we use one base version for a series of code commits, a revised method (and even some of its predecessors) may not exist in the base version because they are added after the base version. In such a case, we transitively determine $M$'s predecessors (callers) until we reach methods in the base version. For example, if $C$ and $D$ in Figure~\ref{fig:performance-model} are added between the base version and the code commit under analysis, we determine that $D$ is invoked by $B$ and $C$, and $C$ is invoked by $A$, so that we add $C$ and $D$ to the performance model. 

When the new code version of the revised methods is available, we  
statically determine the direct and transitive callers and callees for the revised method, and one remaining challenge is to resolve polymorphism, where one method invocation may have multiple targeted method bodies. Although it is straightforward to apply off-the-shelf points-to analysis, since we have the profiling information of the base version, we make use of the information to acquire a more precise call graph. Specifically, if a method invocation is not involved in the method diff (i.e., the method invocation can be mapped to the same method invocation in the base version, such as $G$ in Figure~\ref{fig:performance-model}), we assume that its target is not changed and we use the same targeted method body as recorded for the base version. Otherwise, we apply points-to analysis~\cite{Spark} in Soot~\cite{Soot} to find the possible targeted method bodies for the method invocation. When a method invocation is mapped to multiple method bodies, we add all bodies to the new performance model, and we divide the estimated frequency of the method invocation by the number of possible targets to attain the invocation frequency of each target. 

%Analyzing the new code version of the revised method, it is straightforward to build a partial call graph with it as the root node. Since  resolve polymorphism, we use the following heuristic. 

%Therefore, by listing all the added and deleted methods and their execution time summary will provide an estimate of 

%So combining the profiling information and control flow analysis together may provide accurate estimate of cost of modified method.

%Estimate the cost of a modified method need to identify all the changes inside the method which are added and deleted.




\subsubsection{Execution-Time Change of Method Bodies}
\label{subsub:findExetime}

The method bodies invoked from a revised method fall into three  categories. The first category is removed method bodies. Their execution-time data are recorded in the performance model of the base version, and their new execution time is estimated as 0.

The second category includes method bodies already existing in the performance model of the base version. Such method bodies include both those defined in the source code and those defined in the JDK library, or those without source code. For existing method bodies, we simply use the recorded average execution time in the base version as their estimated average execution time. For bodies of JDK library methods, we profile Dacapo~\cite{dacapo} to acquire the average execution time of those common JDK library methods. For methods without source code or those not invoked by Dacapo, we use the average execution time of all method bodies in the profile as their estimated execution time, as we have no further information. Note that when a method from the second category is added to the performance model, its original execution time is set as 0.

The third category includes newly added method bodies in the source code. \textit{Note that such method bodies include both those added to the source code in the code commit and those added in any other code commits between the base version and the code commit under analysis.} They also include method bodies defined in libraries but are newly reached due to code revisions from the base version to the new version. For a newly added method body (such as $H$ in Figure~\ref{fig:performance-model}), as discussed in Section~\ref{subsub:findInvoke}, we extract all its callee method bodies, and add them to the performance model (such as $J$ and $K$ in Figure~\ref{fig:performance-model}), and then we calculate its execution-time change using our performance model.

%If an added callee method body still belongs to category-2, then we can iteratively extract its callee method bodies. The extraction process will terminate when all the newly added callee method body either belongs to category-1, or is defined in Java SDK. 



%Changes inside a method normally consists of addition and deletion of existing method call or addition of newly method call. 
%Profile all the performance test case and store their average execution time  will provide the cost of existing method call.
%A newly added method also consists of existing method, JDK library call and newly added method. So profiling JDK library function call
%also helpful to estimate the cost of newly added method. if no information available then statically analyse the change and assign the cost based on the existence of loop and hot api call like database, storage and network.
 
\subsubsection{Invocation Frequencies of Method Bodies}

Given a newly added or revised method $m_x$, for each method $m_y$ that is directly invoked in $m_x$, we estimate the invocation frequency of $m_y$ in $m_x$ (denoted as $fq(m_x, m_y)$) to apply our performance model on the new version. Our estimation technique is based on the control flow graph of $m_x$ and the average iteration counts of loops in $m_x$. Specifically, for any code block $b$ in $m_x$, we use $fq(b, m_y)$ to denote the invocation frequency of $m_y$ from $b$ for each execution of $b$. If $b$ is a basic block without branches and loops, $fq(b, m_y)$ is exactly the number of invocation statements to $m_y$ in $b$; such number can be easily counted statically. Then we calculate $fq(m_x, m_y)$ by applying the inference rules for sequential, branch, and loop structures in Formulas~\ref{equa:seq}-\ref{equa:loop} below recursively on the code blocks of $m_x$:

   
\begin{equation}
\label{equa:seq}
fq([\text{$b_1$; $b_2$}], m_y) = fq(b_1, m_y) + fq(b_2, m_y)
\end{equation}
\begin{equation}
\label{equa:branch}
fq([\text{if() $b_1$ else $b_2$}], m_y) = Max(fq(b_1, m_y), fq(b_2, m_y))
\end{equation}
\begin{equation}
\label{equa:loop}
fq([\text{while$_i$ () $b$}]) = fq(b, m_y) \times C(loop_i)
\end{equation}

In the inference rules, the only unknown parameter is $C(loop_i)$, which denotes the average iteration count of the i$^{th}$ loop in $A$. As an example, given the control flow graph in Figure~\ref{fig:cfg-sample} of m$_x$, for any $m_y$ that $m_x$ invokes, $fq(m_x, m_y)$ can be estimated as in Formula~\ref{equa:example}.

\begin{figure}
\centering
	\includegraphics[width=3.5in, height=1in]{performance/images/cfg-new.pdf}
	
	\caption{An Example Control Flow Graph}	
	
	\label{fig:cfg-sample}
\end{figure}


%Ifwe try to estimate $freq(A, B)$ as a function of average iterations in $A$

%Generally, a change set may introduce performance regression in two ways: one case is the change itself in the hot path of execution and another case is that changes modifies some control variables which result some part of code executing very frequently. We can estimate how frequently the method executing in two ways: Loop correlation and Collection propagation. %Generally the changes inside a method does not lie in a single execution path. Control flow analysis of a modified method helps to identify all the paths inside the method and the changes in the path. 

   
\begin{multline}
\label{equa:example}
fq(m_x), m_y)=fq(A, m_y) + (Max(fq(D, m_y), fq(E, m_y))\\ + fq(F, m_y)) \times C(Loop_B) + fq(G, m_y)
\end{multline}
\vspace{-0.5cm}

%shown a simple CFG of a change of a modified method where each node represents a block of CFG and each annotated with cost which is sum of all the method execution time inside the block. There are two path from node B to node F.  $Cost\textsubscript{BDF}$ is the cost of path $BCDF$ and $Cost\textsubscript{BEF}$ is the cost of path $BCEF$. So the maximum cost will the maximum of this two path multiplied by the loop counter which is the execution time of changes in the method.


 
%\subsection{How to decide How many times a method will execute}
%Generally, a change set may introduce performance regression in two ways: one case is the change itself in the
%hot path of execution and another case is that changes modifies some control %variables which result some part of code
%executing very frequently. We can estimate how frequently the method executing in two ways: Loop correlation and Collection propagation.  

\subsection{Loop-Count Estimation with Collection-Loop Correlation} 

With the estimation of invocation frequency, the only remaining unknown parameter in the performance model of the new version is the loop count of all loops. If a loop exists in the base version and is not affected by the code commit, we directly use the recorded profile from the base version to acquire the iteration count. Two more complicated cases are (1) when a new loop is added, and (2) when the code commit affects the iteration count of an existing loop. Here is our insight: for collection-intensive software, we can construct the correlation between collection sizes and loop counts, and use iteration counts of known loops to infer that of unknown loops, as well as a code commit's impact on iteration counts of known loops. 

\subsubsection{Correlating Loops and Collections} In particular, we consider the following two types of dependencies between loops and collection variables:

\begin{itemize}
	\item Iteration Dependency. A loop $L$ is iteration-dependent on a collection variable $v$ if $L$'s loop condition depends on the size attribute of $v$.
	\item Operation Dependency. A collection variable $v$ is operation-dependent on a loop $L$ if there exists an element addition or removal on $v$ in $L$.
	\vspace{-0.15cm}
\end{itemize}

To identify iteration dependencies, for a For-Each loop (e.g, \CodeIn{for(A a : ListOfA)}), we simply consider that the loop is iteration-dependent on the collection variable being iterated via the loop. For other loops, we use standard inter-procedural data flow analysis~\cite{SootIFDS}~\cite{IFDS} to track data dependency backward from the loop condition expression, until we reach a size/length attribute of an array or a known collection class from the Java Collection Library. To make sure that the collection size is comparable with the loop count, we consider only two types of data dependencies: (1) direct assignment (e.g., \CodeIn{a = b;}), and (2) addition or subtraction expression with one operand as constant (e.g., \CodeIn{a = b.size() - 1}). 

To identify operation dependencies, for each loop $L$, we check its body for element-addition and element-removal operations on collection variables. For any other method invocations in the loop, we recursively go into the body of each invoked method to further look for such operations. However, we do not consider nested-loop blocks in $L$ or a method invoked from $L$, because such blocks are dependent on their direct enclosing loop. For example, in Figure~\ref{fig:operationDepend}, collection variable $v$ is operation-dependent on Loop$_A$, but $w$ is not (it is operation-dependent on Loop$_B$). 

\begin{figure}
\centering
	\includegraphics[width=3.5in, height=1.3in]{performance/images/operationDepend.pdf}
	
	\caption{Code Sample of Operation Dependency}	
		
	\label{fig:operationDepend}
\end{figure}




After identifying these two types of dependency relations, we further apply points-to analysis~\cite{Spark} to identify alias relations among collection variables. Note that in our analyses we consider only variables of known collection classes from the Java Collection Library. User-defined collections are also common. However, since we use inter-procedural analysis for identifying both types of dependencies, as long as the user-defined collections extend or wrap Java-Collection classes from the Java Collection Library, we are able to handle these user-defined collections by building dependencies directly on the Java-Collection variables inside them. Also note that  we identify all dependencies and alias relations on the base version and record the results so that we need to re-analyze only the revised/added methods when a code commit comes. 

\vspace{-0.2cm}
\subsubsection{Iteration-Count Inference}


With the dependencies identified among collections and loops, when a new code commit comes, we use Algorithm~\ref{alg:infer} to infer the iteration count of new loops and update the iteration count of existing affected loops. 

In the algorithm, we use a work queue to iteratively update sizes of collection variables and loop-iteration counts (stored in $lCount$), and we use the combined map of $MapI$ and $MapO$ to transit between collections and loops. In particular, as shown in Lines 6-11, we update the iteration count of each loop at most once, to avoid infinite update process caused by cyclic dependency (the more in-depth reason is that we use numbers to represent iteration counts, which are not in a bounded domain). In the end, we remove collection variables from $lCount$ to retain only the loops in the map.

\setlength{\textfloatsep}{6pt}
\begin{algorithm}[t]
	\begin{algorithmic}[1]		
		\REQUIRE~~\\
		$MapO$ is a map from collection variables to loops\\
		$MapI$ is a map from loops to collection variables\\
		$lCount$ is a map from loops to iteration counts\\
		\ENSURE~~\\ updated $lCount$\\
		\STATE{$Q \leftarrow MapI.keys()$}
		\STATE{$Map \leftarrow MapO \bigcup MapI$}
		\WHILE{$Q \neq \emptyset$}
		\STATE{$top \leftarrow Q.pop()$}
		\FORALL{$val \in Map.get(top)$}
		\IF{$val \notin lCount.keys() $}
		\STATE{$lCount.add(val, lCount.get(top)$}
		\STATE{$Q.add(val)$}
		\ELSIF{$val$ is a collection variable}
		\STATE{$lCount.set(val, lCount.get(val) + lCount.get(top)$}
		\STATE{$Q.add(val)$}
		\ENDIF
		\ENDFOR
		\ENDWHILE
		\STATE{$lCount.removeAll(MapO.keys())$}
	\end{algorithmic}
	\caption{Iteration Count Inference}
	\label{alg:infer}
\end{algorithm}

\vspace{-0.2cm}
\subsection{Test Case Prioritization}
\vspace{-0.1cm}
Once our approach estimates the performance impact of the code commit on each test case, we can rank test cases according to their relative performance impact. We use the main method as the root for system tests and each test method as the root for unit tests. We consider both positive and negative effect on execute time as it is often also important for developers to understand whether and where their commit is able to enhance the software performance. 


%In the figure ~\ref{fig:approach_overviw_1}, changes in the method named as getNameSpaceURI is trivial to developer because the indirect cost is not visible to developer. We can see that grand parent in the calling stack has loop in nested which execute more that 8000 time in both case. As a result the changes happen in hot path introduce high cost. Furthermore, developer may not be aware of another loop in path which is a child method in calling stack. As a result average counter of this loop in each test case more than 144000 times which introduce performance regression. To address this problem we introduce call graph with calling context that will provide estimation how many times the modified method getNameSpaceURI could be called from parent. if we know the frequency of the method getNameSpaceURI in the context then it is easy to say how many time the newly added method will be invoked. Similarly the newly added child has a loop that correlated with array size provide the context how many time the methods inside loop will be executed. 

%\begin{figure}
%  \centering
%  \includegraphics[width=\columnwidth,height=3in]{images/call-graph-example.pdf}
%  \caption{Overview of a change call graph in Xalan}	
%  \label{fig:approach_overviw_1}
%\end{figure}

%\subsubsection{Collection Propagation}
%The most common nature of loop in source code is iterate of over array or collections and changes inside loop introduce cost. 
%Generally, collection or array are propagated two ways: parameter passing and field variable. In the figure ~\ref{fig:approach_overviw_2} we can see that three different methods has loop that depends on the size of ArrayList and most
%importantly they are alias. if one of method loop counter is known then any new invocation of other method's loop counter can be predicted which will be helpful to estimate how many times the method under a loop will be called. So static analysis and alias analysis are important way to the solution of the problem.
%
%\begin{figure}
% \centering
%  \includegraphics[width=\columnwidth,height=3in]{images/Alias-example.pdf}
%	\caption{Overview of a alias in Xalan}	  
%  \label{fig:approach_overviw_2}
%\end{figure}
%
%
%\begin{algorithm}
%\SetAlgoLined
%\SetKwInOut{Input}{Input}\SetKwInOut{Output}{Output}
%
%\Input{git version1 head and version2 head}
%\Output{List Added, List Deleted}
%
%\BlankLine
%
%$fileDiff$ = git diff [--options] version1 version2\;
%$listDiff$ = Diffparser($fileDiff$)\;
%$listDiff$ = Filter($listDiff$)\;
%$listAdded$ =[] \;
%$listDeleted$ =[] \;
%\ForEach{each diff $d$ in $listDiff$}{
%  
%	\uIf{$added$}{
%  add tuple $<$class, method, startLine, endLine$>$ to $listAdded$\;
%	}\uElseIf{$deletd$}{
%   add tuple $<$class, method, startLine, endLine$>$ to $listDeleted$\;
%  }\Else{
%    do nothing\;  
%  }
%
%}
%
%\Return{[$listAdded$,$listDeleted$]}
%
%\caption{Change List Generation}
%\label{alg:the_alg_1}
%\end{algorithm}
%
%
%\begin{algorithm}
%\SetAlgoLined
%\SetKwInOut{Input}{Input}\SetKwInOut{Output}{Output}
%
%\Input{$listAdded$,$listDeleted$}
%\Output{Rank of Test suite}
%
%\BlankLine
%
%$listTest$ = getAllTestSuite()\;
%$globalSummary$ = GlobalSummary()\;
%$mapCost$ = $<$test$,$cost$>$ \;
%\ForEach{each Test $t$ in $listTest$}{
%
%  $callGraph$ = LoadCallGraph($t$)\;
%  $localSummary$ = LocalSummary($t$)\;
%  $loopCounter$ = LoopCounter($t$)\;
%	$cost$ = 0;\;
%	
%	\uIf{$listAdded$ contain in $callGraph$}{
%	     buildCFG($listAdded$)\;
%	     cost += estimate addition cost of change from loop counter, local and global method execution summary\;       
%	}\uElseIf{$listDeleted$ contain in $callGraph$}{
%	     buildCFG($listDeleted$)\;
%       cost -= estimate deletion cost of change from loop counter, local and global method execution summary\;     
%	}\Else{
%    do nothing\;  
%  }
%  
%  update $mapCost$\;
%
%}
%
%$mapScore$ = $<$test$,$score$>$ \;
%\ForEach{each Test $t$ in $mapCost$}{
%   
%   $score$ = ($oldTime$ + $cost$)/ $oldTime$ \;
%   update $mapScore$\;
%    
%}
%
%
%$rankedList$ = runRankingAlgo($mapScore$)\;
%
%\Return{[rankedList]}
%
%\caption{Change Impact Performance Analysis}
%\label{alg:the_alg_2}
%\end{algorithm}

%\subsection{Collection-aware Performance Impact Analysis}
%
%Our change impact performance analysis algorithm \ref{alg:the_alg} takes two git head version of source code and automatically generate
%the diff between the two version. Statically analyse the generated diff and extract the list of methods added and deleted in the change source code with start and end line of a change inside a method. In the figure \ref{fig:approach-overviw-3} a modified method simple 
%CFG is shown. Maximum path among the two path is multiplied by loop counter consider as the cost impact. For each test case we calculate total impact cost according to the algorithm form line 17 to 30. Finally return the ranked list of test suite on the basis of
%performance impact. The figure \ref{fig:approach-workflow} describe the complete workflow of our approach.
%
%Our approach divided into two stage: generate change list and change impact analysis. Our change list generation algorithm\ref{alg:the_alg_1} takes two git head version of source code and automatically generate the diff between the two version. From line 1 to 3 describe the process of diff generation, parsing and filtering. First it statically analyse the generated diff contain and filter out the insignificant changes and other non non-relevant changes like java doc, comments etc. From line 6 to 14 describe the process of   extracting list of methods added and deleted in the change source code with start and end line number of a change inside a method. Finally return the list of algorithm\ref{alg:the_alg_1} feed to Our change impact analysis algorithm\ref{alg:the_alg_2}.
%From line 4 to 19 it describe the process of change list impact estimation. To estimating the cost it first build CFG of change method and find out the changes in the path of CFG. Then assign cost to each of the block from profile database which is describe in the figure \ref{fig:approach-overviw-3}. After estimating the cost of a method change, total impact of cost is calculated from the call graph which is  describe in the figure \ref{fig:performance-model}.Finally line 20 to 26 describe the return the ranked list of test suite on the basis of performance impact score.
% 


\section{Evaluation}
\label{sec:evaluation}

For our evaluation, we implement PerfRanker based on Soot for static analysis and Java Agent for profiling the base version. 

%In this section, we first introduce the selection of subjects and code commits in subsection~\ref{subsec:subjects}, and introduce the evaluation setup and environment in Subsection~\ref{subsec:setup}. Then, we introduce the metrics for effectiveness measurement in Subsection~\ref{subsec:metrics}, and the evaluation results in Subsection~\ref{subsec:results}. (points-to analysis, call-graph building, control flow analysis, and data flow analysis for correlating collection variables and loops)  (dynamic call graph, iteration number of loops, and execution time of methods)
\subsection{Evaluation Subjects}
\label{subsec:subjects}

We apply PerfRanker on two popular open source projects: Xalan~\cite{xalan} and Apache Commons Math~\cite{commonMath}. Specifically, Xalan is an XSLT processor, and Apache Commons Math is a library for mathematical operations. We choose these two projects to cover both data formatting and mathematical computations, which are two representative time-consuming components in modern software. Xalan is equipped with a performance test suite of 64 test cases. Since Apache Commons Math is not equipped with a performance test suite, we leverage its unit test cases as performance test cases. 

\textbf{Version Selection.} For Apache Commons Math, we use its version on Jan 1, 2013 as its base version. For Xalan, since there are very few code commits after 2013, we use version 2.7.0 as its base version, as 2.7.0 is the first Xalan version compatible with Java 6 and higher. For both software projects, we collect all code commits from the base version until Mar 17, 2016, the time when we started collecting data for our work.  From all code commits, we remove those that do not change source files and those that do not involve semantic changes (e.g., renaming variables), as developers can easily determine that those commits will not affect software performance. Furthermore, we choose as our code-commit set the top 15 code commits whose changed code portions are covered by most test cases, where test prioritization is most needed. 

%For each selected code commit, we execute test cases on the commit for 5,000 times to determine performance regressions.

In Table~\ref{tab:subjects}, we present some statistics of the studied subjects and versions. The table shows that either project has more than 300K lines of code. Furthermore, there are hundreds of code commits and changed files between the base version and our selected code commits. In our evaluation, we do not update the base version, so the overhead of profiling the base version is low compared with the number of code commits under study.  More details about our evaluation subjects can be found on our project website~\cite{perfranker}.

\begin{table}
	\centering
	\caption{Evaluation Subjects}	
	
	
	\label{tab:subjects}
	\begin{tabular}{|l|r|r|} 
		\hline
		Subject  & Xalan  & Apache Commons Math  \\ 
		\hline
		Base Ver. & 2\_7\_0 & Jan 1st, 2013 \\ 
		\hline
		Size (LOC) of Base Ver. & 413,534   & 398,171  \\ 
		\hline		
		\# Commits Since Base Ver. & 354   & 1,321 \\ 
		\hline
		\# Changed Files  & 1,206  & 1,613 \\ 
		\hline
		Last Commit Date &Aug 11, 2015&Mar 17, 2016     \\ 
		\hline
%		Loop Exercise By Test & 364                         & 4151                             \\ \hline
	\end{tabular}
	\vspace{+1cm}
\end{table}





%We choose xalan version 2.7.0 as base version because version 2.6 is not compatible to jdk 1.6 to upper version. Benchmarking framework for XSLT, developed by Saxonica contains a set of test material, a set of test drivers for various XSLT processors, and tools for analyzing the test results.In the benchmark  we found 128 performance test case and only 64 of them run successfully with Xalan version 2.7.0. Similarly we chose 3900 test case for common math and studied the commits from github between  Dec 29, 2012 and Mar 17, 2016.

\subsection{Evaluation Setup}
\label{subsec:setup}


To determine performance regressions as the ground truth of performance changes for all test cases and code commits, we execute  the test cases for 5,000 times on the base version and each code commit under study. Furthermore, we execute the base version with our Java Agent to record the dynamic call graph and the execution time of each method, as well as the iteration number of each loop. To record average execution time of methods defined in the JDK library, we execute the Dacapo benchmark 9.12~\cite{dacapo} with profiling (we remove Xalan from the benchmark to avoid bias). All the executions are conducted on a Dell x630 PowerEdge Server with 32 cores and 256GB memory, and the server is used exclusively for our evaluation to avoid noises. 


%In Table~\ref{tab:profile}, we show top 5 frequently called JDK API methods and their average execution time in nano seconds. As described in Section~\ref{subsec:model}, this profile data is used in estimating the execution time of newly reached method bodies defined in JDK. 

%\begin{table}[] 
%	\small
%	\centering
%	\caption{Profiling Results}
%	\label{tab:profile}
%	\begin{tabular}{|l|r|r|} 
%		\hline
%		JDK api               & Average Time (ns)                      & Frequency               \\ \hline
%		java.util.Iterator:hasNext & 18	& 84,651,629  \\ \hline
%		java.util.Iterator:next	& 110	& 75,253,712  \\ \hline
%		java.util.List:get	& 22	& 38,302,721  \\ \hline
%		java.util.Map:get	& 100	& 21,632,743  \\ \hline
%		java.util.List:add	& 37	& 7,048,361  \\ \hline
%		... &            &   \\ \hline
%	\end{tabular}
%\end{table}


%We run the selected test sets on base version of each project and record context sensitive execution summaries of methods, loop counter and run time call graph. We can calculated average execution time of method and loop counter from all the profile data which is consider as global summary in our algorithm and local summary as per test case specific profile data. 
%We compared our approach to change aware random ranking which discussed in \ref{subsec:evaluation-result}. Prioritization technique metrics provide to testers the possibility to order their test cases so that test cases with large priority  are executed first and after test cases with less priority are executed in the regression testing process. APFD is commonly used to evaluate test case prioritization techniques to find functional fault. We adapt the metrics equation according to our needs in the formula \ref{eq:eq-apf}. In the formula p define as position in the ordering and IAPFD define as ideal APFD ranking.  The higher value of nAPFD signifies that highly impacted performance test case are in the top of ordering. 





\subsection{Evaluation Metrics}
\label{subsec:metrics}



To the best of our knowledge, our work is the first on prioritizing performance test cases for code commits, and we propose a set of metrics to evaluate the quality of different rankings. In our evaluation, we consider three ranking metrics: Average Percent of Fault-Detection on Performance (\textit{APFD-P}), normalized Discounted Cumulative Gain (\textit{nDCG}), and \textit{Top-N Percentile}. 

\textbf{APFD-P.} \textit{APFD}~\cite{AlexeyAPFD} is a commonly used metric for assessing a test sequence produced by test-case prioritization. If the test-suite size  is $N$, the total number of faults detected by the test suite is $T$, and the number of faults detected by the first $x$ test cases in the test sequence is $detected(x)$, then the \textit{APFD} of the test sequence can be defined in Formula~\ref{formula:apfd}:



\begin{equation}
APFD = \frac{\sum \limits_{x=1}^{N}\frac{detected(x)}{T}}{N} * 100\%
\label{formula:apfd}
\end{equation}


Unlike functional bugs where a test case either passes or fails, performance regressions are not binary but continuous. Performance downgrades of 20\% and 50\% are both regressions, with different severity. Therefore, instead of counting detected faults to attain the value of $detected(x)$, we replace the value of $detected(x)$ with the accumulated performance change. We define the \textit{performance change} of the $i^{th}$ test case in the test sequence (denoted as $change(i)$) in Formula~\ref{formula:impact} as below, in which $exe(i)$ is the execution time of the $i^{th}$ test case in the current version, and $exe(i_{base})$ is the execution time of the $i^{th}$ test case in the base version. 



 %Specifically, we define the performance change of a code commit on a test case as the relative change on execution time. 



\begin{equation}
change(i) = \frac{|exe(i) - exe(i_{base})|}{exe(i_{base})}
\label{formula:impact}
\end{equation}

Then, we define \textit{APFD-P} the same as  Formula~\ref{formula:apfd}, except that $detected(x)$ is defined as the accumulated performance change, as shown in Formula~\ref{formula:detect}, and $T$ is the sum of performance changes on all test cases. Actually, with such a definition, \textit{APFD-P} can be viewed as \textit{APFD} where all test cases reveal faults, and these faults are weighted by performance changes.

\begin{equation}
detected(x) = \sum_{i=1}^{x}change(i)
\label{formula:detect}
\end{equation}

As an illustrative example, consider 3 tests $t_1$, $t_2$, and $t_3$ with 10\%, 20\%, and 30\% performance downgrades, respectively. The best ranking is $t_3$, $t_2$, $t_1$; the total performance impact is $10\%+20\%+30\% = 60\%$; and the covered performance impact after each test is $30\%/60\% = 50\%$, $(30\%+20\%)/60\% = 83\%$, and $(30\%+20\%+10\%)/60\% = 100\%$. The P-APFD is thus (50\% + 83\% + 100\%)/3 = 78\%. 


%{\fontsize{7}{7}
%	\begin{multline}
%	\begin{gathered}
%	APFD_p =  \sum \limits_{i=1}^{p} impactRatio(p); \\
%	APFD = \sum \limits_{p=1}^{n} APFD_p \\
%	nAPFD = \frac{APFD}{IAPFD}
%	\label{eq:eq-apf}
%	\end{gathered}
%	\end{multline}
%}

\textbf{nDCG.} nDCG~\cite{NDCG} is a metric of ranking widely used in information retrieval. The basic idea is to calculate the relative score a given ranking with an ideal ranking, and the score of an arbitrary ranking is defined below, where $change(i)$ is defined in Formula~\ref{formula:impact}.

%and penalize each element that has high relevance score but is ranked lower in the given ranking with a value being logarithmic to the position difference. 

\begin{equation}
DCG (seq) = change(1) + \sum \limits_{i=2}^{N} \frac{ change(i)}{log_2(i)}
\label{formula:dcg}
\end{equation}


%In Formula~\ref{formula:dcg}, for a given sequence $seq$, we use the performance impact on the $i^{th}$ test case ($impact(i)$) as its relevance score, and $N$ to denote the length of $seq$, then the \textit{DCG} value of a given sequence $seq$ is defined in Formula~\ref{formula:dcg}. The \textit{nDCG} value of a given sequence $seq$ is its \textit{DCG} value normalized by the \textit{DCG} value of an ideal sequence, and is thus defined in Formula~\ref{formula:ndcg}, in which $ideal$ is the ideal ranking of the test cases (i.e., in descending order of performance changes).


%\begin{equation}
%nDCG (seq) = \frac{DCG(seq)}{DCG(ideal)}
%\label{formula:ndcg}
%\end{equation}


%The higher value of nDCG signifies that highly impacted performance test case are in the top of ordering.


\textbf{Top-N Percentile.} The \textit{APFD-P} and \textit{nDCG} defined earlier are adapted versions of widely used metrics, and can be used to compare different prioritization approaches. However, they are not sufficiently intuitive to help understand how much developers can benefit from an approach. Therefore, in our evaluation, we also measure how high percentage of top-ranked test cases in a test sequence need to be executed to cover the test cases with top $N$ performance impacts (we use 1 and 3 for $N$). For example, if the test cases with top 1, 2, and 3 performance impacts are ranked in the $2^{nd}$, $9^{th}$, and $5^{th}$ positions in a test sequence with length 100, then the top 1, 2, and 3 percentiles are 2\%, 9\%, and 9\%, respectively. 

\subsection{Baseline Approaches Under Comparison}

Although we are not aware of approaches specifically designed for prioritizing performance test cases in regression testing, it is possible to adapt existing approaches for performance test prioritization. In our evaluation, we compare our approach with three baseline approaches: Change-Aware Random, Change-Aware Coverage, and Change-Aware Loop Coverage.

Specifically, in all baseline approaches, we apply change-impact analysis to rule out the test cases that do not cover any revised methods. Since our performance-impact analysis includes basic change-impact analysis, for fair comparison, we apply this change-impact analysis in all baseline approaches. Note that we gathered coverage information from the base version, and we use the same technique as in our approach when selecting the code commits affecting most test cases in the three baseline approaches. 

After selecting the relevant test cases, the \textit{Change-Aware Random (CAR)} approach simply ranks the test cases in random order\footnote{To acquire more stable results, we use the average result of 100 random ordered test sequences as the result for CAR.}. The \textit{Change-Aware Coverage (CAC)} approach applies coverage-based test prioritization~\cite{Rother99:testprio} on the covered methods with the additional strategy~\cite{additionalTestPrior}, being a state-of-the-art approach in defect-oriented test prioritization. The basic idea is to first select the test case with the highest coverage, and iteratively select the test case that covers the most not-covered code portions as the next test case. In our evaluation, we use method coverage as the criterion, being consistent with the granularity of our performance model. The \textit{Change-Aware Loop Coverage (CALC)} approach is the same as CAC, except for using coverage of loops instead of methods as the criterion. 

\subsection{Quantitative Evaluation}
\label{subsec:results}
In our quantitative evaluation, we compare our approach and the three baseline approaches on all three metrics. 

\subsubsection{\textit{APFD-P} Metric}

%Our approach validated by real-world code changes and evaluate the tool on 354 commits of Xalan and 1321 commits of Apache common math.
%After filtering the result is shown in the figures. The filtered commits by our tool either only change non-source files or have insignificant changes on source files. Interestingly, filtering already reduces a significant number of commits not worth consideration for performance regression testing. We evaluated our result in three different metrics: APFD, DCG and Top Percentage aspect. In the case of APFD and DCG we compared our approach with change aware random approach. In change aware random, we generated 100 random ordering of test sets and calculated average nAPFD and nDCG. %APFD is computed after the prioritization only to measure the performance of the prioritization technique. 


Figures~\ref{fig:common-math-apfd} and~\ref{fig:xalan-apfd} show the comparison results between our approach and three baseline approaches on the \textit{APFD-P} metric. In the two figures and all the following figures, the X axis lists all the code commits studied chronologically, and the Y axis shows the \textit{APFD-P} value (or \textit{nDCG} value). We use different colors to represent different approaches consistently for all figures according to the legend in Figure~\ref{fig:common-math-apfd}. 

\begin{figure}
		\centering
		\includegraphics[width=4in, height=2.5in]{performance/images/common-math-apfd.pdf}
		
		\caption{\textit{APFD-P} Comparison on Apache Commons Math}	
		\label{fig:common-math-apfd}


\end{figure}

\begin{figure}
		\centering
		\includegraphics[width=4in, height=2.1in]{performance/images/xalan-apfd.pdf}
			
		\caption{\textit{APFD-P} Comparison on Xalan}
	
		\label{fig:xalan-apfd}
\end{figure}

%\begin{figure}
%		\centering
%		\includegraphics[width=\columnwidth]{images/common-math-coverage-apfd.pdf}
%		\caption{APFD-P Comparison with Change-Aware Coverage on Apache Commons Math}	
%		\label{fig:common-math-coverage-apfd}
%\end{figure}
%
%\begin{figure}
%		\centering
%		\includegraphics[width=\columnwidth]{images/xalan-coverage-apfd.pdf}
%		\caption{APFD-P Comparison with Change-Aware Random on Xalan}	
%		\label{fig:xalan-coverage-apfd}
%\end{figure}

Figures~\ref{fig:common-math-apfd} and~\ref{fig:xalan-apfd} show that our approach is able to achieve over 80\% \textit{APFD-P} value in most code commits affecting performance, and outperforms or rivals all baseline approaches in all code commits affecting performance from both subject projects. Specifically, for Apache Commons Math, our approach achieves an average \textit{APFD-P} value of 83.7\%, compared with 64.3\% by CAR, 64.6\% by CAC, and 66.1\% by CALC. For Xalan, our approach achieves an average \textit{APFD-P} value of 83.5\%, compared with 65.8\% by CAR, 63.6\% by CAC, and 59.8\% by CALC. Therefore, the improvement on the average \textit{APFD-P} is at least 17 percentage points, compared with baseline approaches on both projects. Furthermore, we do not observe significant effectiveness downgrade in the later versions, indicating that one base version can be used for a relatively long time.

%although we concede that the result may be due to the 

%Therefore, our approach is able to improve the APFD-P 


%In the figure \ref{fig:common-math-apfd}, we can see that overall nAPFD of our approach in Apache common math on the average 21.5\% greater than change aware random approach. In all the version our approach out perform random approach.





\subsubsection{\textit{nDCG} Metrics}

Similarly, Figures~\ref{fig:common-math-dcg} and~\ref{fig:xalan-dcg} show the comparison results between our approach and the three baseline approaches on the \textit{nDCG} metric. 

\begin{figure}
\centering
		\includegraphics[width=4in, height=2.1in]{performance/images/common-math-dcg.pdf}
			
		\caption{\textit{nDCG} Comparison on Apache Commons Math}	
		\label{fig:common-math-dcg}
\end{figure}

\begin{figure}
\centering
		
		\includegraphics[width=4in, height=2.1in]{performance/images/xalan-dcg.pdf}
			
		\caption{\textit{nDCG} Comparison on Xalan}	
		\label{fig:xalan-dcg}
\end{figure}


%\begin{figure}
%		\includegraphics[width=\columnwidth]{images/common-math-coverage-dcg.pdf}
%		\caption{nDCG Comparison with Change-Aware Coverage on Apache Commons Math}	
%		\label{fig:common-math-coverage-dcg}
%\end{figure}
%
%\begin{figure}
%		\includegraphics[width=\columnwidth]{images/xalan-coverage-dcg.pdf}
%		\caption{nDCG Comparison with Change-Aware Coverage on Xalan}	
%		\label{fig:xalan-coverage-dcg}
%\end{figure}

The figures show that our approach outperforms or rivals baseline approaches on \textit{nDCG} in almost all code commits from both subject projects. Specifically, for Apache Commons Math, our approach achieves an average \textit{nDCG} value of 74.5\%, compared with 47.2\% by CAR, 46.5\% by CAC, and 48.4\% by CALC. For Xalan, our approach achieves an average \textit{nDCG} value of 71.7\%, compared with 43.0\% by CAR, 41.4\% by CAC, and 37.4\% by CALC. Therefore, the improvement on the average \textit{nDCG} is over 26 percentage points on both projects. We also observe that there is one code commit from Xalan, in which our approach performs slightly worse than CAR on \textit{nDCG}. In Section~\ref{sec:qualitative}, we further discuss the details of this code commit in Listing~\ref{list:example-n1}.

Finally, we observe that compared with \textit{APFD-P}, the \textit{nDCG} values are generally lower and vary more significantly from code commit to code commit. The reason is that an \textit{nDCG} value is more sensitive to the rank of test cases with the highest performance impacts. For example, consider a test sequence with 100 test cases and only 1 test case has performance impact, and the performance impact value is 100\%. When this test case is ranked top, both \textit{APFD-P} and \textit{nDCG} are 1.0. However, if this test case is ranked $25^{th}$ in the sequence, the \textit{APFD-P} value is still as high as 75\%, but the \textit{nDCG} value becomes $1/log_2(25)$, which is less than 25\%. This result is reasonable because in information retrieval (where \textit{nDCG} was first  proposed), ranking the most relevant result at the $25^{th}$ position is very bad, but in test prioritization (where \textit{APFD} was first proposed), ranking the test case at the $25^{th}$ position is not so bad, because only 25\% test cases need to be executed to execute the test case. Therefore, which of \textit{APFD-P} and \textit{nDCG} is a better metric may depend on whether developers are interested in only a few most severely affected test cases, or a larger number of test cases whose performance is affected. 

%find that overall nDCG is higher comparison to nAPFD due to reason of penalising top impacted test case which are lower in the order. In some version improvement is much higher comparative to  


%The proposition of high nDCG is that highly relevant result appearing at the top. In the figure \ref{fig:common-math-dcg}, we can see that overall nDCG of our approach in Apache common math on the average 29.29\% greater than change aware random approach. 



%some other version for several reason is discussed above. Similarly, in the figure \ref{fig:xalan-dcg}, we can see that overall nDCG of our approach in Xalan on the average 27.21\% greater than change aware random approach.  \\
 


%We further compare our approach with change coverage approach. In figure \ref{fig:common-math-coverage-apfd}  and \ref{fig:xalan-coverage-apfd} shows the comparison of our approach with change coverage approach on  metric APFD. Our approach in common math on the average APFD score 0.85 whereas change coverage  average APFD score 0.68 (17\% greater). Similary, Our approach in xalan on the average APFD score 0.83 whereas change coverage  average APFD score 0.63 (19\% greater).In figure \ref{fig:common-math-coverage-dcg}  and \ref{fig:xalan-coverage-dcg} shows the comparison of our approach with change coverage approach on  metric DCG. Our approach in common math on the average DCG score 0.77 whereas change coverage  average DCG score 0.53 (24\% greater). Similary, Our approach in xalan on the average DCG score 0.71 whereas change coverage  average DCG score 0.41 (30\% greater).\\

\subsubsection{Top-N Percentile}

While \textit{APFD-P} and \textit{nDCG} are normalized quantitative metrics for our problem, they are not sufficiently intuitive for understanding the direct benefit of our approach on developers. Therefore, we further measure how many test cases developers need to consider if they want to cover the top 3 most affected test cases. Tables~\ref{tab:math} and~\ref{tab:xalan} show the results. Columns 1 and 2 present the code commit number and the total number of test cases affected by the code commit, respectively. Columns 3-6 and 7-10 present the top proportion of ranked test cases required to cover top 1 and 3 most-performance-affected test cases\footnote{The results for top 2 show similar trends and are available on the project website~\cite{perfranker}}. 

%In each column group, the first column presents our result, and the remaining columns present the results of CAC, CALC, and CAR, respectively. 

%In each cell, the number in the bracket shows the percentage of test cases need to be considered. 



\begin{table}
\scriptsize
	\centering
	\caption{Top-N Percentile of Apache Commons Math}
	\label{tab:math}	
	
	\begin{tabular}{r|r||r|r|r|r||r|r|r|r}
	\hline
C &T & \multicolumn{4}{c||}{Top 1} & \multicolumn{4}{c}{Top 3} 
\\  \cline{3-10}
\#  & \#   & Our & CAC & CALC & CAR &  Our & CAC & CALC & CAR\\\hline
	1 & 55 & 2\% & 3\%& \textbf{1\%} & 49\% & \textbf{6\%} & 43 & 27\%& 74\%\\
	2 & 60     & 40\%& 61\%& \textbf{38\%} & 49\% & \textbf{51\%} & 61\% & 93\% & 74\%\\
	3 & 152     & \textbf{2\%} & 84\%& 79\%& 49\%  & \textbf{28\%} & 88\% & 88\% & 74\% \\
	4 & 97      & \textbf{3\%} & 18\%& 87\% &  49\%  & \textbf{5\%} & 18\% & 87\% & 74\% \\
	5 & 130    & \textbf{4\%} & 29\% & 43\% & 49\% & \textbf{4\%} & 96\% & 97\% & 74\%  \\
	6 & 15     & \textbf{13\%} & 40\%& 33\%&46\% & 66\% & \textbf{40\%} & 100\% &  73\% \\
	7 & 18      & \textbf{5\%} & \textbf{5\%} & 88\% & 47\% & \textbf{15\%} & 40\% & 88\% & 74\%\\
	8 & 10     & \textbf{10\%} & \textbf{10\%} & 40\% &45\%  & \textbf{60\%} & \textbf{60\%} & 70\%& 70\%\\
	9 & 12     & \textbf{8\%} & 66\%&75\% & 45\% & \textbf{50\%} &66\% & 75\%& 72\%\\
	10 & 12      & \textbf{8\%} & 50\% & 83\% & 45\% & 83\% & \textbf{50\%} & 100\% & 72\% \\
	11 & 36      & \textbf{5\%} & 94\%& 38\% & 48\% & 66\% & 94\%& \textbf{58\%} & 74\%\\
	12 & 13     & \textbf{7\%} & 76\% & 38\%& 45\% & 76\% & 84\%& 100\%& \textbf{72\%} \\
	
	13 & 711    & \textbf{7\%} & 81\%& 84\% & 49\% & \textbf{7\%} & 81\%& 84\%& 74\% \\
	14 & 39      & \textbf{2\%} & 84\%& 25\%& 48\% & \textbf{12\%} & 84\% & 79\% & 74\% \\
	15 & 34      & \textbf{11\%} & 52\% & 17\% & 48\% & \textbf{26\%} & 58\%& \textbf{26\%} & 74\%\\ \hline
	Avg      &   & \textbf{8\%} & 50\% & 51\% & 48\%  & \textbf{37\%} & 65\% & 78\% & 73\% \\  \hline
	\end{tabular}
\end{table} 

\begin{table}
\centering 
\caption{Top-N Percentile of Xalan}
	
\label{tab:xalan}
\scriptsize
\begin{tabular}{r|r||r|r|r|r||r|r|r|r}
\hline
C &T & \multicolumn{4}{c||}{Top 1} & \multicolumn{4}{c}{Top 3} 
\\  \cline{3-10}
\#  & \#   & Our & CAC  & CALC & CAR&  Our  & CAC  & CALC & CAR\\\hline
1 & 63     & 20\%& 46\% & \textbf{14\%} & 49\% & \textbf{36\%} & 60\% & 50\% & 74\% \\
2 & 63   & \textbf{17\%} & 85\% & 22\% & 49\%& \textbf{17\%} & 85\%& 80\%&74\% \\
3 & 63      & \textbf{7\%} & 85\% & 50\% & 49\%  & \textbf{38\%} & 85\% & 66\% & 74\% \\
4 & 63   &  20\%& 80\% & \textbf{7\%} & 49\% & \textbf{20\%} & 85\% & 73\% & 74\% \\
5 & 63   & \textbf{6\%} & 85\% &  100\% & 49\% & \textbf{53\%}  & 85\% & 100\% & 74\%\\
6 & 63      & \textbf{11\%} & 63\% & 31\% & 49\% & \textbf{26\%} & 76\% & 77\% & 74\% \\
7 & 63   & \textbf{11\%} & 46\% &  12\% & 49\% & \textbf{15\%} & 85\% & 93\% & 74\% \\
8 & 63    & \textbf{9\%}  & 35\% & 36\% & 49\% & \textbf{20\%} & 82\% & 95\% & 74\% \\
9 & 58      & \textbf{3\%} & 96\% &  5\% & 49\% & \textbf{25\%} & 96\% & 98\% & 74\% \\
10 & 63   & 30\% & 47\% & \textbf{17\%} & 49\% & \textbf{31\%} & 85\% & 84\% & 74\% \\
11 & 63   & \textbf{1\%} & 31\% & 12\% & 49\% &  \textbf{7\%}  & 62\% & 74\% & 74\% \\
12 & 63     & \textbf{23\%} & 85\% & 88\% & 49\% & \textbf{34\%} & 90\% & 88\% & 74\%\\
13 & 63     & \textbf{12\%} & 46\%& 82\% & 49\%  & \textbf{53\%}  & 65\%& 82\% &74\% \\
14 & 58    & 34\% & 34\% & \textbf{8\%} & 49\% & 43\% & \textbf{37\%}  & 72\% & 74\% \\ 
15 & 63     & 36\% & \textbf{4\%} & 12\% & 49\% & \textbf{36\%} & 79\%& 61\% & 74\% \\ \hline
Avg        & & \textbf{16\%} & 57\%  & 34\% & 49\%& \textbf{30\%} & 77\%  & 79\% & 74\% \\		\hline
\end{tabular}
\end{table}

%For example, change version-3 touches 13 test case and actual impact score of top test case 1st,2nd and 3rd are 0.75, 0.07 and 0.06 respectively. Our approach ranked them with impact score 7.57,2.36 and 2.35 where overall nAPFD score of approach is 0.98 compared to change aware random score 0.63 that is 35\% improvement.


%Performance regression testing cost reduction is highly desirable. For example, running few top impacted test case can save a lot of testing time. Keep that goal in mind, we evaluated our result and We consider only those changes that touches more than three test case in the table \ref{tab:math} and \ref{tab:xalan}. 

Tables~\ref{tab:math} and~\ref{tab:xalan} show that on average our approach is able to cover Top 1 and 3 most-performance-affected test cases within top 8\%, 21\%, and 37\% of ranked test cases in Apache Commons Math, and 16\%, 22\%, and 30\% of test cases in Xalan. The improvements over the baselines approaches are at least 33\%, 37\%, and 28\%. Furthermore, on top 1 coverage, our approach outperforms or rivals the best baseline approach on 13 code commits from Apache Commons Math, and 10 code commits from Xalan. On top 3 coverage, our approach achieves the highest percentage on 11 code commits from Apache Commons Math, and 14 code commits from Xalan. 

%Also, our approach generates better result for Top 1 for code commits affecting fewer test cases, but better result for Top 3 for code commits affecting more test cases. The major reason is that, for code commit affecting fewer test cases, maybe only 1 or 2 test cases are affected with high performance impact. So the 3rd most affected test case may be not very different from other test cases. Actually, one drawback of using Top N coverage as metrics is that, it does not taking into account the actual performance impact, which is considered in $APFD-P$ and $nDCG$.


%consistently better than the two baseline approaches on most code commits, and 


\subsubsection{Overhead and Performance}
In test prioritization, it is important to make sure that the time spent on prioritization is much smaller than the execution time of performance test cases. To confirm indeed that is the case, we record the overhead and execution time of our analysis. Our profiling of the base version has a 1.92 times overhead on Apache Commons Math, and 5.18 times overhead on Xalan. The static analysis per test case takes 29.90 seconds on Apache Commons Maths, and 34.35 seconds on Xalan. Finally, for Apache Commons Math, the average, minimal, and maximal time for analyzing a code commit are 45.35 seconds, 4.23 seconds, and 262.30 seconds, respectively, while for Xalan, the average, minimal, and maximal time for analyzing a code commit are 9 seconds, 3.36 seconds, and 21.80 seconds, respectively. In contrast, it takes averagely 52 (3) minutes to execute test suite of Apache Commons Math (Xalan) for 173 (47) times to achieve an expectation of equal to or less than 10\% execution-time variance.


%we can see that, top1 contains on the average within 8\% in common math and 16\% in xalan. That means more than 85\% test cases can be eliminated. Similarly top2 and top3 contain on the average within  21\% and 37\% respectively in common math and   22\% and 30\% respectively in common xalan. So on the average our approach eliminate 70\% test case to select top 3 impacted test case.


%These table show the result of Top1, Top2 and Top3 impacted test case position in the ranking order and contain within percentage result of apache common math and xalan. For example, first row in the table \ref{tab:math} represents that the changes in version-1 impacted test case 1st, 2nd and 3rd position in the ranked list is 11th,1st and 2nd respectively and top 3 impacted test case contain within 17\% in the ranked list. The changes in version-9 touches 55 test cases and their relative ordering perfectly match the actual ordering of top 3. The main reason is that relative impact difference between the test case distinguishable which means that 1st relative impact 3 to 5 times compared to 2nd and 3rd in this case. But the change on version-13 touches 711 test cases and the impact on test case are very similar and little difference in their relative order which is top1's relative impact score 1.05 to 1.13 times compared to 2nd and 3rd. Within 0.54 score interval contains 653 test case. So we can say that a good separation margin in performance impact among the order test case provide better ranking order. Though their position in the ranking above 50, top 3 contains within 7\%. 


%There is another commit version where our approach perform better than random but it is below 25\% compared to ideal. It is interesting that top 3 contain within 7\% in the ranked list which means it can eliminate 93\% test case. The reason of this issue is explained in later paragraph under Top percentage result evaluation. 



%\subsubsection{Summary of Findings}
%To sum up, our quantitative analysis has the following major findings. 

%\begin{itemize}
%	\item Our technique outperforms both 
%	\item 
%\end{itemize}


\subsection{Successful and Challenging Examples}
\label{sec:qualitative}
%In all the version our approach out perform random approach except one version which touches 60 test case and interestingly top 3 contain within 51\%. Though our approach is below random in this case, it can eliminate 60\% test case to select top impacted test case. The overall ranking does not perform well in this change because of the reason that conditional return or throws statement  added in the top of method body is shown in the listing\ref{fig:example-2}. As a result some of test case skip the whole execution of method which spend time inside loop in the method body in earlier version. 

In this section, with representative examples of code commits, we explain why our approach performs well on some code commits but not so well on some others. 

%Due to space limit, for each code commit, we present only the relevant changes instead of the whole commit. More details about our code commit examples can be found on our project website~\cite{perfranker}.

\textbf{Successful Example 1.} Listing~\ref{list:example-p1} shows the simplified code change of code commit hash d074054... of Xalan. On this code commit, our approach is able to improve the \textit{APFD-P} and \textit{nDCG} values by at least 13.53 and 31.22, respectively, compared with the three baseline approaches. In this example, several statements are added inside a loop. Our approach can accurately estimate the performance change because (1) Line 2 in the code is an existing loop and from the profile database for the base version we can find out exactly how many times it is executed, and (2) the added method invocations are combinations of existing method invocations whose execution time is already recorded for the base version. \\

%Combining the information in our performance model, we can precisely predict the performance impact of the change for each test case.


{\fontsize{10}{10}
\begin{lstlisting}[columns=flexible,language=Java,caption=Change Inside a Loop, label={list:example-p1}]
  int nAttrs = m_avts.size();
  for (int i = (nAttrs - 1); i >= 0; i--){
    ...
+   AVT avt = (AVT) m_avts.get(i);
+   avt.fixupVariables(vnames, cstate.getGlobalsSize());
    ...
  } 
\end{lstlisting}
}


\textbf{Successful Example 2.} Listing~\ref{list:example-p2} shows the simplified code change of code commit hash 64ec535... of Xalan. On this code commit, our approach is able to improve the \textit{APFD-P} and \textit{nDCG} values by at least 17.14 and 24.33, respectively, compared with the baseline approaches. In this example, the loop at Line 4 depends on the collection variable \CodeIn{m\_prefixMappings} at Line 1 whose size can be inferred from the recorded number of iterations of existing loops. In this case, our approach can accurately estimate the iteration count of this new loop and estimate the overall performance impact based on the estimated iteration count.\\

{\fontsize{10}{10}
	\begin{lstlisting}[columns=flexible,language=Java,caption=A Newly Added Loop Correlating to an Existing Collection, label={list:example-p2}]
	+ int nDecls = m_prefixMappings.size();
	+      
	+ for (int i = 0; i < nDecls; i += 2){
	+   prefix = (String) m_prefixMappings.elementAt(i);
	+   ...
	+ }     
	\end{lstlisting}
}

%Our approach can easily identity hot path and changes that lie in the hot path from our performance model which is constructed on the call graph. Beside these any intra procedure change that related to loop can estimate impact cost except few cases. But we believe that actual improvement is more than this score because running the regression with top impacted test case is enough for exposing performance problem and the probability of selecting top impacted test case 7\% in random. 

%Collection are generally propagated through function parameter and field variable. In Listing~\ref{list:example-p3}, at line 1 introduce new array list and at line 6 this list is updated under a existing loop. We can easily estimate the length of new array list. New method introduce that is not shown here that iterate of over the new list, our technique can predict the iteration of that loop. 

%{\fontsize{7}{7}
%\begin{lstlisting}[language=Java,caption=Collection Propagation, label={list:example-5}]
%+ Collections.sort(limits);

%+ for (int i = 0; i < limits.size(); i += 2) {
%  +list.add(new Arc(limits.get(i), limits.get(i + 1)));
%+}
%\end{lstlisting}
%}

%{\fontsize{7}{7}
%	\begin{lstlisting}[language=Java,caption=Correlation New and Existing Collection, label={list:example-p3}]
%	+ List processedDefs = new ArrayList();
%	
%	for (int i = 0; i < nAttrs; i++){
%	+
%	
%	+ processedDefs.add(attrDef);
%	+
%	
%	}
%	\end{lstlisting}
%}

%\textbf{Positive Example 3.} The simplified code change of code commit of Apache common math is presented in Listing~\ref{list:example-p3}. On this code commit, our approach is able to enhance $APFD-P$ and $nDCG$ by at least 34.12, and 48.35, compared with 3 baseline approaches but it is not presented in $APFD-P$ and $nDCG$ graph because the number of test cases affected by the change are less than 10. In Listing~\ref{list:example-p3}, at Lines 3-5, elements from the collection variable \CodeIn{limits} are added to another collection variable \CodeIn{list}. In this case, our collection propagation technique is able to link the two collection variables, and propagate the impact to the loops depending on variable \CodeIn{list}. It should be noted that, the iteration number of the loop is actually half of the size of \CodeIn{limits}. In our approach, for simplicity, we ignore different operations on the index variables and simply deem the loop-iteration numbers to be the same as the size of collections it depends on. Since test prioritization does not require exact prediction of performance impact, we found our approach to be still effective with such approximations (e.g., in this example). 
%
%{\fontsize{7}{7}
%\begin{lstlisting}[language=Java,caption=Collection Propagation, label={list:example-p3}]
%+final List<Arc> list = new ArrayList<Arc>();
%+ Collections.sort(limits);
% + for (int i = 0; i < limits.size(); i += 2) {
%  +list.add(new Arc(limits.get(i), limits.get(i + 1)));
%+}
%\end{lstlisting}
%}
%
%Although our approach is almost consistently better than the two baseline approaches and the improvement is significant, we still want to investigate the cases where our approach does not perform well (e.g., the one case where our approach performs worse than Change-aware Random, and the cases where the improvement is not significant). 

\textbf{Challenging Example 1.} Listing~\ref{list:example-n1} shows the simplified code change of code commit hash 90e428d... of Apache Commons Math. On this code commit, with respect to the \textit{nDCG} metric, our approach performs better than the CAC baseline approach but slightly worse than the CAR baseline approach. In the example, at Line 2, an invocation to method \CodeIn{checkParameters()} is added, and the method may throw an exception. In this example, the execution time of \CodeIn{checkParameters()} can be easily estimated with our performance model. However, if the exception is thrown, the rest of the method will not be executed. Although we are able to estimate the execution time of the method's remaining part, it is impossible to estimate the probability of throwing the exception, as \CodeIn{checkParameters()} is a newly added method without any profile information. In such cases, if the probability of throwing the exception is higher in some test cases, the reduction of execution time due to the exception will be the dominating factor and result in inaccuracy in our prioritization.\\ 


{\fontsize{10}{10}
\begin{lstlisting}[columns=flexible,language=Java,caption=Return or Throw Exception at Beginning, label={list:example-n1}]
  protected PointVectorValuePair doOptimize(){
+   checkParameters();
    ...	
  }
+ private void checkParameters() {
+   ...
+   throw new MathUnsupportedOperationException;
+ }
\end{lstlisting}
}


\textbf{Challenging Example 2.} There are also cases where developers added a loop that is not relevant to any existing collection variables. As one of such examples, Listing~\ref{list:example-n2} shows the simplified code change of code commit hash a51119c... of Apache Commons Math. On this code commit, the improvement of our approach over the best result of the three baseline approaches is only 0.8 for \textit{APFD-P} and 6.0 for \textit{nDCG}. 
%
In the example, Line 2 introduces a new loop that does not correlate to any existing collection or array. In such a case, our approach cannot determine the iteration count of this loop and the depth of recursion at Line 8. To still provide prioritization results, our approach uses the average iteration counts of all known loops to estimate the iteration count of this new loop and always estimates the recursion depth to be 1. However, our prioritization result becomes less precise due to such coarse approximation.\\

{\fontsize{10}{10}
\begin{lstlisting}[columns=flexible,language=Java,caption= New Loop with No Correlation, label={list:example-n2}]
  public long nextLong(final long lower, final long upper){
+   while (true) {
      ...
+     if (r >= lower && r <= upper) {
+       return r;
+     }              
+   }
+   return lower + nextLong(getRan(), max);  
 }
\end{lstlisting}

}


%In some version improvement is much higher comparative to  some other version for several reason. For example,in the listing\ref{list:example-2} conditional return or throws statement  added in the top of method body introduce complex logic which is hard to determine which test case execute the full body of the method. As a result introduce imprecise impact cost. Another reason is that a new method is added with a loop in the listing \ref{list:example-1} and 


\subsection{Threats to Validity}

Major threats to internal validity are potential faults in the implementation of our approach and baseline  approaches, potential errors in computing evaluation results of various metrics, and the various factors affecting the recorded execution time for the test cases. To reduce such threats, we carefully implement and inspect all the programs, and execute the test cases for 5,000 times to reduce random noises in execution time. Major threats to external validity are that our evaluation results may be specific to the code commits and subjects studied. To reduce the threats, we evaluate our approach on both data processing/formatting software and mathematical computing software, and both a unit test suite and a performance test suite. 

%Also, we use an objective standard to select the code commits where test prioritization is most useful. To further reduce the threat, we plan to evaluate our approach on more subject projects and more code commits. 


\section{Discussion}
\label{sec:discuss}
%\subsection{Limitations of Our Approach}
%As the first step towards effective prioritization of performance test cases, our approach has a number of known limitations as follows. 

\textbf{Handling Recursions.} Recursions and loops are two major ways to execute a piece of code iteratively. Our approach constructs dependency relationships between loops and collection variables to estimate the iteration counts of loops. For recursion cycles existing in the base version, we simply update execution-time changes along the cycle only once, and multiply the execution-time change by averaging the invocation depths, attained via dividing the total execution frequency of all methods in the cycle by the product of invocation-frequency sum of these methods from outside the cycle and the accumulated recursive invocations inside the cycle (multiplying invocation frequencies along the cycle once). As an example, consider a cyclic call graph: $X \to A$, $Y \to B$, $A \to B$, and $B \to A$. When $t(B)$ is changed, the impact is propagated to $A$ as $t(A)=t(B)*f_{AB}$. The propagation then stops to break the cycle. After that, impact on X becomes $f_{XA}*depth*t(A)$ where $depth$ is the cycle's number of execution iterations, estimated as below:  
\vspace{-0.15cm}
\begin{equation}
\frac{total(A)+total(B)}{(f_{XA}+f_{YB})*f_{AB}*f_{BA}}
\end{equation}

\noindent where $total(F)$ is the total frequency of method $F$ in profile, and the divisor is the product of all calling frequencies from outside and inside of a cycle. For newly added recursions, our approach currently does not support estimation of the invocation depth, and simply assumes the invocation depth to be 1. In future work, we plan to develop analysis to estimate the termination condition of newly added recursions and relate invocation depths to collection variables.

% The reasons are 1) according to literature~\cite{JIN12}, most performance bugs are related to loops, and 2) the termination condition of recursions are more complicated than loops, and thus it is more difficult to relate invocation depths of recursions with collection variables. 

\textbf{Approximation in Performance Estimation.} Since our approach is not able to make any assumption on the given code commit, we have to make coarse approximations for the parameters of our performance model. For example, we ignore non-method-call instructions in revised methods, assume all newly added recursions to have invocation depth 1, and use the average recorded execution time of existing methods and JDK library methods to estimate their execution time in the new version. More advanced analysis can result in more accurate execution-time estimation, yet with higher overhead in the prioritization process, so future investigation is needed for the best trade-off. 

%Actually, the execution time of a method invocation may largely depend on its input and invocation context. So the estimation would be more precise if we can estimate execution time of existing methods or JDK methods as functions of method-input sizes instead of constants. While such idea is possible based on our recorded profiles, a major challenge is to estimate the size of input in a newly added/revised method. We believe that such estimation can be partly done by extending our collection-loop correlation analysis. However, even when our current approach uses relatively coarser approximations, our evaluation results demonstrate its effectiveness. 

\textbf{Supporting Multi-threaded Programs.} Multi-threaded programs are widely used for high-performance systems. In multi-threaded programs, methods and statement blocks can be executed concurrently, and thus our performance model can be inaccurate because the product of invocation frequency and average execution time of a method is no longer the total execution time. To address concurrent execution, we need to analyze the base version to find out the methods that can be executed concurrently. Then, we can give such methods a penalizing coefficient to reflect the extent of concurrency. We plan to explore this direction in  future work. 

%\subsection{Metrics for Performance Test Prioritization}
%In our evaluation, we considered 3 metrics to measure the effectiveness of performance test prioritization. As discussed in Section~\ref{}, $APFD-P$ emphasize more on the overall performance impact on test cases, while $nDCG$ is more sensitive to the ranking of most affected test cases. Top-3 Coverage considers only the top affected test cases, but is more intuitive than the other two. More investigation and user studies may be required to find out which metric is more suitable for a given scenario. Also, our approach and all of the 3 metrics weighted all test cases equal other than the code commit's performance impact on them. In practice, the importance of a test case may differ a lot depending their execution frequency in real-world systems. A 2-3\% overhead on a very frequent operation may not be acceptable, but a 10\% 
%overhead on a rarely performed operation may not matter much. So a metric closer to practice may also need to take into account execution frequency of test cases. 

%\subsection{Selection of Base Versions}
\textbf{Selection of Base Versions.}
The overhead of our approach largely depends on the required number of base versions. In our evaluation, we use one base version to estimate the subsequent code commits ranging over more than 3 years and 300 code commits. The evaluation results do not show significant effectiveness downgrade as time goes by. One potential reason is that, both software projects in our evaluation are in their stable phase and the code commits are less likely to interfere with each other. 

%
%If the software project is evolving very fast, it is possible that we need to update the base version more frequently, but in such a scenario, the benefit of test prioritization will also be larger due to the frequent code commits. It should be possible to predict the time for updating the base version by observing the downgrade of prioritization effectiveness, and such prediction technique warrants future investigation. 








	\vspace{-0.2cm}
\section{Related Work}
\label{sec:related}
	\vspace{-0.1cm}
	
\textbf{Performance Testing and Faults.} Previous work focuses on generating performance test infrastructures and test cases, such as automated performance benchmarking~\cite{KALIBERA}, model-based performance testing framework for workloads~\cite{BARNA11}, using genetic algorithms to expose performance regressions~\cite{LUO16}, learning-based performance testing~\cite{GRECHANIK12}, symbolic-execution-based load-test generation~\cite{ZHANG11}, probabilistic symbolic execution~\cite{Chen2016}, and profiling-based test generation to reach performance bottlenecks~\cite{luoinput2016}. Pradel et al.~\cite{PradelISSTA2014} propose  an approach to support generation of multi-threaded tests based on single-threaded tests. Kwon et al.~\cite{ATC2013} propose an approach to predict execution time of a given input for Android apps. Bound analyses~\cite{SPEED} try to statically estimate the upper bound of loop iterations regarding input sizes, but they cannot be directly applied as the size of collection variables under a  certain test can be difficult to determine. Most recently, Padhye and Sen~\cite{PadhyeICSE2017} propose an  approach to identify collection traversals in program code; such approach has the potential to be used for execution-time prediction. In contrast to such previous work, our approach focuses on prioritizing existing performance test cases. The most related work in this direction is done by Huang et al.~\cite{huang2014performance}, whose differences with our approach are elaborated in Section~\ref{sec:intro}. 

Another related area is research on performance faults, including studies on performance faults~\cite{JIN12, PerfBugStudy}, static performance-fault detection \cite{Nistor14, JOVIC11, KILLIAN10, YAN12}, debugging of known performance faults \cite{SHEN05,HAN12,LEUNG07,AGUILERA03}, automatic patches of performance faults \cite{Nistor15}, and analysis of performance-testing results~\cite{FOO11,FOO10}. 


%The default approach for regression testing is to retest all test cases after releasing a new version, which is an expensive proposition. To solve this problem, there are good collection of industry case studies and research effort on performance regression testing in software systems. 

\noindent\textbf{Test Prioritization and Impact Analysis.} Test prioritization is a well explored area in regression testing to reduce test cost~\cite{HARROLD93,BLACK04,ZHONG06} or to detect functional faults earlier~\cite{ELBAUM00,KIM02,LI07}. Mocking~\cite{MockStudy} is another approach to reduce test cost, but it does not work for performance testing as mocked methods do not have normal execution time. Another related area is test selection or reduction~\cite{ROTHERMEL97,CHEN94,Hao2009} which sets a threshold or other criteria to select/remove part of the test cases. Most of the proposed efforts are based on some coverage criterion for test cases, and/or impact analysis of code commits. The impact analysis falls into three categories: static change impact analysis~\cite{TURVERref,Arnoldref96,Wang2010ASE}, dynamic impact analysis
~\cite{LAW03,ORSO11,APIWATTANAPON05}, and version-history-based impact analysis~\cite{ZIMMERMANN04,SHERRIFF08,MengHima}. Our approach leverages a similar strategy to rank performance tests according to the change impact on them. However, we propose specific techniques to estimate performance impacts, such as collection-loop correlation and performance impact analysis. 

%Functional regression testing is a well explored area to reduce testing cost by test case selection based on test case property andcode modification(), test suite reduction by removing redundancy in test suite () and test cases prioritization orders test case execution in a way to hope (). Different from these work, our goal is to reduce performance regression testing overhead via test suite prioritization based on change impact analysis whether an operation is expensive or lies in hot path. 

%\noindent\textbf{Impact Analysis.} The evolution of software systems and ongoing changes demand for explicit means to assess the impact of a change on existing artifacts and concepts. Thus, software change impact analysis is in the focus of researchers in software engineering. The important difference is that Our proposed method focus on the performance test suite prioritization  via performance impact implication of change.

\section{Future Works}
\label{sec:future}
In the future, we plan to further explore the following research directions. 

First of all, our study focuses on Java software libraries, so our conclusion may not be generalized to other programming languages. Therefore, we plan to conduct similar studies on software libraries written in other languages, especially non-object-oriented languages to confirm or extend our conclusion. We also plan to inspect more backward-incompatibility-related bug reports. 

Second, as we mentioned in Section~\ref{sec:studylib}, regression testing with developers' test suite may find only a small portion behavioral incompatibilities, and thus results in a very course underestimation of the number of behavioral incompatibilities. In the future, we plan to leverage automatic test generation and more advanced automatic test oracles to better detect behavioral backward incompatibilities. 

Third, due to the difference in the popularity of API methods, the potential influence of backward incompatibilities varies. A backward incompatibility is more important if the relevant API method is used (directly or indirectly) more widely. We plan to further study the influence of behavioral incompatibilities and signature incompatibilities. 

Fourth, in our study, we find a number of challenges and research opportunities including behavioral incompatibilities related to reflections, call backs, GUI, and execution environments, better documentation of behavioral incompatibilities, etc. We plan to address some of these challenges in the future. 
	\vspace{-0.2cm}
\section{Conclusion}
\label{sec:conclusion}
	\vspace{-0.1cm}
	
In this paper, we present a novel approach to prioritizing performance test cases according to a code commit's performance impact on them. With our approach, developers can execute most-affected test cases earlier and for more times to confirm a performance regression. Our evaluation results show that our approach is able to achieve large improvement over three baseline approaches, and to cover top 3 most-performance-affected test cases within 37\% and 30\% test cases on Apache Commons Math and Xalan, respectively. 

%To further enhance our approach, we plan to work on the following directions in future. First, we plan to evaluate our approach on more subjects and code commits, especially software of different programming languages or with frequent and large commits. Secondly, we plan to develop techniques for more accurate estimation of execution time considering method inputs as factors. Thirdly, we plan to extend our performance model for multi-thread software. Fourth, we plan to perform user studies to evaluate the effectiveness of our metrics in real world settings.




\chapter{PerfRanker: Prioritization of Performance Regression Tests for Collection-Intensive Software}


\section{Introduction}
\label{sec:intro}
During software evolution, frequent code changes, often including problematic changes, may degrade software performance. For example, a study~\cite{huang2014performance} found that upgrading from MySQL 4.1 to 5.0 caused the loading time of the same web page to increase from 1 second to 20 seconds in a production e-commerce website. Even small performance degradation may result in severe consequence. For example, Google could lose 20\% traffic due to an increase of 500ms latency~\cite{Google}. Amazon could have 1\% decrease in sales due to a 100ms delay in page rendering~\cite{Stevefamov}. \\

Developers can apply systematic, continuous performance regression testing to reveal such performance regressions in early stages~\cite{foxref,poliniref,Ejref,MITCHELL,KALIBERA}. But due to its high overhead, performance regression testing is expensive to conduct frequently. Actually, the typical execution cost of popular performance benchmarks varies from tens of minutes to tens of hours~\cite{huang2014performance}, so it is impractical to run all performance test cases for each code commit. Recently, PerfScope~\cite{huang2014performance} was proposed to predict whether a code commit may significantly affect software performance and thus require performance testing. Specifically, PerfScope extracts various features from the original version and the code commit, and trains a classification model for prediction. Although PerfScope helps reduce code commits for performance regression testing, its empirical evaluation shows that a non-trivial proportion of code commits still require performance testing; thus, there is still a strong need of reducing the cost of conducting performance regression testing on a code commit, even after applying PerfScope in practice.\\

%Such problematic changes are referred to as performance regressions in this paper. and GCC from 4.3 to 4.5, Mozilla developers experienced an up to 19\% performance regression, which forced them to consider a complete switchover.Performance regression testing is a major approach to detecting performance regressions, This is widely advocated in academia, open source community and industry .for web servers is 3 minute to 1 hr, for databases is 10 minutes to 3 hrs, for compilers is 1 hr to 20 hrs and for operating systems is 2 hrs to 24 hrs . With the high revision frequency and high testing cost, 


%
%it does not consider the performance impact of code commits on individual test cases, and thus results in two limitations. First,
%

To address such strong need, developers shall prioritize performance test cases on a code commit for three main reasons. First, there can be high cost to execute all performance test cases on a code commit for large systems in practice. Second, as reported in a previous industrial study~\cite{TSEPerform} and our study in Section~\ref{sec:motivation}, various random factors may affect the observed execution time, so it typically requires a large number of repetitive executions to confirm a performance regression. Therefore, with prioritized test cases, developers can better distribute testing resources  (i.e., do more executions on test cases likely to trigger performance regressions). 
Third, a code commit may accelerate some test cases while slowing down others. It is often important for the developers to understand the performance of their software under different scenarios, while a coarse-grained commit-level technique is not helpful on this requirement. \\



%research effort by addressed this problem with PerfScope, or regressional performance testing. Specifically, for each new code commit, PerfScope applied to the code change various program analysis (e.g., hot path analysis, bound analysis, and data dependency analysis) to  to 

%For example, our study shows that it averagely requires 150 executions to confirm a 10\% performance difference, which is typically considered significant, and dynamic techniques bringing in more than 10\% overhead are typically not considered for deployment-time usage~\cite{}. 


%Mozilla’s Talos performance regression detection system~\cite{firefoxTalos} runs performance tests every time a change is pushed to the Firefox source repository~\cite{firefoxperf}. In our empirical evaluation, we observe that one code commit may affect more than 100 test cases. 


 


%There is a useful work on the performance risk implication of code change. But their approach may not accurately assess the risk of performance regression issues because of generic nature of static modeling and lacking profiling information. Furthermore, prioritizing commits is not enough to address performance regression because  a code changes touches several test cases is very common during the evolution of software. In our study, we found that some commits touches more than 100 test cases. So the key difference is that we focus  on the prioritizing test cases in regressional performance testing via performance impact analysis of code changes that includes one time profiling and static modeling.

To develop an effective test-prioritization solution for performance regression testing, we focus on \textit{collection-intensive software}, an important type of modern software whose execution time is heavily spent on loading, manipulating, and writing collections of data. Collections are widely used in software for scalable data storage and processing, and thus collection-intensive software is very common. Examples include libraries for data structures, text formating and parsing, mathematics, image processing, etc. Also, collection-intensive software is often used as components in complex systems. Moreover, a recent study~\cite{JIN12} shows that a large portion of performance bugs are related to loops, which are often used to iterate through collections. Our statistics show that 89\% and 77\% of loops iterate through collections for our two subjects Xalan and Apache Commons Math, respectively.\\

%improper iteration on collections has been identified as one of the most
%

For collection-intensive software, a straightforward approach to prioritizing performance test cases on a code commit would be measuring collection iterations (e.g., loops) impacted by the code commit and executed by each test case. However, such an approach may not be precise enough to differentiate test cases in the presence of newly added iterations, manipulations, and processing of collections, as well as their effect on existing collection iterations. Consider the simplified code example from Xalan in Listing~\ref{list:example-p3}. The code commit involves a new loop, and its location may or may not be at the hot spot for all or most test cases. Therefore, its impact on different test cases may largely depend on the different iteration counts of the added loop, the side effect of changing variable \CodeIn{list}, and the operations in the loop. Since Loop$_B$ depends on a collection variable \CodeIn{limits}, which further depends on Loop$_A$, we can infer the test-case-specific iteration count of Loop$_B$ from that of Loop$_A$; such iteration count can be acquired by profiling the base version for all test cases. Furthermore, we can infer the effect of adding \CodeIn{list.add(...)} on \CodeIn{list} with the iteration count of Loop$_B$, and update the iteration count of loops dependent on \CodeIn{list}. Moreover, we can enhance the estimation precision by using test-case-specific execution time of operations (e.g., \CodeIn{new Arc(...)}). 


{\fontsize{10}{10}
	\begin{lstlisting}[columns=flexible,language=Java,caption=Collection Loop Correlation, label={list:example-p3}]
	  while(i <= m\_size){ //Loop A
	    limits.add(new Limit(...))
	    ...
	  }
	  ...
	+ Collections.sort(limits);
	+ for (int i = 0; i < limits.size()-1; i++) { //Loop B
	+   list.add(new Arc(limits.get(i), ...));
	+ }
	\end{lstlisting}
}


These observations inspire us for three main insights to effectively model a code commit and its effect on existing collections and their iterations. First, collection sizes (e.g., \CodeIn{limits.size()}) and loop-iteration counts (e.g., Loop$_B$) can often be correlated, so collection sizes can be inferred from loop-iteration numbers and vice versa. Also, collection variables (e.g., \CodeIn{limits}) can be used as bridges to infer iteration counts of new loops (e.g., Loop$_B$) from existing loops (e.g., Loop$_A$). Second, collection manipulations (e.g., \CodeIn{list.add(...)}) are often inside loops, so the size of collections referred by collection variables (e.g. \CodeIn{list}) can be estimated from loop-iteration counts (e.g., Loop$_B$). Third, due to the large number of elements in collections, the average processing time of elements (e.g., \CodeIn{new Arc(...)}) is relatively stable, so a method's average execution time in the new version may be estimated from that in the base version.\\ 

Based on the three insights, we propose PerfRanker, which consists of four automatic steps. First, on a base version, we execute each test case in a profiling mode to collect information about the test execution, including the runtime call graph and the iteration counts of all executed loops. We also perform static analysis to capture the dependency among collection objects and loops. Second, based on the profiling information, we construct a performance model for each test case. Third, given a code commit, we estimate the execution time of each test case on the new version (formed by the code commit) by extending and revising its old performance model. We use profiling information and loop-collection correlations to infer parameters of the new performance model, and refer to this step as \textit{Performance Impact Analysis}. Fourth, we rank all the test cases based on the performance impact on them.\\

%its impact on each test case from two aspects: the execution time of the added and removed code in the revised method, and the execution time of the loops affected by the collection objects which are return values of revised methods. 




%To address the two limitations above, we propose a novel approach to prioritize individual test cases in performance regression testing via performance impact analysis, which estimates the impact of a given code revision on the execution time of a given test case. 



%our algorithm automatically takes the information in each code commit by statically analyzing the added code and removed code from automatically generated diff of each commit. Incorporating the profile and static information into each test case's call graph we determine the run-time cost and the execution frequency of the code affected by code changes. 

We implement our approach and apply it on two sets of code commits collected from popular open source collection-intensive projects: Apache Commons Math and Xalan. To measure the effectiveness of test case prioritization for a code commit in performance regression testing, we use three metrics: (1) \textit{APFD-P} (Average Percentage Fault Detected for Performance), an adapted version of the APFD metric~\cite{AlexeyAPFD} for performance testing, (2) \textit{DCG}~\cite{NDCG}, a general metric for comparing the similarity of two sequences, and (3) \textit{Top-N Percentile}, which calculates the percentage of test cases needed to be executed to cover the top N test cases whose execution time is most affected by the code commit. Our evaluation results show that, compared with the best of the three other baseline approaches, our approach achieves an average improvement of 17.6 percentage points on \textit{APFD-P} and 27.4 percentage points on DCG. Furthermore, for Apache Commons Math and Xalan, our approach is able to rank top 1 affected test case within top 8\% and top 16\% test cases, and top 3 affected test cases within top 37\% and 30\% test cases, respectively. 

%Since we are not aware of previous research efforts on prioritizing test cases for performance regression testing, we developed  for Apache Commons Math, and an improvement of 16.6 percentage points on APFD and 27.2 percentage points on DCG for Xalan, respectively

%calculated impact score is used with adaptive APFD and DCG metric formula to rank the test case which shows that they can significantly reduce the cost of performance regression. 


This paper makes the following major contributions:
  \vspace{-0.1cm}
\begin{itemize}
  \item A novel approach to prioritizing test cases in performance regression testing of collection-intensive software.
  
  
%supported by a novel technique called performance impact analysis which estimates the performance impact of code revisions on test cases.
  
%  \item A motivation study on the number of executions required to acquire stable performance testing results.
    
    
  \item Adaptation of the APFD metric to measure the result of test prioritization for performance regression testing. 
    
  \item An evaluation of our approach on real-world code commits from two popular open source collection-intensive projects. 
  
\end{itemize}
  
  
%The rest parts of this paper are organized as follows. In Section~\ref{sec:motivation}, we present a motivation study showing that multiple test executions are required to confirm a performance regression. In Section~\ref{sec:approach}, we introduce our approach and performance impact analysis in details. We present our evaluation results in Section~\ref{sec:evaluation}, and discuss some related issues in Section~\ref{sec:discuss}. Before we conclude in Section~\ref{sec:conclusion}, we introduce related research efforts in Section~\ref{sec:related}, and indicate some future work in Section~\ref{sec:future}.
\section{Motivation}
\label{sec:motivation}

%As mentioned in the introduction, besides the fact that performance test cases often involves larger inputs and thus consumes more resources and take more time, the performance variance among different executions of a same test case provides us an even stronger motivation for test prioritization in performance regression testing. 

%Theoretically, for each test case that does not involve random process (e.g., generating random numbers, randomly selecting next method invocations), all of its executions should go over exactly the same instruction sequence, and thus take exactly the same time. 

\begin{figure}
	\centering
	\includegraphics[width=0.45\textwidth]{performance/images/standard-deviation.pdf}
	\caption{Relative Standard Deviation vs. Sample Size}	
	\label{fig:motiv-math}	
	
\end{figure}

In this section, we provide preliminary study results to motivate prioritization of performance regression tests, due to high cost of executing the same test case for many times for performance regression testing. In particular, modern mechanisms in hardware and software often bring in random factors impacting performance. Some well-known examples are the randomness in scheduling cores and buses in multi-core systems~\cite{MultiCoreRandom}, in caching policies~\cite{CacheRandom}, and in the garbage-collection process~\cite{GBRandom}, etc. These factors interact with each other and amplify their effect so that the execution time of a test case may vary substantially from time to time. 

To neutralize such randomness, researchers or developers execute a test case multiple times and calculate the average performance~\cite{LiuCGO}. To better understand this requirement, we perform a motivating study (with more details on the project website~\cite{perfranker}) on the two open source projects used in evaluating our approach. In particular, we execute the test cases for 5,000 times, and randomly select samples with different sizes to calculate their standard deviation. In Figure~\ref{fig:motiv-math}, we show how the execution time's relative standard deviation (y axis) changes as the execution times (x axis) increase from 1 to 5,000. The figure shows that the average relative standard deviation with sample size 1 is over 20\% in both projects. In other words, if only one execution is used, the recorded execution time is expected to have more than 20\% difference from the average execution time in the 5,000 executions. Note that 20\% is a very large variance because 10\% performance enhancement is typically considered significant, and techniques incurring over 10\% overhead are often considered too slow for deployment purposes~\cite{DoubleTake}. The figure also shows that 173 and 47 executions are required to achieve relative standard deviation less than 10\% for Apache Commons Math and Xalan, respectively. Executing the whole test suite for many times can be prohibitively expensive, calling for prioritization of performance regression tests, as targeted by our approach. 


%The number of executions to acquire precise performance measurement may vary from scenario to scenario, and there is not a comprehensive study for it. In this paper, to understand how many executions are required to achieve precise result for our evaluation settings (i.e., performance testing of Java Projects on a state-of-the-art multi-core server), we performed a motivation study on the two test suite of our evaluation subjects: Apache Commons Maths and Xalan. 

%\textbf{Study Design.} For each project, we execute the test suite for 5,000 times on a quiet\footnote{``Quiet'' means that no other user process is running on the server.} DELL X630 server with 32 cores and 256GB memory. From the executions, for each test case, we recorded its execution time as a sequence with 5,000 data points\footnote{To avoid potential noises such as test-case loading or IO blocking, we removed the outliers that are beyond 2 times of standard deviation, which is a standard data purification process~\cite{}, and removed 0.9\% data points on average.}. In our study, we define ~\textit{Sample Size} as the number of iterations a test case is executed, and ~\textit{Sample Performance} as the average execution time calculated from the executions. For a small sample size (e.g., 1), the sample performance from different samples tends to have larger standard deviation, and when the sample size gets larger, the standard deviation among samples tends to get smaller, and our goal is to find the sample size that can lead to small enough standard deviation. 
%
%For each sample size $i$, we use all the sub-sequences with length $i$ of the original data-point sequence whose length is $N$\footnote{The value of $N$ is slightly smaller than 5,000 due to data purification.}
%
%Specifically, if the $k^{th}$ data point in the original data-point sequence is denoted as $ex(k)$, the first sample, the $k^{th}$ sample, and the last sample in the set will be:
%
%\begin{multline}
%sample(i, 1) = \{ex(1), ex(2), ...., ex(i)\}\\
%sample(i, k) = \{ex(k), ex(k+1$ $mod$ $N), ...., ex((k+i)$ $mod$ $N)\}\\
%sample(i, N) = \{ex(N), ex(1), ...., ex(i-1)\}\\
%\end{multline}
%
%Therefore, our sample set for sample size $i$ can be viewed as executing the test case for $i$ times from any start point in the original execution sequence. Based on these sample sets, if we denote the sample performance of $sample(i, k)$ as $perf(i, k)$, the relative standard deviation for sample size $i$ can be calculated as follows. 
%
%\begin{equation}
%Var(i) = \frac{Std(\cup{k=1}{N}{\{perf(i, k)\}\}})}{\Sigma{k=1}{N}{ex(k)}/N}
%\end{equation}
%
%In the formula, to normalize the standard deviation for different test cases and projects, we divide it with the average execution time of the test case for all $N$ executions. Therefore, we actually use the average execution time of the 5,000 executions to approximate the ideally precise execution time. It should be noted that, 5,000 is much larger than 10 to 100 which are common used in performance related research efforts in literature~\cite{}~\cite{}, so we believe that the approximation should be reasonably accurate.  For the sample sizes from 1 to 5000, we draw the trends of relative standard deviation (average for all test cases)
%\textbf{Study Results.} 


% 1,580 and 406 executions are required to achieve relative standard deviation less than 5\%. 



%in which horizontal axes represent sample sizes in log scale and vertical axes represent the relative standard deviation. 


%It should be noted that, our study has some limitations such as the purification of execution-time outliers which may exist in the real world testing, and the fact that sample sets are not independent with each other because there are overlaps between neighboring sample sets (e.g., $sample(i, 1)$ and $sample(i, 2))$. However, these two limitations will both reduce the observed relative standard deviation, indicating that we actually provided a lower bound of relative standard deviation over sample sizes. Therefore, as a motivation study, it show that we do need several hundred executions to confirm a performance regression around 10\%. Based on this fact, the prioritization of test cases is very helpful, because we can focus the highly prioritized test cases, and execute them for more times to better confirm perform regressions. 





%We can see that at first iteration average deviation from ideal is 20.79\% and reduced to 5\% with 1580 iteration. Finally the overall variance decreased to 1\% with iteration 4200. Furthermore, deviation decreases sharply from 20\% to 10\% with 173 times of execution whereas reduction from 10\% to 5\% take 1407 iteration. So we can say that large number of iteration of execution of test case are needed to get a stable execution time of a performance test case. A small extent of performance degradation may result in severe consequence motivate us that running performance test needs more time and resources. It is hard for a developer to understand performance impact with few runs.\\


%In the evaluation of techniques for compilers and systems, to reduce some randomness, researchers sometimes choose to turn off some features, such as doing garbage collection at fixed time point. However, 

%{\fontsize{7}{7}
%\begin{multline}
%\begin{gathered}
%Average = Avg(1,2,3,...,5000)\\
%Var(i) = \frac{ Std( Avg(1,2,...,i),...,Avg(5000,1,...,i-1))}{Average}\\
%Iteration(i) = \frac{ \sum \limits_{k=1}^{alltest} Var(i)}{alltest}
%\label{eq:eq-motivation}
%\end{gathered}
%\end{multline}
%}





%We run real test cases and captured average execution time of two popular open source project apache common math and xalan. \textbf{Common math}:  We chose 2703 test cases which are affected by code changes from a collection 3900 test cases and each of test cases  exercise 5000 time and record their execution time. In the formula ~\ref{eq:eq-motivation} function Avg and Std corresponding to average and standard deviation. For iteration-1, it becomes {\fontsize{7}{7}\[ Var(1) = \frac{Std(1,2,3,...,5000)}{Average}\]} which is fraction of standard derivation of 5000 data and average of 5000 data. The value of Var(1) represents how far the deviation is from the ideal case with a single iteration.For iteration-2, it becomes {\fontsize{7}{7} \[Var(2) = \frac{ Std( Avg(1,2), Avg(2,3),...,Avg(5000,1))}{Average}\]} where each data of standard deviation is average of 2 data which means that 2-iteration on the average how much deviate from ideal case. For iteration-5000, it becomes {\fontsize{7}{7}\[Var(5000) = \frac{ Std(Avg(1,2,...,5000),...,Avg(5000,1,...,4999))}{Average}\]} where each data of standard deviation is average of 5000 data which means Var(5000) value equals to zero that we considered as ideal or base case. According to the formula~\ref{eq:eq-motivation}, we generate a graph in the figure ~\ref{fig:common-math-motiv} 

\section{Approach}
\label{sec:approach}

In this section, we introduce our test prioritization approach in detail. In particular, we first present the overview of our approach, major technical challenges, and our performance model. After that, we introduce performance-impact estimation of a code commit based on our performance model, including method-execution-time estimation, and method-invocation-frequency estimation based on collection-loop correlation and iteration-count inference.


\subsection{Overview}

%To prioritize performance test cases, the core difficult

%The objective of PerfChecker is to examine a code commit content and determine the cost of a change, and then rank the performance test case on basis of a change performance impact cost which helps to reduce the cost in performance regression testing. Profiler takes the base version code and statically modifies the code in order to extract profile information. Run all the performance test case and store method execution time, execution frequency and loop counter in profile database.Profile database also store JDK library method's execution summary. 

\begin{figure}
\centering
	
	\includegraphics[width=3.5in, height=3.5in]{performance/images/Workflow2.pdf}
	\caption{Workflow of Our Approach}	
	\label{fig:approach-workflow}
		
\end{figure}

Figure~\ref{fig:approach-workflow} shows the workflow of our approach. The input to our approach includes the project code base, its code commits, and performance test cases. The output of our approach is an ordered list of performance test cases. The first step of our approach is to  profile the base version of the project under test. During the profiling, for each test case, we record the dynamic call graph as its original performance model. We also record average execution time and frequency of methods, and iteration counts of all loops. At the same time, we statically analyze the base version to gather dependencies between loops and collection variables, as well as aliases among collection variables. When a new code commit comes, we conduct performance impact analysis to estimate its performance impact on all test cases, and prioritize test cases accordingly. 


%Another component is Alias analyzer which statically analyze the source code loop controlling variable, array and collection with their alias and store their information in database. Diff generator produce diff of each committed code into the repository. Parser parses information regarding the changed files, lines and types (add, delete) from the generated diff file content. Filter prunes out insignificant change such as stylish change or renaming. If the changes are significant, the commit will be considered to fed to the performance impact analyzer. Performance impact analyzer analyse the cost of the change by constructing CFG of the modified method and integrating the profile information from the database into the call graph of the test case. At the final stage it produce a sorted list of test case according the cost performance impact score. 



 
 
%The challenge is how to estimate the cost about performance impact of each of test case after a change without actually running the software.We first briefly describe the input-output of our approach and major component. 

%Although it is straightforward to prioritize test cases based on the code commit's performance impact on them, 

\textbf{Technical Challenges.} Performance impact analysis is the core of our approach. Although we focus on collection-intensive software, it is still challenging to conduct performance impact analysis, facing three major technical challenges:

\begin{itemize}

\item Challenge 1. A code commit may include any type and scope of code changes, from one-line revision, to feature addition and interface revision. Therefore, there is a strong need of a unified and formal presentation for code commits.  

\item Challenge 2. A code commit may contain newly added code, especially new loops. No execution information of such code is available, but given that loops can have high impact on performance, there is a strong need of estimating the code commit's execution time and frequency.
 
\item Challenge 3. Even if the execution time of changed code in a code commit has little impact on performance, the code commit may include changes on collection variables, eventually affecting the performance of unchanged code. 

%In our paper, we refer such performance changes as \textit{side performance impact}. 

\end{itemize}

To address Challenge 1, we present a code commit as three sets of methods: added methods, revised methods, and removed methods. As our performance model is based on the dynamic call graph, any code commit can be mapped to a series of operations for method addition, removal, and replacement in the performance model. To address Challenge 2, we leverage the recorded profiling information of the base version as much as possible. Specifically, if an existing method is invoked in the newly added code, we can use the recorded execution time of the existing method as its execution-time estimation for this new invocation. Furthermore, as discussed in Section~\ref{sec:intro}, we use collection variables as bridges to estimate iteration counts of new loops from those of existing loops. To address Challenge 3, we track all the element-addition and element-removal operations of collection variables in the newly added code, and estimate the size change of collection variables from the iteration count of their enclosing loops. This new size is used to update the iteration counts of loops depending on the changed collection variables. 

%Note that, the execution time of added and removed methods can affect existing test executions only through revised methods, where invocations to them are added and removed. %For each revised method, based on the performance model in Section~\ref{subsec:model}
%we estimate the execution time of the pre-commit version and post-commit version, respectively, and estimate the performance impact of a revised method as the execution-time difference between its pre-commit version and post-commit version. Then, the directly performance impact of a code commit can be calculated as the performance impact sum of all revised methods, together with side performance impact. 



%and if a newly added loop depends on an existing collection variable, 

%we can use the recorded size of the collection variable to estimate the number of iterations of the loop. The detailed process of estimating direct performance impact is presented in Section~\ref{subsec:direct}. 



%their new value ranges in the new version according to the iteration number of the writing operations. Then, for each collection variable $v$, if there is any loop $l$ depending on $v$, we calculate the side performance impact of $l$ with $v$'s new value range and sum up side performance impact of all such loops. The detailed process of estimating side performance impact is presented in Section~\ref{subsec:side}.

Since we focus on collection-intensive software, we consider only loops whose iteration number depends on collection variables, e.g., variables of array type, and other collection types defined in Java Utility Collections. Note that there are also some loops whose iteration number depends on simple integers, such as a loop to sum up a numbers from $i$ to $j$, but such loops are not common in collection-intensive software, and our evaluation results show that our approach is effective on both data-processing software (Xalan) an mathematics software (Apache Commons Maths). 

%The rationale behind this design decision are (1) according to literatures~\cite{}~\cite{}, most performance issues are related to loops; and (2) most loops in programs are for manipulating iterative data structures (i.e., collection variables).  


%This is a limitation of our performance impact analysis, b ut 
  
\subsection{Performance Model}
\label{subsec:model}
\begin{figure}[t]
\centering
  \includegraphics[width=4in, height=3in]{performance/images/performance-model.pdf}
  
   \caption{An Example Performance Model}	
    \label{fig:performance-model}
   
 \end{figure}

In this subsection, we introduce our performance model to break  down execution time of a test case to all the methods invoked by the test case. The basic intuition behind our model is that the execution time of a method invocation $M$ is the execution-time sum of all method invocations directly invoked in $M$, together with the execution time of instructions in $M$. Since most basic operations in Java programs are performed by JDK library methods (e.g., a string concatenation), the latter part is typically trivial compared with the former part, so our performance model ignores instructions in $M$ itself, but focuses only on methods that $M$ invokes. 

%Specifically, for each test case, our performance model is based on the dynamic call graph of the test case. So the nodes of the graph are method bodies, and the edges are invocations. In the graph, we record two attributes: invocation frequency, and average execution time. For example, on the directed edge from method $A$ to method $B$, we record the average number of invocations from $A$ to $B$ in each invocation of $A$, and on the node $B$, we record the average execution time of $B$.  

We illustrate our model in Figure~\ref{fig:performance-model}, where each node represents a method and each directed edge represents an invocation relation. Here each node annotated with label $t_{avg}$, which represents the average execution time of a method. Each edge in the graph is labeled with $f_{AB}$, which represents the invocation frequency of method B from method A. Given a code commit, the performance model of the post-commit version can be acquired by adding and removing nodes and edges to the original performance model. For example, in Figure~\ref{fig:performance-model}, method $D$ is a revised method and it now calls (1) method $G$, which it originally calls, (2) $E$, which is an existing method in the base version, and (3) $H$, which is a newly added method. With the average execution time and invocation frequency of all invoked methods in $D$, we are able to calculate an execution-time estimation for revised method $D$. The new execution time at $D$ can be propagated upward to its ancestors, until the main method is reached and a new estimation of the whole program's execution time can be made. 


% $m\textsubscript{K}$ are changed and they are called from multiple context $m\textsubscript{B}$,$m\textsubscript{G}$  and $m\textsubscript{H}$. So the total cost is calculated from the modification cost of method $m\textsubscript{J}$  and $m\textsubscript{K}$ by multiplying the frequency  $f\textsubscript{BJ}$,$f\textsubscript{GJ}$,$f\textsubscript{HK}$ and $f\textsubscript{GK}$. Finally, The cost is propagated from child to parent in the graph.

\subsection{Performance Impact Analysis}
\label{subsec:direct}

The basic idea of performance impact analysis is to calculate the  execution-time change of each revised method $M$ in the code commit. Then, through propagating the execution-time change to $M$'s predecessors in the performance model, we can calculate the execution-time change of the whole test case at the root node. 

%We focus only on revised methods because added and removed methods may affect software performance only through revised methods (i.e., by adding or removing method invocations).

We realize this idea in three steps. First, for each revised method, we extend the performance model to either add it and/or some of its callees (and transitive callees). Second, we estimate the execution-time change of each method in the new performance model. Third, we estimate the invocation frequency on the edges of the new performance model. We next introduce the three steps in details.



\subsubsection{Model Extension}
\label{subsub:findInvoke}

For each revised method, we add its direct and transitive callees (e.g., methods $E$ through $K$ in Figure~\ref{fig:performance-model} for the revised method $m_D$) into the performance model, if they do not already exist in the model. In this recursive process, we terminate the extension of a method node if it is an unrevised method existing in the base version, a JDK library method, or a method whose source code is not available. Since we use one base version for a series of code commits, a revised method (and even some of its predecessors) may not exist in the base version because they are added after the base version. In such a case, we transitively determine $M$'s predecessors (callers) until we reach methods in the base version. For example, if $C$ and $D$ in Figure~\ref{fig:performance-model} are added between the base version and the code commit under analysis, we determine that $D$ is invoked by $B$ and $C$, and $C$ is invoked by $A$, so that we add $C$ and $D$ to the performance model. 

When the new code version of the revised methods is available, we  
statically determine the direct and transitive callers and callees for the revised method, and one remaining challenge is to resolve polymorphism, where one method invocation may have multiple targeted method bodies. Although it is straightforward to apply off-the-shelf points-to analysis, since we have the profiling information of the base version, we make use of the information to acquire a more precise call graph. Specifically, if a method invocation is not involved in the method diff (i.e., the method invocation can be mapped to the same method invocation in the base version, such as $G$ in Figure~\ref{fig:performance-model}), we assume that its target is not changed and we use the same targeted method body as recorded for the base version. Otherwise, we apply points-to analysis~\cite{Spark} in Soot~\cite{Soot} to find the possible targeted method bodies for the method invocation. When a method invocation is mapped to multiple method bodies, we add all bodies to the new performance model, and we divide the estimated frequency of the method invocation by the number of possible targets to attain the invocation frequency of each target. 

%Analyzing the new code version of the revised method, it is straightforward to build a partial call graph with it as the root node. Since  resolve polymorphism, we use the following heuristic. 

%Therefore, by listing all the added and deleted methods and their execution time summary will provide an estimate of 

%So combining the profiling information and control flow analysis together may provide accurate estimate of cost of modified method.

%Estimate the cost of a modified method need to identify all the changes inside the method which are added and deleted.




\subsubsection{Execution-Time Change of Method Bodies}
\label{subsub:findExetime}

The method bodies invoked from a revised method fall into three  categories. The first category is removed method bodies. Their execution-time data are recorded in the performance model of the base version, and their new execution time is estimated as 0.

The second category includes method bodies already existing in the performance model of the base version. Such method bodies include both those defined in the source code and those defined in the JDK library, or those without source code. For existing method bodies, we simply use the recorded average execution time in the base version as their estimated average execution time. For bodies of JDK library methods, we profile Dacapo~\cite{dacapo} to acquire the average execution time of those common JDK library methods. For methods without source code or those not invoked by Dacapo, we use the average execution time of all method bodies in the profile as their estimated execution time, as we have no further information. Note that when a method from the second category is added to the performance model, its original execution time is set as 0.

The third category includes newly added method bodies in the source code. \textit{Note that such method bodies include both those added to the source code in the code commit and those added in any other code commits between the base version and the code commit under analysis.} They also include method bodies defined in libraries but are newly reached due to code revisions from the base version to the new version. For a newly added method body (such as $H$ in Figure~\ref{fig:performance-model}), as discussed in Section~\ref{subsub:findInvoke}, we extract all its callee method bodies, and add them to the performance model (such as $J$ and $K$ in Figure~\ref{fig:performance-model}), and then we calculate its execution-time change using our performance model.

%If an added callee method body still belongs to category-2, then we can iteratively extract its callee method bodies. The extraction process will terminate when all the newly added callee method body either belongs to category-1, or is defined in Java SDK. 



%Changes inside a method normally consists of addition and deletion of existing method call or addition of newly method call. 
%Profile all the performance test case and store their average execution time  will provide the cost of existing method call.
%A newly added method also consists of existing method, JDK library call and newly added method. So profiling JDK library function call
%also helpful to estimate the cost of newly added method. if no information available then statically analyse the change and assign the cost based on the existence of loop and hot api call like database, storage and network.
 
\subsubsection{Invocation Frequencies of Method Bodies}

Given a newly added or revised method $m_x$, for each method $m_y$ that is directly invoked in $m_x$, we estimate the invocation frequency of $m_y$ in $m_x$ (denoted as $fq(m_x, m_y)$) to apply our performance model on the new version. Our estimation technique is based on the control flow graph of $m_x$ and the average iteration counts of loops in $m_x$. Specifically, for any code block $b$ in $m_x$, we use $fq(b, m_y)$ to denote the invocation frequency of $m_y$ from $b$ for each execution of $b$. If $b$ is a basic block without branches and loops, $fq(b, m_y)$ is exactly the number of invocation statements to $m_y$ in $b$; such number can be easily counted statically. Then we calculate $fq(m_x, m_y)$ by applying the inference rules for sequential, branch, and loop structures in Formulas~\ref{equa:seq}-\ref{equa:loop} below recursively on the code blocks of $m_x$:

   
\begin{equation}
\label{equa:seq}
fq([\text{$b_1$; $b_2$}], m_y) = fq(b_1, m_y) + fq(b_2, m_y)
\end{equation}
\begin{equation}
\label{equa:branch}
fq([\text{if() $b_1$ else $b_2$}], m_y) = Max(fq(b_1, m_y), fq(b_2, m_y))
\end{equation}
\begin{equation}
\label{equa:loop}
fq([\text{while$_i$ () $b$}]) = fq(b, m_y) \times C(loop_i)
\end{equation}

In the inference rules, the only unknown parameter is $C(loop_i)$, which denotes the average iteration count of the i$^{th}$ loop in $A$. As an example, given the control flow graph in Figure~\ref{fig:cfg-sample} of m$_x$, for any $m_y$ that $m_x$ invokes, $fq(m_x, m_y)$ can be estimated as in Formula~\ref{equa:example}.

\begin{figure}
\centering
	\includegraphics[width=3.5in, height=1in]{performance/images/cfg-new.pdf}
	
	\caption{An Example Control Flow Graph}	
	
	\label{fig:cfg-sample}
\end{figure}


%Ifwe try to estimate $freq(A, B)$ as a function of average iterations in $A$

%Generally, a change set may introduce performance regression in two ways: one case is the change itself in the hot path of execution and another case is that changes modifies some control variables which result some part of code executing very frequently. We can estimate how frequently the method executing in two ways: Loop correlation and Collection propagation. %Generally the changes inside a method does not lie in a single execution path. Control flow analysis of a modified method helps to identify all the paths inside the method and the changes in the path. 

   
\begin{multline}
\label{equa:example}
fq(m_x), m_y)=fq(A, m_y) + (Max(fq(D, m_y), fq(E, m_y))\\ + fq(F, m_y)) \times C(Loop_B) + fq(G, m_y)
\end{multline}
\vspace{-0.5cm}

%shown a simple CFG of a change of a modified method where each node represents a block of CFG and each annotated with cost which is sum of all the method execution time inside the block. There are two path from node B to node F.  $Cost\textsubscript{BDF}$ is the cost of path $BCDF$ and $Cost\textsubscript{BEF}$ is the cost of path $BCEF$. So the maximum cost will the maximum of this two path multiplied by the loop counter which is the execution time of changes in the method.


 
%\subsection{How to decide How many times a method will execute}
%Generally, a change set may introduce performance regression in two ways: one case is the change itself in the
%hot path of execution and another case is that changes modifies some control %variables which result some part of code
%executing very frequently. We can estimate how frequently the method executing in two ways: Loop correlation and Collection propagation.  

\subsection{Loop-Count Estimation with Collection-Loop Correlation} 

With the estimation of invocation frequency, the only remaining unknown parameter in the performance model of the new version is the loop count of all loops. If a loop exists in the base version and is not affected by the code commit, we directly use the recorded profile from the base version to acquire the iteration count. Two more complicated cases are (1) when a new loop is added, and (2) when the code commit affects the iteration count of an existing loop. Here is our insight: for collection-intensive software, we can construct the correlation between collection sizes and loop counts, and use iteration counts of known loops to infer that of unknown loops, as well as a code commit's impact on iteration counts of known loops. 

\subsubsection{Correlating Loops and Collections} In particular, we consider the following two types of dependencies between loops and collection variables:

\begin{itemize}
	\item Iteration Dependency. A loop $L$ is iteration-dependent on a collection variable $v$ if $L$'s loop condition depends on the size attribute of $v$.
	\item Operation Dependency. A collection variable $v$ is operation-dependent on a loop $L$ if there exists an element addition or removal on $v$ in $L$.
	\vspace{-0.15cm}
\end{itemize}

To identify iteration dependencies, for a For-Each loop (e.g, \CodeIn{for(A a : ListOfA)}), we simply consider that the loop is iteration-dependent on the collection variable being iterated via the loop. For other loops, we use standard inter-procedural data flow analysis~\cite{SootIFDS}~\cite{IFDS} to track data dependency backward from the loop condition expression, until we reach a size/length attribute of an array or a known collection class from the Java Collection Library. To make sure that the collection size is comparable with the loop count, we consider only two types of data dependencies: (1) direct assignment (e.g., \CodeIn{a = b;}), and (2) addition or subtraction expression with one operand as constant (e.g., \CodeIn{a = b.size() - 1}). 

To identify operation dependencies, for each loop $L$, we check its body for element-addition and element-removal operations on collection variables. For any other method invocations in the loop, we recursively go into the body of each invoked method to further look for such operations. However, we do not consider nested-loop blocks in $L$ or a method invoked from $L$, because such blocks are dependent on their direct enclosing loop. For example, in Figure~\ref{fig:operationDepend}, collection variable $v$ is operation-dependent on Loop$_A$, but $w$ is not (it is operation-dependent on Loop$_B$). 

\begin{figure}
\centering
	\includegraphics[width=3.5in, height=1.3in]{performance/images/operationDepend.pdf}
	
	\caption{Code Sample of Operation Dependency}	
		
	\label{fig:operationDepend}
\end{figure}




After identifying these two types of dependency relations, we further apply points-to analysis~\cite{Spark} to identify alias relations among collection variables. Note that in our analyses we consider only variables of known collection classes from the Java Collection Library. User-defined collections are also common. However, since we use inter-procedural analysis for identifying both types of dependencies, as long as the user-defined collections extend or wrap Java-Collection classes from the Java Collection Library, we are able to handle these user-defined collections by building dependencies directly on the Java-Collection variables inside them. Also note that  we identify all dependencies and alias relations on the base version and record the results so that we need to re-analyze only the revised/added methods when a code commit comes. 

\vspace{-0.2cm}
\subsubsection{Iteration-Count Inference}


With the dependencies identified among collections and loops, when a new code commit comes, we use Algorithm~\ref{alg:infer} to infer the iteration count of new loops and update the iteration count of existing affected loops. 

In the algorithm, we use a work queue to iteratively update sizes of collection variables and loop-iteration counts (stored in $lCount$), and we use the combined map of $MapI$ and $MapO$ to transit between collections and loops. In particular, as shown in Lines 6-11, we update the iteration count of each loop at most once, to avoid infinite update process caused by cyclic dependency (the more in-depth reason is that we use numbers to represent iteration counts, which are not in a bounded domain). In the end, we remove collection variables from $lCount$ to retain only the loops in the map.

\setlength{\textfloatsep}{6pt}
\begin{algorithm}[t]
	\begin{algorithmic}[1]		
		\REQUIRE~~\\
		$MapO$ is a map from collection variables to loops\\
		$MapI$ is a map from loops to collection variables\\
		$lCount$ is a map from loops to iteration counts\\
		\ENSURE~~\\ updated $lCount$\\
		\STATE{$Q \leftarrow MapI.keys()$}
		\STATE{$Map \leftarrow MapO \bigcup MapI$}
		\WHILE{$Q \neq \emptyset$}
		\STATE{$top \leftarrow Q.pop()$}
		\FORALL{$val \in Map.get(top)$}
		\IF{$val \notin lCount.keys() $}
		\STATE{$lCount.add(val, lCount.get(top)$}
		\STATE{$Q.add(val)$}
		\ELSIF{$val$ is a collection variable}
		\STATE{$lCount.set(val, lCount.get(val) + lCount.get(top)$}
		\STATE{$Q.add(val)$}
		\ENDIF
		\ENDFOR
		\ENDWHILE
		\STATE{$lCount.removeAll(MapO.keys())$}
	\end{algorithmic}
	\caption{Iteration Count Inference}
	\label{alg:infer}
\end{algorithm}

\vspace{-0.2cm}
\subsection{Test Case Prioritization}
\vspace{-0.1cm}
Once our approach estimates the performance impact of the code commit on each test case, we can rank test cases according to their relative performance impact. We use the main method as the root for system tests and each test method as the root for unit tests. We consider both positive and negative effect on execute time as it is often also important for developers to understand whether and where their commit is able to enhance the software performance. 


%In the figure ~\ref{fig:approach_overviw_1}, changes in the method named as getNameSpaceURI is trivial to developer because the indirect cost is not visible to developer. We can see that grand parent in the calling stack has loop in nested which execute more that 8000 time in both case. As a result the changes happen in hot path introduce high cost. Furthermore, developer may not be aware of another loop in path which is a child method in calling stack. As a result average counter of this loop in each test case more than 144000 times which introduce performance regression. To address this problem we introduce call graph with calling context that will provide estimation how many times the modified method getNameSpaceURI could be called from parent. if we know the frequency of the method getNameSpaceURI in the context then it is easy to say how many time the newly added method will be invoked. Similarly the newly added child has a loop that correlated with array size provide the context how many time the methods inside loop will be executed. 

%\begin{figure}
%  \centering
%  \includegraphics[width=\columnwidth,height=3in]{images/call-graph-example.pdf}
%  \caption{Overview of a change call graph in Xalan}	
%  \label{fig:approach_overviw_1}
%\end{figure}

%\subsubsection{Collection Propagation}
%The most common nature of loop in source code is iterate of over array or collections and changes inside loop introduce cost. 
%Generally, collection or array are propagated two ways: parameter passing and field variable. In the figure ~\ref{fig:approach_overviw_2} we can see that three different methods has loop that depends on the size of ArrayList and most
%importantly they are alias. if one of method loop counter is known then any new invocation of other method's loop counter can be predicted which will be helpful to estimate how many times the method under a loop will be called. So static analysis and alias analysis are important way to the solution of the problem.
%
%\begin{figure}
% \centering
%  \includegraphics[width=\columnwidth,height=3in]{images/Alias-example.pdf}
%	\caption{Overview of a alias in Xalan}	  
%  \label{fig:approach_overviw_2}
%\end{figure}
%
%
%\begin{algorithm}
%\SetAlgoLined
%\SetKwInOut{Input}{Input}\SetKwInOut{Output}{Output}
%
%\Input{git version1 head and version2 head}
%\Output{List Added, List Deleted}
%
%\BlankLine
%
%$fileDiff$ = git diff [--options] version1 version2\;
%$listDiff$ = Diffparser($fileDiff$)\;
%$listDiff$ = Filter($listDiff$)\;
%$listAdded$ =[] \;
%$listDeleted$ =[] \;
%\ForEach{each diff $d$ in $listDiff$}{
%  
%	\uIf{$added$}{
%  add tuple $<$class, method, startLine, endLine$>$ to $listAdded$\;
%	}\uElseIf{$deletd$}{
%   add tuple $<$class, method, startLine, endLine$>$ to $listDeleted$\;
%  }\Else{
%    do nothing\;  
%  }
%
%}
%
%\Return{[$listAdded$,$listDeleted$]}
%
%\caption{Change List Generation}
%\label{alg:the_alg_1}
%\end{algorithm}
%
%
%\begin{algorithm}
%\SetAlgoLined
%\SetKwInOut{Input}{Input}\SetKwInOut{Output}{Output}
%
%\Input{$listAdded$,$listDeleted$}
%\Output{Rank of Test suite}
%
%\BlankLine
%
%$listTest$ = getAllTestSuite()\;
%$globalSummary$ = GlobalSummary()\;
%$mapCost$ = $<$test$,$cost$>$ \;
%\ForEach{each Test $t$ in $listTest$}{
%
%  $callGraph$ = LoadCallGraph($t$)\;
%  $localSummary$ = LocalSummary($t$)\;
%  $loopCounter$ = LoopCounter($t$)\;
%	$cost$ = 0;\;
%	
%	\uIf{$listAdded$ contain in $callGraph$}{
%	     buildCFG($listAdded$)\;
%	     cost += estimate addition cost of change from loop counter, local and global method execution summary\;       
%	}\uElseIf{$listDeleted$ contain in $callGraph$}{
%	     buildCFG($listDeleted$)\;
%       cost -= estimate deletion cost of change from loop counter, local and global method execution summary\;     
%	}\Else{
%    do nothing\;  
%  }
%  
%  update $mapCost$\;
%
%}
%
%$mapScore$ = $<$test$,$score$>$ \;
%\ForEach{each Test $t$ in $mapCost$}{
%   
%   $score$ = ($oldTime$ + $cost$)/ $oldTime$ \;
%   update $mapScore$\;
%    
%}
%
%
%$rankedList$ = runRankingAlgo($mapScore$)\;
%
%\Return{[rankedList]}
%
%\caption{Change Impact Performance Analysis}
%\label{alg:the_alg_2}
%\end{algorithm}

%\subsection{Collection-aware Performance Impact Analysis}
%
%Our change impact performance analysis algorithm \ref{alg:the_alg} takes two git head version of source code and automatically generate
%the diff between the two version. Statically analyse the generated diff and extract the list of methods added and deleted in the change source code with start and end line of a change inside a method. In the figure \ref{fig:approach-overviw-3} a modified method simple 
%CFG is shown. Maximum path among the two path is multiplied by loop counter consider as the cost impact. For each test case we calculate total impact cost according to the algorithm form line 17 to 30. Finally return the ranked list of test suite on the basis of
%performance impact. The figure \ref{fig:approach-workflow} describe the complete workflow of our approach.
%
%Our approach divided into two stage: generate change list and change impact analysis. Our change list generation algorithm\ref{alg:the_alg_1} takes two git head version of source code and automatically generate the diff between the two version. From line 1 to 3 describe the process of diff generation, parsing and filtering. First it statically analyse the generated diff contain and filter out the insignificant changes and other non non-relevant changes like java doc, comments etc. From line 6 to 14 describe the process of   extracting list of methods added and deleted in the change source code with start and end line number of a change inside a method. Finally return the list of algorithm\ref{alg:the_alg_1} feed to Our change impact analysis algorithm\ref{alg:the_alg_2}.
%From line 4 to 19 it describe the process of change list impact estimation. To estimating the cost it first build CFG of change method and find out the changes in the path of CFG. Then assign cost to each of the block from profile database which is describe in the figure \ref{fig:approach-overviw-3}. After estimating the cost of a method change, total impact of cost is calculated from the call graph which is  describe in the figure \ref{fig:performance-model}.Finally line 20 to 26 describe the return the ranked list of test suite on the basis of performance impact score.
% 


\section{Evaluation}
\label{sec:evaluation}

For our evaluation, we implement PerfRanker based on Soot for static analysis and Java Agent for profiling the base version. 

%In this section, we first introduce the selection of subjects and code commits in subsection~\ref{subsec:subjects}, and introduce the evaluation setup and environment in Subsection~\ref{subsec:setup}. Then, we introduce the metrics for effectiveness measurement in Subsection~\ref{subsec:metrics}, and the evaluation results in Subsection~\ref{subsec:results}. (points-to analysis, call-graph building, control flow analysis, and data flow analysis for correlating collection variables and loops)  (dynamic call graph, iteration number of loops, and execution time of methods)
\subsection{Evaluation Subjects}
\label{subsec:subjects}

We apply PerfRanker on two popular open source projects: Xalan~\cite{xalan} and Apache Commons Math~\cite{commonMath}. Specifically, Xalan is an XSLT processor, and Apache Commons Math is a library for mathematical operations. We choose these two projects to cover both data formatting and mathematical computations, which are two representative time-consuming components in modern software. Xalan is equipped with a performance test suite of 64 test cases. Since Apache Commons Math is not equipped with a performance test suite, we leverage its unit test cases as performance test cases. 

\textbf{Version Selection.} For Apache Commons Math, we use its version on Jan 1, 2013 as its base version. For Xalan, since there are very few code commits after 2013, we use version 2.7.0 as its base version, as 2.7.0 is the first Xalan version compatible with Java 6 and higher. For both software projects, we collect all code commits from the base version until Mar 17, 2016, the time when we started collecting data for our work.  From all code commits, we remove those that do not change source files and those that do not involve semantic changes (e.g., renaming variables), as developers can easily determine that those commits will not affect software performance. Furthermore, we choose as our code-commit set the top 15 code commits whose changed code portions are covered by most test cases, where test prioritization is most needed. 

%For each selected code commit, we execute test cases on the commit for 5,000 times to determine performance regressions.

In Table~\ref{tab:subjects}, we present some statistics of the studied subjects and versions. The table shows that either project has more than 300K lines of code. Furthermore, there are hundreds of code commits and changed files between the base version and our selected code commits. In our evaluation, we do not update the base version, so the overhead of profiling the base version is low compared with the number of code commits under study.  More details about our evaluation subjects can be found on our project website~\cite{perfranker}.

\begin{table}
	\centering
	\caption{Evaluation Subjects}	
	
	
	\label{tab:subjects}
	\begin{tabular}{|l|r|r|} 
		\hline
		Subject  & Xalan  & Apache Commons Math  \\ 
		\hline
		Base Ver. & 2\_7\_0 & Jan 1st, 2013 \\ 
		\hline
		Size (LOC) of Base Ver. & 413,534   & 398,171  \\ 
		\hline		
		\# Commits Since Base Ver. & 354   & 1,321 \\ 
		\hline
		\# Changed Files  & 1,206  & 1,613 \\ 
		\hline
		Last Commit Date &Aug 11, 2015&Mar 17, 2016     \\ 
		\hline
%		Loop Exercise By Test & 364                         & 4151                             \\ \hline
	\end{tabular}
	\vspace{+1cm}
\end{table}





%We choose xalan version 2.7.0 as base version because version 2.6 is not compatible to jdk 1.6 to upper version. Benchmarking framework for XSLT, developed by Saxonica contains a set of test material, a set of test drivers for various XSLT processors, and tools for analyzing the test results.In the benchmark  we found 128 performance test case and only 64 of them run successfully with Xalan version 2.7.0. Similarly we chose 3900 test case for common math and studied the commits from github between  Dec 29, 2012 and Mar 17, 2016.

\subsection{Evaluation Setup}
\label{subsec:setup}


To determine performance regressions as the ground truth of performance changes for all test cases and code commits, we execute  the test cases for 5,000 times on the base version and each code commit under study. Furthermore, we execute the base version with our Java Agent to record the dynamic call graph and the execution time of each method, as well as the iteration number of each loop. To record average execution time of methods defined in the JDK library, we execute the Dacapo benchmark 9.12~\cite{dacapo} with profiling (we remove Xalan from the benchmark to avoid bias). All the executions are conducted on a Dell x630 PowerEdge Server with 32 cores and 256GB memory, and the server is used exclusively for our evaluation to avoid noises. 


%In Table~\ref{tab:profile}, we show top 5 frequently called JDK API methods and their average execution time in nano seconds. As described in Section~\ref{subsec:model}, this profile data is used in estimating the execution time of newly reached method bodies defined in JDK. 

%\begin{table}[] 
%	\small
%	\centering
%	\caption{Profiling Results}
%	\label{tab:profile}
%	\begin{tabular}{|l|r|r|} 
%		\hline
%		JDK api               & Average Time (ns)                      & Frequency               \\ \hline
%		java.util.Iterator:hasNext & 18	& 84,651,629  \\ \hline
%		java.util.Iterator:next	& 110	& 75,253,712  \\ \hline
%		java.util.List:get	& 22	& 38,302,721  \\ \hline
%		java.util.Map:get	& 100	& 21,632,743  \\ \hline
%		java.util.List:add	& 37	& 7,048,361  \\ \hline
%		... &            &   \\ \hline
%	\end{tabular}
%\end{table}


%We run the selected test sets on base version of each project and record context sensitive execution summaries of methods, loop counter and run time call graph. We can calculated average execution time of method and loop counter from all the profile data which is consider as global summary in our algorithm and local summary as per test case specific profile data. 
%We compared our approach to change aware random ranking which discussed in \ref{subsec:evaluation-result}. Prioritization technique metrics provide to testers the possibility to order their test cases so that test cases with large priority  are executed first and after test cases with less priority are executed in the regression testing process. APFD is commonly used to evaluate test case prioritization techniques to find functional fault. We adapt the metrics equation according to our needs in the formula \ref{eq:eq-apf}. In the formula p define as position in the ordering and IAPFD define as ideal APFD ranking.  The higher value of nAPFD signifies that highly impacted performance test case are in the top of ordering. 





\subsection{Evaluation Metrics}
\label{subsec:metrics}



To the best of our knowledge, our work is the first on prioritizing performance test cases for code commits, and we propose a set of metrics to evaluate the quality of different rankings. In our evaluation, we consider three ranking metrics: Average Percent of Fault-Detection on Performance (\textit{APFD-P}), normalized Discounted Cumulative Gain (\textit{nDCG}), and \textit{Top-N Percentile}. 

\textbf{APFD-P.} \textit{APFD}~\cite{AlexeyAPFD} is a commonly used metric for assessing a test sequence produced by test-case prioritization. If the test-suite size  is $N$, the total number of faults detected by the test suite is $T$, and the number of faults detected by the first $x$ test cases in the test sequence is $detected(x)$, then the \textit{APFD} of the test sequence can be defined in Formula~\ref{formula:apfd}:



\begin{equation}
APFD = \frac{\sum \limits_{x=1}^{N}\frac{detected(x)}{T}}{N} * 100\%
\label{formula:apfd}
\end{equation}


Unlike functional bugs where a test case either passes or fails, performance regressions are not binary but continuous. Performance downgrades of 20\% and 50\% are both regressions, with different severity. Therefore, instead of counting detected faults to attain the value of $detected(x)$, we replace the value of $detected(x)$ with the accumulated performance change. We define the \textit{performance change} of the $i^{th}$ test case in the test sequence (denoted as $change(i)$) in Formula~\ref{formula:impact} as below, in which $exe(i)$ is the execution time of the $i^{th}$ test case in the current version, and $exe(i_{base})$ is the execution time of the $i^{th}$ test case in the base version. 



 %Specifically, we define the performance change of a code commit on a test case as the relative change on execution time. 



\begin{equation}
change(i) = \frac{|exe(i) - exe(i_{base})|}{exe(i_{base})}
\label{formula:impact}
\end{equation}

Then, we define \textit{APFD-P} the same as  Formula~\ref{formula:apfd}, except that $detected(x)$ is defined as the accumulated performance change, as shown in Formula~\ref{formula:detect}, and $T$ is the sum of performance changes on all test cases. Actually, with such a definition, \textit{APFD-P} can be viewed as \textit{APFD} where all test cases reveal faults, and these faults are weighted by performance changes.

\begin{equation}
detected(x) = \sum_{i=1}^{x}change(i)
\label{formula:detect}
\end{equation}

As an illustrative example, consider 3 tests $t_1$, $t_2$, and $t_3$ with 10\%, 20\%, and 30\% performance downgrades, respectively. The best ranking is $t_3$, $t_2$, $t_1$; the total performance impact is $10\%+20\%+30\% = 60\%$; and the covered performance impact after each test is $30\%/60\% = 50\%$, $(30\%+20\%)/60\% = 83\%$, and $(30\%+20\%+10\%)/60\% = 100\%$. The P-APFD is thus (50\% + 83\% + 100\%)/3 = 78\%. 


%{\fontsize{7}{7}
%	\begin{multline}
%	\begin{gathered}
%	APFD_p =  \sum \limits_{i=1}^{p} impactRatio(p); \\
%	APFD = \sum \limits_{p=1}^{n} APFD_p \\
%	nAPFD = \frac{APFD}{IAPFD}
%	\label{eq:eq-apf}
%	\end{gathered}
%	\end{multline}
%}

\textbf{nDCG.} nDCG~\cite{NDCG} is a metric of ranking widely used in information retrieval. The basic idea is to calculate the relative score a given ranking with an ideal ranking, and the score of an arbitrary ranking is defined below, where $change(i)$ is defined in Formula~\ref{formula:impact}.

%and penalize each element that has high relevance score but is ranked lower in the given ranking with a value being logarithmic to the position difference. 

\begin{equation}
DCG (seq) = change(1) + \sum \limits_{i=2}^{N} \frac{ change(i)}{log_2(i)}
\label{formula:dcg}
\end{equation}


%In Formula~\ref{formula:dcg}, for a given sequence $seq$, we use the performance impact on the $i^{th}$ test case ($impact(i)$) as its relevance score, and $N$ to denote the length of $seq$, then the \textit{DCG} value of a given sequence $seq$ is defined in Formula~\ref{formula:dcg}. The \textit{nDCG} value of a given sequence $seq$ is its \textit{DCG} value normalized by the \textit{DCG} value of an ideal sequence, and is thus defined in Formula~\ref{formula:ndcg}, in which $ideal$ is the ideal ranking of the test cases (i.e., in descending order of performance changes).


%\begin{equation}
%nDCG (seq) = \frac{DCG(seq)}{DCG(ideal)}
%\label{formula:ndcg}
%\end{equation}


%The higher value of nDCG signifies that highly impacted performance test case are in the top of ordering.


\textbf{Top-N Percentile.} The \textit{APFD-P} and \textit{nDCG} defined earlier are adapted versions of widely used metrics, and can be used to compare different prioritization approaches. However, they are not sufficiently intuitive to help understand how much developers can benefit from an approach. Therefore, in our evaluation, we also measure how high percentage of top-ranked test cases in a test sequence need to be executed to cover the test cases with top $N$ performance impacts (we use 1 and 3 for $N$). For example, if the test cases with top 1, 2, and 3 performance impacts are ranked in the $2^{nd}$, $9^{th}$, and $5^{th}$ positions in a test sequence with length 100, then the top 1, 2, and 3 percentiles are 2\%, 9\%, and 9\%, respectively. 

\subsection{Baseline Approaches Under Comparison}

Although we are not aware of approaches specifically designed for prioritizing performance test cases in regression testing, it is possible to adapt existing approaches for performance test prioritization. In our evaluation, we compare our approach with three baseline approaches: Change-Aware Random, Change-Aware Coverage, and Change-Aware Loop Coverage.

Specifically, in all baseline approaches, we apply change-impact analysis to rule out the test cases that do not cover any revised methods. Since our performance-impact analysis includes basic change-impact analysis, for fair comparison, we apply this change-impact analysis in all baseline approaches. Note that we gathered coverage information from the base version, and we use the same technique as in our approach when selecting the code commits affecting most test cases in the three baseline approaches. 

After selecting the relevant test cases, the \textit{Change-Aware Random (CAR)} approach simply ranks the test cases in random order\footnote{To acquire more stable results, we use the average result of 100 random ordered test sequences as the result for CAR.}. The \textit{Change-Aware Coverage (CAC)} approach applies coverage-based test prioritization~\cite{Rother99:testprio} on the covered methods with the additional strategy~\cite{additionalTestPrior}, being a state-of-the-art approach in defect-oriented test prioritization. The basic idea is to first select the test case with the highest coverage, and iteratively select the test case that covers the most not-covered code portions as the next test case. In our evaluation, we use method coverage as the criterion, being consistent with the granularity of our performance model. The \textit{Change-Aware Loop Coverage (CALC)} approach is the same as CAC, except for using coverage of loops instead of methods as the criterion. 

\subsection{Quantitative Evaluation}
\label{subsec:results}
In our quantitative evaluation, we compare our approach and the three baseline approaches on all three metrics. 

\subsubsection{\textit{APFD-P} Metric}

%Our approach validated by real-world code changes and evaluate the tool on 354 commits of Xalan and 1321 commits of Apache common math.
%After filtering the result is shown in the figures. The filtered commits by our tool either only change non-source files or have insignificant changes on source files. Interestingly, filtering already reduces a significant number of commits not worth consideration for performance regression testing. We evaluated our result in three different metrics: APFD, DCG and Top Percentage aspect. In the case of APFD and DCG we compared our approach with change aware random approach. In change aware random, we generated 100 random ordering of test sets and calculated average nAPFD and nDCG. %APFD is computed after the prioritization only to measure the performance of the prioritization technique. 


Figures~\ref{fig:common-math-apfd} and~\ref{fig:xalan-apfd} show the comparison results between our approach and three baseline approaches on the \textit{APFD-P} metric. In the two figures and all the following figures, the X axis lists all the code commits studied chronologically, and the Y axis shows the \textit{APFD-P} value (or \textit{nDCG} value). We use different colors to represent different approaches consistently for all figures according to the legend in Figure~\ref{fig:common-math-apfd}. 

\begin{figure}
		\centering
		\includegraphics[width=4in, height=2.5in]{performance/images/common-math-apfd.pdf}
		
		\caption{\textit{APFD-P} Comparison on Apache Commons Math}	
		\label{fig:common-math-apfd}


\end{figure}

\begin{figure}
		\centering
		\includegraphics[width=4in, height=2.1in]{performance/images/xalan-apfd.pdf}
			
		\caption{\textit{APFD-P} Comparison on Xalan}
	
		\label{fig:xalan-apfd}
\end{figure}

%\begin{figure}
%		\centering
%		\includegraphics[width=\columnwidth]{images/common-math-coverage-apfd.pdf}
%		\caption{APFD-P Comparison with Change-Aware Coverage on Apache Commons Math}	
%		\label{fig:common-math-coverage-apfd}
%\end{figure}
%
%\begin{figure}
%		\centering
%		\includegraphics[width=\columnwidth]{images/xalan-coverage-apfd.pdf}
%		\caption{APFD-P Comparison with Change-Aware Random on Xalan}	
%		\label{fig:xalan-coverage-apfd}
%\end{figure}

Figures~\ref{fig:common-math-apfd} and~\ref{fig:xalan-apfd} show that our approach is able to achieve over 80\% \textit{APFD-P} value in most code commits affecting performance, and outperforms or rivals all baseline approaches in all code commits affecting performance from both subject projects. Specifically, for Apache Commons Math, our approach achieves an average \textit{APFD-P} value of 83.7\%, compared with 64.3\% by CAR, 64.6\% by CAC, and 66.1\% by CALC. For Xalan, our approach achieves an average \textit{APFD-P} value of 83.5\%, compared with 65.8\% by CAR, 63.6\% by CAC, and 59.8\% by CALC. Therefore, the improvement on the average \textit{APFD-P} is at least 17 percentage points, compared with baseline approaches on both projects. Furthermore, we do not observe significant effectiveness downgrade in the later versions, indicating that one base version can be used for a relatively long time.

%although we concede that the result may be due to the 

%Therefore, our approach is able to improve the APFD-P 


%In the figure \ref{fig:common-math-apfd}, we can see that overall nAPFD of our approach in Apache common math on the average 21.5\% greater than change aware random approach. In all the version our approach out perform random approach.





\subsubsection{\textit{nDCG} Metrics}

Similarly, Figures~\ref{fig:common-math-dcg} and~\ref{fig:xalan-dcg} show the comparison results between our approach and the three baseline approaches on the \textit{nDCG} metric. 

\begin{figure}
\centering
		\includegraphics[width=4in, height=2.1in]{performance/images/common-math-dcg.pdf}
			
		\caption{\textit{nDCG} Comparison on Apache Commons Math}	
		\label{fig:common-math-dcg}
\end{figure}

\begin{figure}
\centering
		
		\includegraphics[width=4in, height=2.1in]{performance/images/xalan-dcg.pdf}
			
		\caption{\textit{nDCG} Comparison on Xalan}	
		\label{fig:xalan-dcg}
\end{figure}


%\begin{figure}
%		\includegraphics[width=\columnwidth]{images/common-math-coverage-dcg.pdf}
%		\caption{nDCG Comparison with Change-Aware Coverage on Apache Commons Math}	
%		\label{fig:common-math-coverage-dcg}
%\end{figure}
%
%\begin{figure}
%		\includegraphics[width=\columnwidth]{images/xalan-coverage-dcg.pdf}
%		\caption{nDCG Comparison with Change-Aware Coverage on Xalan}	
%		\label{fig:xalan-coverage-dcg}
%\end{figure}

The figures show that our approach outperforms or rivals baseline approaches on \textit{nDCG} in almost all code commits from both subject projects. Specifically, for Apache Commons Math, our approach achieves an average \textit{nDCG} value of 74.5\%, compared with 47.2\% by CAR, 46.5\% by CAC, and 48.4\% by CALC. For Xalan, our approach achieves an average \textit{nDCG} value of 71.7\%, compared with 43.0\% by CAR, 41.4\% by CAC, and 37.4\% by CALC. Therefore, the improvement on the average \textit{nDCG} is over 26 percentage points on both projects. We also observe that there is one code commit from Xalan, in which our approach performs slightly worse than CAR on \textit{nDCG}. In Section~\ref{sec:qualitative}, we further discuss the details of this code commit in Listing~\ref{list:example-n1}.

Finally, we observe that compared with \textit{APFD-P}, the \textit{nDCG} values are generally lower and vary more significantly from code commit to code commit. The reason is that an \textit{nDCG} value is more sensitive to the rank of test cases with the highest performance impacts. For example, consider a test sequence with 100 test cases and only 1 test case has performance impact, and the performance impact value is 100\%. When this test case is ranked top, both \textit{APFD-P} and \textit{nDCG} are 1.0. However, if this test case is ranked $25^{th}$ in the sequence, the \textit{APFD-P} value is still as high as 75\%, but the \textit{nDCG} value becomes $1/log_2(25)$, which is less than 25\%. This result is reasonable because in information retrieval (where \textit{nDCG} was first  proposed), ranking the most relevant result at the $25^{th}$ position is very bad, but in test prioritization (where \textit{APFD} was first proposed), ranking the test case at the $25^{th}$ position is not so bad, because only 25\% test cases need to be executed to execute the test case. Therefore, which of \textit{APFD-P} and \textit{nDCG} is a better metric may depend on whether developers are interested in only a few most severely affected test cases, or a larger number of test cases whose performance is affected. 

%find that overall nDCG is higher comparison to nAPFD due to reason of penalising top impacted test case which are lower in the order. In some version improvement is much higher comparative to  


%The proposition of high nDCG is that highly relevant result appearing at the top. In the figure \ref{fig:common-math-dcg}, we can see that overall nDCG of our approach in Apache common math on the average 29.29\% greater than change aware random approach. 



%some other version for several reason is discussed above. Similarly, in the figure \ref{fig:xalan-dcg}, we can see that overall nDCG of our approach in Xalan on the average 27.21\% greater than change aware random approach.  \\
 


%We further compare our approach with change coverage approach. In figure \ref{fig:common-math-coverage-apfd}  and \ref{fig:xalan-coverage-apfd} shows the comparison of our approach with change coverage approach on  metric APFD. Our approach in common math on the average APFD score 0.85 whereas change coverage  average APFD score 0.68 (17\% greater). Similary, Our approach in xalan on the average APFD score 0.83 whereas change coverage  average APFD score 0.63 (19\% greater).In figure \ref{fig:common-math-coverage-dcg}  and \ref{fig:xalan-coverage-dcg} shows the comparison of our approach with change coverage approach on  metric DCG. Our approach in common math on the average DCG score 0.77 whereas change coverage  average DCG score 0.53 (24\% greater). Similary, Our approach in xalan on the average DCG score 0.71 whereas change coverage  average DCG score 0.41 (30\% greater).\\

\subsubsection{Top-N Percentile}

While \textit{APFD-P} and \textit{nDCG} are normalized quantitative metrics for our problem, they are not sufficiently intuitive for understanding the direct benefit of our approach on developers. Therefore, we further measure how many test cases developers need to consider if they want to cover the top 3 most affected test cases. Tables~\ref{tab:math} and~\ref{tab:xalan} show the results. Columns 1 and 2 present the code commit number and the total number of test cases affected by the code commit, respectively. Columns 3-6 and 7-10 present the top proportion of ranked test cases required to cover top 1 and 3 most-performance-affected test cases\footnote{The results for top 2 show similar trends and are available on the project website~\cite{perfranker}}. 

%In each column group, the first column presents our result, and the remaining columns present the results of CAC, CALC, and CAR, respectively. 

%In each cell, the number in the bracket shows the percentage of test cases need to be considered. 



\begin{table}
\scriptsize
	\centering
	\caption{Top-N Percentile of Apache Commons Math}
	\label{tab:math}	
	
	\begin{tabular}{r|r||r|r|r|r||r|r|r|r}
	\hline
C &T & \multicolumn{4}{c||}{Top 1} & \multicolumn{4}{c}{Top 3} 
\\  \cline{3-10}
\#  & \#   & Our & CAC & CALC & CAR &  Our & CAC & CALC & CAR\\\hline
	1 & 55 & 2\% & 3\%& \textbf{1\%} & 49\% & \textbf{6\%} & 43 & 27\%& 74\%\\
	2 & 60     & 40\%& 61\%& \textbf{38\%} & 49\% & \textbf{51\%} & 61\% & 93\% & 74\%\\
	3 & 152     & \textbf{2\%} & 84\%& 79\%& 49\%  & \textbf{28\%} & 88\% & 88\% & 74\% \\
	4 & 97      & \textbf{3\%} & 18\%& 87\% &  49\%  & \textbf{5\%} & 18\% & 87\% & 74\% \\
	5 & 130    & \textbf{4\%} & 29\% & 43\% & 49\% & \textbf{4\%} & 96\% & 97\% & 74\%  \\
	6 & 15     & \textbf{13\%} & 40\%& 33\%&46\% & 66\% & \textbf{40\%} & 100\% &  73\% \\
	7 & 18      & \textbf{5\%} & \textbf{5\%} & 88\% & 47\% & \textbf{15\%} & 40\% & 88\% & 74\%\\
	8 & 10     & \textbf{10\%} & \textbf{10\%} & 40\% &45\%  & \textbf{60\%} & \textbf{60\%} & 70\%& 70\%\\
	9 & 12     & \textbf{8\%} & 66\%&75\% & 45\% & \textbf{50\%} &66\% & 75\%& 72\%\\
	10 & 12      & \textbf{8\%} & 50\% & 83\% & 45\% & 83\% & \textbf{50\%} & 100\% & 72\% \\
	11 & 36      & \textbf{5\%} & 94\%& 38\% & 48\% & 66\% & 94\%& \textbf{58\%} & 74\%\\
	12 & 13     & \textbf{7\%} & 76\% & 38\%& 45\% & 76\% & 84\%& 100\%& \textbf{72\%} \\
	
	13 & 711    & \textbf{7\%} & 81\%& 84\% & 49\% & \textbf{7\%} & 81\%& 84\%& 74\% \\
	14 & 39      & \textbf{2\%} & 84\%& 25\%& 48\% & \textbf{12\%} & 84\% & 79\% & 74\% \\
	15 & 34      & \textbf{11\%} & 52\% & 17\% & 48\% & \textbf{26\%} & 58\%& \textbf{26\%} & 74\%\\ \hline
	Avg      &   & \textbf{8\%} & 50\% & 51\% & 48\%  & \textbf{37\%} & 65\% & 78\% & 73\% \\  \hline
	\end{tabular}
\end{table} 

\begin{table}
\centering 
\caption{Top-N Percentile of Xalan}
	
\label{tab:xalan}
\scriptsize
\begin{tabular}{r|r||r|r|r|r||r|r|r|r}
\hline
C &T & \multicolumn{4}{c||}{Top 1} & \multicolumn{4}{c}{Top 3} 
\\  \cline{3-10}
\#  & \#   & Our & CAC  & CALC & CAR&  Our  & CAC  & CALC & CAR\\\hline
1 & 63     & 20\%& 46\% & \textbf{14\%} & 49\% & \textbf{36\%} & 60\% & 50\% & 74\% \\
2 & 63   & \textbf{17\%} & 85\% & 22\% & 49\%& \textbf{17\%} & 85\%& 80\%&74\% \\
3 & 63      & \textbf{7\%} & 85\% & 50\% & 49\%  & \textbf{38\%} & 85\% & 66\% & 74\% \\
4 & 63   &  20\%& 80\% & \textbf{7\%} & 49\% & \textbf{20\%} & 85\% & 73\% & 74\% \\
5 & 63   & \textbf{6\%} & 85\% &  100\% & 49\% & \textbf{53\%}  & 85\% & 100\% & 74\%\\
6 & 63      & \textbf{11\%} & 63\% & 31\% & 49\% & \textbf{26\%} & 76\% & 77\% & 74\% \\
7 & 63   & \textbf{11\%} & 46\% &  12\% & 49\% & \textbf{15\%} & 85\% & 93\% & 74\% \\
8 & 63    & \textbf{9\%}  & 35\% & 36\% & 49\% & \textbf{20\%} & 82\% & 95\% & 74\% \\
9 & 58      & \textbf{3\%} & 96\% &  5\% & 49\% & \textbf{25\%} & 96\% & 98\% & 74\% \\
10 & 63   & 30\% & 47\% & \textbf{17\%} & 49\% & \textbf{31\%} & 85\% & 84\% & 74\% \\
11 & 63   & \textbf{1\%} & 31\% & 12\% & 49\% &  \textbf{7\%}  & 62\% & 74\% & 74\% \\
12 & 63     & \textbf{23\%} & 85\% & 88\% & 49\% & \textbf{34\%} & 90\% & 88\% & 74\%\\
13 & 63     & \textbf{12\%} & 46\%& 82\% & 49\%  & \textbf{53\%}  & 65\%& 82\% &74\% \\
14 & 58    & 34\% & 34\% & \textbf{8\%} & 49\% & 43\% & \textbf{37\%}  & 72\% & 74\% \\ 
15 & 63     & 36\% & \textbf{4\%} & 12\% & 49\% & \textbf{36\%} & 79\%& 61\% & 74\% \\ \hline
Avg        & & \textbf{16\%} & 57\%  & 34\% & 49\%& \textbf{30\%} & 77\%  & 79\% & 74\% \\		\hline
\end{tabular}
\end{table}

%For example, change version-3 touches 13 test case and actual impact score of top test case 1st,2nd and 3rd are 0.75, 0.07 and 0.06 respectively. Our approach ranked them with impact score 7.57,2.36 and 2.35 where overall nAPFD score of approach is 0.98 compared to change aware random score 0.63 that is 35\% improvement.


%Performance regression testing cost reduction is highly desirable. For example, running few top impacted test case can save a lot of testing time. Keep that goal in mind, we evaluated our result and We consider only those changes that touches more than three test case in the table \ref{tab:math} and \ref{tab:xalan}. 

Tables~\ref{tab:math} and~\ref{tab:xalan} show that on average our approach is able to cover Top 1 and 3 most-performance-affected test cases within top 8\%, 21\%, and 37\% of ranked test cases in Apache Commons Math, and 16\%, 22\%, and 30\% of test cases in Xalan. The improvements over the baselines approaches are at least 33\%, 37\%, and 28\%. Furthermore, on top 1 coverage, our approach outperforms or rivals the best baseline approach on 13 code commits from Apache Commons Math, and 10 code commits from Xalan. On top 3 coverage, our approach achieves the highest percentage on 11 code commits from Apache Commons Math, and 14 code commits from Xalan. 

%Also, our approach generates better result for Top 1 for code commits affecting fewer test cases, but better result for Top 3 for code commits affecting more test cases. The major reason is that, for code commit affecting fewer test cases, maybe only 1 or 2 test cases are affected with high performance impact. So the 3rd most affected test case may be not very different from other test cases. Actually, one drawback of using Top N coverage as metrics is that, it does not taking into account the actual performance impact, which is considered in $APFD-P$ and $nDCG$.


%consistently better than the two baseline approaches on most code commits, and 


\subsubsection{Overhead and Performance}
In test prioritization, it is important to make sure that the time spent on prioritization is much smaller than the execution time of performance test cases. To confirm indeed that is the case, we record the overhead and execution time of our analysis. Our profiling of the base version has a 1.92 times overhead on Apache Commons Math, and 5.18 times overhead on Xalan. The static analysis per test case takes 29.90 seconds on Apache Commons Maths, and 34.35 seconds on Xalan. Finally, for Apache Commons Math, the average, minimal, and maximal time for analyzing a code commit are 45.35 seconds, 4.23 seconds, and 262.30 seconds, respectively, while for Xalan, the average, minimal, and maximal time for analyzing a code commit are 9 seconds, 3.36 seconds, and 21.80 seconds, respectively. In contrast, it takes averagely 52 (3) minutes to execute test suite of Apache Commons Math (Xalan) for 173 (47) times to achieve an expectation of equal to or less than 10\% execution-time variance.


%we can see that, top1 contains on the average within 8\% in common math and 16\% in xalan. That means more than 85\% test cases can be eliminated. Similarly top2 and top3 contain on the average within  21\% and 37\% respectively in common math and   22\% and 30\% respectively in common xalan. So on the average our approach eliminate 70\% test case to select top 3 impacted test case.


%These table show the result of Top1, Top2 and Top3 impacted test case position in the ranking order and contain within percentage result of apache common math and xalan. For example, first row in the table \ref{tab:math} represents that the changes in version-1 impacted test case 1st, 2nd and 3rd position in the ranked list is 11th,1st and 2nd respectively and top 3 impacted test case contain within 17\% in the ranked list. The changes in version-9 touches 55 test cases and their relative ordering perfectly match the actual ordering of top 3. The main reason is that relative impact difference between the test case distinguishable which means that 1st relative impact 3 to 5 times compared to 2nd and 3rd in this case. But the change on version-13 touches 711 test cases and the impact on test case are very similar and little difference in their relative order which is top1's relative impact score 1.05 to 1.13 times compared to 2nd and 3rd. Within 0.54 score interval contains 653 test case. So we can say that a good separation margin in performance impact among the order test case provide better ranking order. Though their position in the ranking above 50, top 3 contains within 7\%. 


%There is another commit version where our approach perform better than random but it is below 25\% compared to ideal. It is interesting that top 3 contain within 7\% in the ranked list which means it can eliminate 93\% test case. The reason of this issue is explained in later paragraph under Top percentage result evaluation. 



%\subsubsection{Summary of Findings}
%To sum up, our quantitative analysis has the following major findings. 

%\begin{itemize}
%	\item Our technique outperforms both 
%	\item 
%\end{itemize}


\subsection{Successful and Challenging Examples}
\label{sec:qualitative}
%In all the version our approach out perform random approach except one version which touches 60 test case and interestingly top 3 contain within 51\%. Though our approach is below random in this case, it can eliminate 60\% test case to select top impacted test case. The overall ranking does not perform well in this change because of the reason that conditional return or throws statement  added in the top of method body is shown in the listing\ref{fig:example-2}. As a result some of test case skip the whole execution of method which spend time inside loop in the method body in earlier version. 

In this section, with representative examples of code commits, we explain why our approach performs well on some code commits but not so well on some others. 

%Due to space limit, for each code commit, we present only the relevant changes instead of the whole commit. More details about our code commit examples can be found on our project website~\cite{perfranker}.

\textbf{Successful Example 1.} Listing~\ref{list:example-p1} shows the simplified code change of code commit hash d074054... of Xalan. On this code commit, our approach is able to improve the \textit{APFD-P} and \textit{nDCG} values by at least 13.53 and 31.22, respectively, compared with the three baseline approaches. In this example, several statements are added inside a loop. Our approach can accurately estimate the performance change because (1) Line 2 in the code is an existing loop and from the profile database for the base version we can find out exactly how many times it is executed, and (2) the added method invocations are combinations of existing method invocations whose execution time is already recorded for the base version. \\

%Combining the information in our performance model, we can precisely predict the performance impact of the change for each test case.


{\fontsize{10}{10}
\begin{lstlisting}[columns=flexible,language=Java,caption=Change Inside a Loop, label={list:example-p1}]
  int nAttrs = m_avts.size();
  for (int i = (nAttrs - 1); i >= 0; i--){
    ...
+   AVT avt = (AVT) m_avts.get(i);
+   avt.fixupVariables(vnames, cstate.getGlobalsSize());
    ...
  } 
\end{lstlisting}
}


\textbf{Successful Example 2.} Listing~\ref{list:example-p2} shows the simplified code change of code commit hash 64ec535... of Xalan. On this code commit, our approach is able to improve the \textit{APFD-P} and \textit{nDCG} values by at least 17.14 and 24.33, respectively, compared with the baseline approaches. In this example, the loop at Line 4 depends on the collection variable \CodeIn{m\_prefixMappings} at Line 1 whose size can be inferred from the recorded number of iterations of existing loops. In this case, our approach can accurately estimate the iteration count of this new loop and estimate the overall performance impact based on the estimated iteration count.\\

{\fontsize{10}{10}
	\begin{lstlisting}[columns=flexible,language=Java,caption=A Newly Added Loop Correlating to an Existing Collection, label={list:example-p2}]
	+ int nDecls = m_prefixMappings.size();
	+      
	+ for (int i = 0; i < nDecls; i += 2){
	+   prefix = (String) m_prefixMappings.elementAt(i);
	+   ...
	+ }     
	\end{lstlisting}
}

%Our approach can easily identity hot path and changes that lie in the hot path from our performance model which is constructed on the call graph. Beside these any intra procedure change that related to loop can estimate impact cost except few cases. But we believe that actual improvement is more than this score because running the regression with top impacted test case is enough for exposing performance problem and the probability of selecting top impacted test case 7\% in random. 

%Collection are generally propagated through function parameter and field variable. In Listing~\ref{list:example-p3}, at line 1 introduce new array list and at line 6 this list is updated under a existing loop. We can easily estimate the length of new array list. New method introduce that is not shown here that iterate of over the new list, our technique can predict the iteration of that loop. 

%{\fontsize{7}{7}
%\begin{lstlisting}[language=Java,caption=Collection Propagation, label={list:example-5}]
%+ Collections.sort(limits);

%+ for (int i = 0; i < limits.size(); i += 2) {
%  +list.add(new Arc(limits.get(i), limits.get(i + 1)));
%+}
%\end{lstlisting}
%}

%{\fontsize{7}{7}
%	\begin{lstlisting}[language=Java,caption=Correlation New and Existing Collection, label={list:example-p3}]
%	+ List processedDefs = new ArrayList();
%	
%	for (int i = 0; i < nAttrs; i++){
%	+
%	
%	+ processedDefs.add(attrDef);
%	+
%	
%	}
%	\end{lstlisting}
%}

%\textbf{Positive Example 3.} The simplified code change of code commit of Apache common math is presented in Listing~\ref{list:example-p3}. On this code commit, our approach is able to enhance $APFD-P$ and $nDCG$ by at least 34.12, and 48.35, compared with 3 baseline approaches but it is not presented in $APFD-P$ and $nDCG$ graph because the number of test cases affected by the change are less than 10. In Listing~\ref{list:example-p3}, at Lines 3-5, elements from the collection variable \CodeIn{limits} are added to another collection variable \CodeIn{list}. In this case, our collection propagation technique is able to link the two collection variables, and propagate the impact to the loops depending on variable \CodeIn{list}. It should be noted that, the iteration number of the loop is actually half of the size of \CodeIn{limits}. In our approach, for simplicity, we ignore different operations on the index variables and simply deem the loop-iteration numbers to be the same as the size of collections it depends on. Since test prioritization does not require exact prediction of performance impact, we found our approach to be still effective with such approximations (e.g., in this example). 
%
%{\fontsize{7}{7}
%\begin{lstlisting}[language=Java,caption=Collection Propagation, label={list:example-p3}]
%+final List<Arc> list = new ArrayList<Arc>();
%+ Collections.sort(limits);
% + for (int i = 0; i < limits.size(); i += 2) {
%  +list.add(new Arc(limits.get(i), limits.get(i + 1)));
%+}
%\end{lstlisting}
%}
%
%Although our approach is almost consistently better than the two baseline approaches and the improvement is significant, we still want to investigate the cases where our approach does not perform well (e.g., the one case where our approach performs worse than Change-aware Random, and the cases where the improvement is not significant). 

\textbf{Challenging Example 1.} Listing~\ref{list:example-n1} shows the simplified code change of code commit hash 90e428d... of Apache Commons Math. On this code commit, with respect to the \textit{nDCG} metric, our approach performs better than the CAC baseline approach but slightly worse than the CAR baseline approach. In the example, at Line 2, an invocation to method \CodeIn{checkParameters()} is added, and the method may throw an exception. In this example, the execution time of \CodeIn{checkParameters()} can be easily estimated with our performance model. However, if the exception is thrown, the rest of the method will not be executed. Although we are able to estimate the execution time of the method's remaining part, it is impossible to estimate the probability of throwing the exception, as \CodeIn{checkParameters()} is a newly added method without any profile information. In such cases, if the probability of throwing the exception is higher in some test cases, the reduction of execution time due to the exception will be the dominating factor and result in inaccuracy in our prioritization.\\ 


{\fontsize{10}{10}
\begin{lstlisting}[columns=flexible,language=Java,caption=Return or Throw Exception at Beginning, label={list:example-n1}]
  protected PointVectorValuePair doOptimize(){
+   checkParameters();
    ...	
  }
+ private void checkParameters() {
+   ...
+   throw new MathUnsupportedOperationException;
+ }
\end{lstlisting}
}


\textbf{Challenging Example 2.} There are also cases where developers added a loop that is not relevant to any existing collection variables. As one of such examples, Listing~\ref{list:example-n2} shows the simplified code change of code commit hash a51119c... of Apache Commons Math. On this code commit, the improvement of our approach over the best result of the three baseline approaches is only 0.8 for \textit{APFD-P} and 6.0 for \textit{nDCG}. 
%
In the example, Line 2 introduces a new loop that does not correlate to any existing collection or array. In such a case, our approach cannot determine the iteration count of this loop and the depth of recursion at Line 8. To still provide prioritization results, our approach uses the average iteration counts of all known loops to estimate the iteration count of this new loop and always estimates the recursion depth to be 1. However, our prioritization result becomes less precise due to such coarse approximation.\\

{\fontsize{10}{10}
\begin{lstlisting}[columns=flexible,language=Java,caption= New Loop with No Correlation, label={list:example-n2}]
  public long nextLong(final long lower, final long upper){
+   while (true) {
      ...
+     if (r >= lower && r <= upper) {
+       return r;
+     }              
+   }
+   return lower + nextLong(getRan(), max);  
 }
\end{lstlisting}

}


%In some version improvement is much higher comparative to  some other version for several reason. For example,in the listing\ref{list:example-2} conditional return or throws statement  added in the top of method body introduce complex logic which is hard to determine which test case execute the full body of the method. As a result introduce imprecise impact cost. Another reason is that a new method is added with a loop in the listing \ref{list:example-1} and 


\subsection{Threats to Validity}

Major threats to internal validity are potential faults in the implementation of our approach and baseline  approaches, potential errors in computing evaluation results of various metrics, and the various factors affecting the recorded execution time for the test cases. To reduce such threats, we carefully implement and inspect all the programs, and execute the test cases for 5,000 times to reduce random noises in execution time. Major threats to external validity are that our evaluation results may be specific to the code commits and subjects studied. To reduce the threats, we evaluate our approach on both data processing/formatting software and mathematical computing software, and both a unit test suite and a performance test suite. 

%Also, we use an objective standard to select the code commits where test prioritization is most useful. To further reduce the threat, we plan to evaluate our approach on more subject projects and more code commits. 


\section{Discussion}
\label{sec:discuss}
%\subsection{Limitations of Our Approach}
%As the first step towards effective prioritization of performance test cases, our approach has a number of known limitations as follows. 

\textbf{Handling Recursions.} Recursions and loops are two major ways to execute a piece of code iteratively. Our approach constructs dependency relationships between loops and collection variables to estimate the iteration counts of loops. For recursion cycles existing in the base version, we simply update execution-time changes along the cycle only once, and multiply the execution-time change by averaging the invocation depths, attained via dividing the total execution frequency of all methods in the cycle by the product of invocation-frequency sum of these methods from outside the cycle and the accumulated recursive invocations inside the cycle (multiplying invocation frequencies along the cycle once). As an example, consider a cyclic call graph: $X \to A$, $Y \to B$, $A \to B$, and $B \to A$. When $t(B)$ is changed, the impact is propagated to $A$ as $t(A)=t(B)*f_{AB}$. The propagation then stops to break the cycle. After that, impact on X becomes $f_{XA}*depth*t(A)$ where $depth$ is the cycle's number of execution iterations, estimated as below:  
\vspace{-0.15cm}
\begin{equation}
\frac{total(A)+total(B)}{(f_{XA}+f_{YB})*f_{AB}*f_{BA}}
\end{equation}

\noindent where $total(F)$ is the total frequency of method $F$ in profile, and the divisor is the product of all calling frequencies from outside and inside of a cycle. For newly added recursions, our approach currently does not support estimation of the invocation depth, and simply assumes the invocation depth to be 1. In future work, we plan to develop analysis to estimate the termination condition of newly added recursions and relate invocation depths to collection variables.

% The reasons are 1) according to literature~\cite{JIN12}, most performance bugs are related to loops, and 2) the termination condition of recursions are more complicated than loops, and thus it is more difficult to relate invocation depths of recursions with collection variables. 

\textbf{Approximation in Performance Estimation.} Since our approach is not able to make any assumption on the given code commit, we have to make coarse approximations for the parameters of our performance model. For example, we ignore non-method-call instructions in revised methods, assume all newly added recursions to have invocation depth 1, and use the average recorded execution time of existing methods and JDK library methods to estimate their execution time in the new version. More advanced analysis can result in more accurate execution-time estimation, yet with higher overhead in the prioritization process, so future investigation is needed for the best trade-off. 

%Actually, the execution time of a method invocation may largely depend on its input and invocation context. So the estimation would be more precise if we can estimate execution time of existing methods or JDK methods as functions of method-input sizes instead of constants. While such idea is possible based on our recorded profiles, a major challenge is to estimate the size of input in a newly added/revised method. We believe that such estimation can be partly done by extending our collection-loop correlation analysis. However, even when our current approach uses relatively coarser approximations, our evaluation results demonstrate its effectiveness. 

\textbf{Supporting Multi-threaded Programs.} Multi-threaded programs are widely used for high-performance systems. In multi-threaded programs, methods and statement blocks can be executed concurrently, and thus our performance model can be inaccurate because the product of invocation frequency and average execution time of a method is no longer the total execution time. To address concurrent execution, we need to analyze the base version to find out the methods that can be executed concurrently. Then, we can give such methods a penalizing coefficient to reflect the extent of concurrency. We plan to explore this direction in  future work. 

%\subsection{Metrics for Performance Test Prioritization}
%In our evaluation, we considered 3 metrics to measure the effectiveness of performance test prioritization. As discussed in Section~\ref{}, $APFD-P$ emphasize more on the overall performance impact on test cases, while $nDCG$ is more sensitive to the ranking of most affected test cases. Top-3 Coverage considers only the top affected test cases, but is more intuitive than the other two. More investigation and user studies may be required to find out which metric is more suitable for a given scenario. Also, our approach and all of the 3 metrics weighted all test cases equal other than the code commit's performance impact on them. In practice, the importance of a test case may differ a lot depending their execution frequency in real-world systems. A 2-3\% overhead on a very frequent operation may not be acceptable, but a 10\% 
%overhead on a rarely performed operation may not matter much. So a metric closer to practice may also need to take into account execution frequency of test cases. 

%\subsection{Selection of Base Versions}
\textbf{Selection of Base Versions.}
The overhead of our approach largely depends on the required number of base versions. In our evaluation, we use one base version to estimate the subsequent code commits ranging over more than 3 years and 300 code commits. The evaluation results do not show significant effectiveness downgrade as time goes by. One potential reason is that, both software projects in our evaluation are in their stable phase and the code commits are less likely to interfere with each other. 

%
%If the software project is evolving very fast, it is possible that we need to update the base version more frequently, but in such a scenario, the benefit of test prioritization will also be larger due to the frequent code commits. It should be possible to predict the time for updating the base version by observing the downgrade of prioritization effectiveness, and such prediction technique warrants future investigation. 








	\vspace{-0.2cm}
\section{Related Work}
\label{sec:related}
	\vspace{-0.1cm}
	
\textbf{Performance Testing and Faults.} Previous work focuses on generating performance test infrastructures and test cases, such as automated performance benchmarking~\cite{KALIBERA}, model-based performance testing framework for workloads~\cite{BARNA11}, using genetic algorithms to expose performance regressions~\cite{LUO16}, learning-based performance testing~\cite{GRECHANIK12}, symbolic-execution-based load-test generation~\cite{ZHANG11}, probabilistic symbolic execution~\cite{Chen2016}, and profiling-based test generation to reach performance bottlenecks~\cite{luoinput2016}. Pradel et al.~\cite{PradelISSTA2014} propose  an approach to support generation of multi-threaded tests based on single-threaded tests. Kwon et al.~\cite{ATC2013} propose an approach to predict execution time of a given input for Android apps. Bound analyses~\cite{SPEED} try to statically estimate the upper bound of loop iterations regarding input sizes, but they cannot be directly applied as the size of collection variables under a  certain test can be difficult to determine. Most recently, Padhye and Sen~\cite{PadhyeICSE2017} propose an  approach to identify collection traversals in program code; such approach has the potential to be used for execution-time prediction. In contrast to such previous work, our approach focuses on prioritizing existing performance test cases. The most related work in this direction is done by Huang et al.~\cite{huang2014performance}, whose differences with our approach are elaborated in Section~\ref{sec:intro}. 

Another related area is research on performance faults, including studies on performance faults~\cite{JIN12, PerfBugStudy}, static performance-fault detection \cite{Nistor14, JOVIC11, KILLIAN10, YAN12}, debugging of known performance faults \cite{SHEN05,HAN12,LEUNG07,AGUILERA03}, automatic patches of performance faults \cite{Nistor15}, and analysis of performance-testing results~\cite{FOO11,FOO10}. 


%The default approach for regression testing is to retest all test cases after releasing a new version, which is an expensive proposition. To solve this problem, there are good collection of industry case studies and research effort on performance regression testing in software systems. 

\noindent\textbf{Test Prioritization and Impact Analysis.} Test prioritization is a well explored area in regression testing to reduce test cost~\cite{HARROLD93,BLACK04,ZHONG06} or to detect functional faults earlier~\cite{ELBAUM00,KIM02,LI07}. Mocking~\cite{MockStudy} is another approach to reduce test cost, but it does not work for performance testing as mocked methods do not have normal execution time. Another related area is test selection or reduction~\cite{ROTHERMEL97,CHEN94,Hao2009} which sets a threshold or other criteria to select/remove part of the test cases. Most of the proposed efforts are based on some coverage criterion for test cases, and/or impact analysis of code commits. The impact analysis falls into three categories: static change impact analysis~\cite{TURVERref,Arnoldref96,Wang2010ASE}, dynamic impact analysis
~\cite{LAW03,ORSO11,APIWATTANAPON05}, and version-history-based impact analysis~\cite{ZIMMERMANN04,SHERRIFF08,MengHima}. Our approach leverages a similar strategy to rank performance tests according to the change impact on them. However, we propose specific techniques to estimate performance impacts, such as collection-loop correlation and performance impact analysis. 

%Functional regression testing is a well explored area to reduce testing cost by test case selection based on test case property andcode modification(), test suite reduction by removing redundancy in test suite () and test cases prioritization orders test case execution in a way to hope (). Different from these work, our goal is to reduce performance regression testing overhead via test suite prioritization based on change impact analysis whether an operation is expensive or lies in hot path. 

%\noindent\textbf{Impact Analysis.} The evolution of software systems and ongoing changes demand for explicit means to assess the impact of a change on existing artifacts and concepts. Thus, software change impact analysis is in the focus of researchers in software engineering. The important difference is that Our proposed method focus on the performance test suite prioritization  via performance impact implication of change.

\section{Future Works}
\label{sec:future}
In the future, we plan to further explore the following research directions. 

First of all, our study focuses on Java software libraries, so our conclusion may not be generalized to other programming languages. Therefore, we plan to conduct similar studies on software libraries written in other languages, especially non-object-oriented languages to confirm or extend our conclusion. We also plan to inspect more backward-incompatibility-related bug reports. 

Second, as we mentioned in Section~\ref{sec:studylib}, regression testing with developers' test suite may find only a small portion behavioral incompatibilities, and thus results in a very course underestimation of the number of behavioral incompatibilities. In the future, we plan to leverage automatic test generation and more advanced automatic test oracles to better detect behavioral backward incompatibilities. 

Third, due to the difference in the popularity of API methods, the potential influence of backward incompatibilities varies. A backward incompatibility is more important if the relevant API method is used (directly or indirectly) more widely. We plan to further study the influence of behavioral incompatibilities and signature incompatibilities. 

Fourth, in our study, we find a number of challenges and research opportunities including behavioral incompatibilities related to reflections, call backs, GUI, and execution environments, better documentation of behavioral incompatibilities, etc. We plan to address some of these challenges in the future. 
	\vspace{-0.2cm}
\section{Conclusion}
\label{sec:conclusion}
	\vspace{-0.1cm}
	
In this paper, we present a novel approach to prioritizing performance test cases according to a code commit's performance impact on them. With our approach, developers can execute most-affected test cases earlier and for more times to confirm a performance regression. Our evaluation results show that our approach is able to achieve large improvement over three baseline approaches, and to cover top 3 most-performance-affected test cases within 37\% and 30\% test cases on Apache Commons Math and Xalan, respectively. 

%To further enhance our approach, we plan to work on the following directions in future. First, we plan to evaluate our approach on more subjects and code commits, especially software of different programming languages or with frequent and large commits. Secondly, we plan to develop techniques for more accurate estimation of execution time considering method inputs as factors. Thirdly, we plan to extend our performance model for multi-thread software. Fourth, we plan to perform user studies to evaluate the effectiveness of our metrics in real world settings.




\chapter{Beyond API Signatures: Behavioral Backward Incompatibilities of Java Software Libraries}

\section{Introduction}
\label{sec:intro}
During software evolution, frequent code changes, often including problematic changes, may degrade software performance. For example, a study~\cite{huang2014performance} found that upgrading from MySQL 4.1 to 5.0 caused the loading time of the same web page to increase from 1 second to 20 seconds in a production e-commerce website. Even small performance degradation may result in severe consequence. For example, Google could lose 20\% traffic due to an increase of 500ms latency~\cite{Google}. Amazon could have 1\% decrease in sales due to a 100ms delay in page rendering~\cite{Stevefamov}. \\

Developers can apply systematic, continuous performance regression testing to reveal such performance regressions in early stages~\cite{foxref,poliniref,Ejref,MITCHELL,KALIBERA}. But due to its high overhead, performance regression testing is expensive to conduct frequently. Actually, the typical execution cost of popular performance benchmarks varies from tens of minutes to tens of hours~\cite{huang2014performance}, so it is impractical to run all performance test cases for each code commit. Recently, PerfScope~\cite{huang2014performance} was proposed to predict whether a code commit may significantly affect software performance and thus require performance testing. Specifically, PerfScope extracts various features from the original version and the code commit, and trains a classification model for prediction. Although PerfScope helps reduce code commits for performance regression testing, its empirical evaluation shows that a non-trivial proportion of code commits still require performance testing; thus, there is still a strong need of reducing the cost of conducting performance regression testing on a code commit, even after applying PerfScope in practice.\\

%Such problematic changes are referred to as performance regressions in this paper. and GCC from 4.3 to 4.5, Mozilla developers experienced an up to 19\% performance regression, which forced them to consider a complete switchover.Performance regression testing is a major approach to detecting performance regressions, This is widely advocated in academia, open source community and industry .for web servers is 3 minute to 1 hr, for databases is 10 minutes to 3 hrs, for compilers is 1 hr to 20 hrs and for operating systems is 2 hrs to 24 hrs . With the high revision frequency and high testing cost, 


%
%it does not consider the performance impact of code commits on individual test cases, and thus results in two limitations. First,
%

To address such strong need, developers shall prioritize performance test cases on a code commit for three main reasons. First, there can be high cost to execute all performance test cases on a code commit for large systems in practice. Second, as reported in a previous industrial study~\cite{TSEPerform} and our study in Section~\ref{sec:motivation}, various random factors may affect the observed execution time, so it typically requires a large number of repetitive executions to confirm a performance regression. Therefore, with prioritized test cases, developers can better distribute testing resources  (i.e., do more executions on test cases likely to trigger performance regressions). 
Third, a code commit may accelerate some test cases while slowing down others. It is often important for the developers to understand the performance of their software under different scenarios, while a coarse-grained commit-level technique is not helpful on this requirement. \\



%research effort by addressed this problem with PerfScope, or regressional performance testing. Specifically, for each new code commit, PerfScope applied to the code change various program analysis (e.g., hot path analysis, bound analysis, and data dependency analysis) to  to 

%For example, our study shows that it averagely requires 150 executions to confirm a 10\% performance difference, which is typically considered significant, and dynamic techniques bringing in more than 10\% overhead are typically not considered for deployment-time usage~\cite{}. 


%Mozilla’s Talos performance regression detection system~\cite{firefoxTalos} runs performance tests every time a change is pushed to the Firefox source repository~\cite{firefoxperf}. In our empirical evaluation, we observe that one code commit may affect more than 100 test cases. 


 


%There is a useful work on the performance risk implication of code change. But their approach may not accurately assess the risk of performance regression issues because of generic nature of static modeling and lacking profiling information. Furthermore, prioritizing commits is not enough to address performance regression because  a code changes touches several test cases is very common during the evolution of software. In our study, we found that some commits touches more than 100 test cases. So the key difference is that we focus  on the prioritizing test cases in regressional performance testing via performance impact analysis of code changes that includes one time profiling and static modeling.

To develop an effective test-prioritization solution for performance regression testing, we focus on \textit{collection-intensive software}, an important type of modern software whose execution time is heavily spent on loading, manipulating, and writing collections of data. Collections are widely used in software for scalable data storage and processing, and thus collection-intensive software is very common. Examples include libraries for data structures, text formating and parsing, mathematics, image processing, etc. Also, collection-intensive software is often used as components in complex systems. Moreover, a recent study~\cite{JIN12} shows that a large portion of performance bugs are related to loops, which are often used to iterate through collections. Our statistics show that 89\% and 77\% of loops iterate through collections for our two subjects Xalan and Apache Commons Math, respectively.\\

%improper iteration on collections has been identified as one of the most
%

For collection-intensive software, a straightforward approach to prioritizing performance test cases on a code commit would be measuring collection iterations (e.g., loops) impacted by the code commit and executed by each test case. However, such an approach may not be precise enough to differentiate test cases in the presence of newly added iterations, manipulations, and processing of collections, as well as their effect on existing collection iterations. Consider the simplified code example from Xalan in Listing~\ref{list:example-p3}. The code commit involves a new loop, and its location may or may not be at the hot spot for all or most test cases. Therefore, its impact on different test cases may largely depend on the different iteration counts of the added loop, the side effect of changing variable \CodeIn{list}, and the operations in the loop. Since Loop$_B$ depends on a collection variable \CodeIn{limits}, which further depends on Loop$_A$, we can infer the test-case-specific iteration count of Loop$_B$ from that of Loop$_A$; such iteration count can be acquired by profiling the base version for all test cases. Furthermore, we can infer the effect of adding \CodeIn{list.add(...)} on \CodeIn{list} with the iteration count of Loop$_B$, and update the iteration count of loops dependent on \CodeIn{list}. Moreover, we can enhance the estimation precision by using test-case-specific execution time of operations (e.g., \CodeIn{new Arc(...)}). 


{\fontsize{10}{10}
	\begin{lstlisting}[columns=flexible,language=Java,caption=Collection Loop Correlation, label={list:example-p3}]
	  while(i <= m\_size){ //Loop A
	    limits.add(new Limit(...))
	    ...
	  }
	  ...
	+ Collections.sort(limits);
	+ for (int i = 0; i < limits.size()-1; i++) { //Loop B
	+   list.add(new Arc(limits.get(i), ...));
	+ }
	\end{lstlisting}
}


These observations inspire us for three main insights to effectively model a code commit and its effect on existing collections and their iterations. First, collection sizes (e.g., \CodeIn{limits.size()}) and loop-iteration counts (e.g., Loop$_B$) can often be correlated, so collection sizes can be inferred from loop-iteration numbers and vice versa. Also, collection variables (e.g., \CodeIn{limits}) can be used as bridges to infer iteration counts of new loops (e.g., Loop$_B$) from existing loops (e.g., Loop$_A$). Second, collection manipulations (e.g., \CodeIn{list.add(...)}) are often inside loops, so the size of collections referred by collection variables (e.g. \CodeIn{list}) can be estimated from loop-iteration counts (e.g., Loop$_B$). Third, due to the large number of elements in collections, the average processing time of elements (e.g., \CodeIn{new Arc(...)}) is relatively stable, so a method's average execution time in the new version may be estimated from that in the base version.\\ 

Based on the three insights, we propose PerfRanker, which consists of four automatic steps. First, on a base version, we execute each test case in a profiling mode to collect information about the test execution, including the runtime call graph and the iteration counts of all executed loops. We also perform static analysis to capture the dependency among collection objects and loops. Second, based on the profiling information, we construct a performance model for each test case. Third, given a code commit, we estimate the execution time of each test case on the new version (formed by the code commit) by extending and revising its old performance model. We use profiling information and loop-collection correlations to infer parameters of the new performance model, and refer to this step as \textit{Performance Impact Analysis}. Fourth, we rank all the test cases based on the performance impact on them.\\

%its impact on each test case from two aspects: the execution time of the added and removed code in the revised method, and the execution time of the loops affected by the collection objects which are return values of revised methods. 




%To address the two limitations above, we propose a novel approach to prioritize individual test cases in performance regression testing via performance impact analysis, which estimates the impact of a given code revision on the execution time of a given test case. 



%our algorithm automatically takes the information in each code commit by statically analyzing the added code and removed code from automatically generated diff of each commit. Incorporating the profile and static information into each test case's call graph we determine the run-time cost and the execution frequency of the code affected by code changes. 

We implement our approach and apply it on two sets of code commits collected from popular open source collection-intensive projects: Apache Commons Math and Xalan. To measure the effectiveness of test case prioritization for a code commit in performance regression testing, we use three metrics: (1) \textit{APFD-P} (Average Percentage Fault Detected for Performance), an adapted version of the APFD metric~\cite{AlexeyAPFD} for performance testing, (2) \textit{DCG}~\cite{NDCG}, a general metric for comparing the similarity of two sequences, and (3) \textit{Top-N Percentile}, which calculates the percentage of test cases needed to be executed to cover the top N test cases whose execution time is most affected by the code commit. Our evaluation results show that, compared with the best of the three other baseline approaches, our approach achieves an average improvement of 17.6 percentage points on \textit{APFD-P} and 27.4 percentage points on DCG. Furthermore, for Apache Commons Math and Xalan, our approach is able to rank top 1 affected test case within top 8\% and top 16\% test cases, and top 3 affected test cases within top 37\% and 30\% test cases, respectively. 

%Since we are not aware of previous research efforts on prioritizing test cases for performance regression testing, we developed  for Apache Commons Math, and an improvement of 16.6 percentage points on APFD and 27.2 percentage points on DCG for Xalan, respectively

%calculated impact score is used with adaptive APFD and DCG metric formula to rank the test case which shows that they can significantly reduce the cost of performance regression. 


This paper makes the following major contributions:
  \vspace{-0.1cm}
\begin{itemize}
  \item A novel approach to prioritizing test cases in performance regression testing of collection-intensive software.
  
  
%supported by a novel technique called performance impact analysis which estimates the performance impact of code revisions on test cases.
  
%  \item A motivation study on the number of executions required to acquire stable performance testing results.
    
    
  \item Adaptation of the APFD metric to measure the result of test prioritization for performance regression testing. 
    
  \item An evaluation of our approach on real-world code commits from two popular open source collection-intensive projects. 
  
\end{itemize}
  
  
%The rest parts of this paper are organized as follows. In Section~\ref{sec:motivation}, we present a motivation study showing that multiple test executions are required to confirm a performance regression. In Section~\ref{sec:approach}, we introduce our approach and performance impact analysis in details. We present our evaluation results in Section~\ref{sec:evaluation}, and discuss some related issues in Section~\ref{sec:discuss}. Before we conclude in Section~\ref{sec:conclusion}, we introduce related research efforts in Section~\ref{sec:related}, and indicate some future work in Section~\ref{sec:future}.

\section{Research Scope}

%As the basis of our study, we clarify our definition of \textit{behavioral backward incompatibilities} in this section. In our paper, \textit{Behavioral backward incompatibilities} refers behavioral changes of a public or protected method / field (i.e., changing the default value of a field) between two consecutive versions of a software library, while the signature of the method / field remains unchanged in the two versions. The behavioral changes include any public accessible value difference of the argument / return value, or different effects on the user interface or the underlying system, after invoking the method or accessing the field. Unlike signature incompatibilities, behavioral incompatibilities are difficult to detect, and the reasons and characteristics of behavior incompatibilities have not been well studied. This is the reason why we try to acquire more understanding of behavioral incompatibilities through studies in this paper.  

%From the aspect of library-developer's intention, BBIs can be divided to intentional behavioral changes and regression faults. Actually, the barrier between the two is often vague, and both types of incompatibilities are causing similar failures at the client side, so we do not differentiate these two types of incompatibilities in our study. We discuss how this may affect our study results in Section~\ref{sec:discuss}. 

In our study, we try to answer the five research questions as follows. 


%\subsection{Research Questions}
%\label{subsec:libRQ}


\begin{itemize}

	\item \textbf{RQ1:} Are BBIs prevalent between consecutive version pairs of Java software libraries?

	\item \textbf{RQ2:} Are most BBIs distribute in the major version upgrades, so that minor version updates are safer?

	\item \textbf{RQ3:} What are the characteristics of BBIs and why library developers bring them in?

	\item \textbf{RQ4:} How are BBIs detected in regression testing compare with BBIs causing real-world bugs?

	\item \textbf{RQ5:} How are the BBI-related bugs fixed in software practice?
\end{itemize}

With the answers of these questions, we expect to understand: (1) whether BBIs are prevalent in the Java software libraries, and a problem that software developers need to face frequently, (2) whether BBIs are distributed most in major version upgrades, which are supposed to have some software interface changes, (3) whether it is possible to classify BBIs into several categories by their characteristics and reasons brought in, so that they can be avoided or detection and resolution techniques can be developed accordingly, (4) whether there are mismatches on the categories of BBIs being most detected and the categories of BBIs mostly likely to cause bugs, and (5) whether BBIs are fixed by library or client developers, and whether there exists certain fixing patterns on BBI related bugs. 

%To answer the four research questions, we designed a study that applies cross-version testing to 68 versions in 15 top Java software libraries, and another study that involves manual inspection of bugs related to behavioral backward incompatibilities in real world.

%\subsection{Types of Backward Incompatibilities}
%\label{subsec:incompType}
%As mentioned in Section~\ref{sec:intro}, in our study, we consider the following two categories of backward incompatibilities\footnote{In the description of the rest of this paper, we consider only public or protected classes, interfaces, methods, fields, unless indicated otherwise.}.

%\textbf{Signature Incompatibilities} are backward incompatibilities caused by the interface signature changes of a software library. They can typically be detected with a recompilation of the client code and the new version of software library, although there are some exceptions such as when the client software is using reflections to invoke library methods. In our study, we considered the following three types of signature incompatibilities in Java software libraries. 

%\begin{itemize}
%	\item Class Incompatibilities: Removal of an existing class / interface, or revision of modifiers (e.g., \CodeIn{static}, \CodeIn{final}) of a class / interface. 

%	\item Hierarchy Incompatibilities: Changing the ancestors of an existing class / interface. Note that such changes will result in compilation or runtime errors in type casts and \CodeIn{instanceof} expressions. 

%	\item Member Incompatibilities: Removal or changing the signature (e.g., parameter / return types, modifiers) of an existing method / field. We also consider as method incompatibilities the addition of an abstract method or a method to an interface, because such additions may result in new implementation requirements in client software. 
%\end{itemize}

%It should be noted that, for some special client software, adding any method to any library class may result in runtime errors (e.g., the client code uses reflection to iterate all methods, and put them into an array wit hard-coded size). However, we believe that such cases are very rare, so we do not deem general addition of classes / methods / fields as backward incompatibilities. 



\section{Cross-Version Testing Study}
\label{sec:studylib}
In this section, we introduce our experimental study\footnote{Our data for this study and the following bug study are both available at \url{https://sites.google.com/site/incompemp2017/}.} on consecutive versions of Java libraries with cross-version testing. 

\subsection{Study Setup}
\subsubsection{Subject Libraries}
In our experimental study, we used 15 popular Java libraries as our subjects as shown in Column 1 of Table~\ref{table:basicInfo}. Specifically, we include in our subjects the two most widely used Java libraries: OpenJDK and the Android framework. Actually, since these two libraries have corresponding runtime platforms (i.e., JVM and Android OS), their BBIs are more likely to cause runtime errors, because the old version of client software may be executed in JVM or Android without recompile, and thus may result in runtime errors. Our subjects also include 7 libraries from Apache, and 6 other third-party libraries. All these libraries are from different domains and are the most popular software libraries in their domains according to a statistics~\cite{QSIC14}~\cite{techstat} of class imports among top 5,000 Java software projects in Github.

%To make our study results more representative, we performed a preliminary study on the usage of software libraries in 10,000 most popular (by the number of stars) Github Java projects. In each Java project, we checked the import statements in Java source files, and for each package prefix (first 3 terms in a package name, or the whole package name if it has less than 3 terms), we count it appears in the import statements of how many different projects. Then, we ranked the package prefixes, and did some manual adjustment (e.g., separating apache commons libraries by looking for first 4 terms, and merge multiple android and Java package prefixes). Finally, we acquired a list of most popular Java software libraries. 

%From the list, we choose the top 2 libraries: Java Runtime and Android Platform, as well as 25 other libraries that are open source, have sufficient test code, history long enough, and standard test code organization (i.e., Maven style organization of test code) to facilitate the automation of our study. Note that least popular library in our subject set ranked 41st in the list. 

\subsubsection{Selection of Version Pairs}
Software developers use different levels of versions to mark different granularity of milestones in software evolution. In our study, we first rule out the alpha and beta versions which are typically immature versions and are not widely used by client software developers. Then, we also need to differentiate major versions and minor versions. According to Semantic Versioning~\cite{SemVer}\cite{SCAM2014}, backward-incompatible API changes can be allowed in major versions (e.g., Java 6), but not minor versions (e.g., Java6u32). Also, major versions are typically developed in separate branches, while minor versions just corresponds to certain commits in the trunk or a branch. 

To acquire a full picture, in our study, we study backward incompatibilities both between two consecutive major versions and within a major version. If a major version has more than two minor versions, we choose the first minor version and the last minor version to form an inner-major-version version pair. For example, ElasticSearch has four minor versions (1.0.0 through 1.0.3) for major version 1.0, and three minor versions (1.1.0 through 1.1.2) for major version 1.1. So, in our study, we choose four versions (1.0.0, 1.0.3, 1.1.0, 1.1.2), and form 1 major version pairs (1.0.3 to 1.1.0) and two minor version pairs (1.0.0 to 1.0.3, 1.1.0 to 1.1.2). We combine minor versions within a major version because they typically contain very small amount of changes (e.g., fixing a bug), and may be inverted in the later updates if bringing in bugs or BBIs. So the combination will remove temporary BBIs (brought in and fixed with in a major version) which may not affect client developers much. We also ruled out the versions that raise compilation errors, or unit test failures / errors\footnote{For JDK and Android, we used the versions despite test failures and errors because some test cases require hardware support that we do not have. For JDK and Android versions, we ignore the test cases that fail on their own version.}. Finally, we use only versions up to Jan. 2015, so that their status  (e.g., documentation content) is relatively stable. Details of our selected versions as presented in Column 2-4 of Table~\ref{table:basicInfo} (Column 5-6 present the release time of the first and last version of the subject used in our study).

%, and the full list of version pairs used in our study are at our project website\footnote{\url{http://xywang.100871.net/empIncomp.html}}.
\begin{table}
	\center 
	\caption{\label{table:basicInfo} Basic Information of Studied Subjects and Versions}
	\begin{tabular}{|l|r|r|r|r|r|}
		\hline 
	      	Subject & St. V.& End V. & \# V.  & St. Time & End Time \\
    		\hline 
			OpenJDK & 7b157 & 8b13 & 2 &  2011-7 & 2014-3 \\
			Android & 4.3.1 & 5.0.1 & 2 &  2013-10 & 2014-12\\
			log4j & 2.0.0 & 2.1 &2 & 2014-7& 2014-10\\
			maven & 3.0.0 & 3.2.5 & 4& 2010-10& 2014-12\\
			bukkit &  1.2.3& 1.7.2 & 6& 2011-12 & 2013-12\\
			beanutils & 1.9.0 & 1.9.2 & 1 &  2008-9 & 2013-12\\
			codec & 1.6 & 1.7 & 1 &  2011-11 & 2012-9\\
			fileupload & 1.2.0 & 1.3.1 & 3 & 2007-2 & 2014-2\\
			commons-io & 2.0 & 2.4  & 4 & 2007-7& 2012-4\\
			ela. Search & 1.0.3 & 1.3.9 & 7& 2014-4& 2015-2\\
			http-core & 4.0.1 & 4.3.3 & 6& 2009-2 & 2014-2\\
			jodatime & 2.0  & 2.7 & 7 & 2011-5& 2015-1\\
			jsoup & 1.1.1 &1.7.3  & 10& 2010-6& 2013-11\\
			neo4j & 1.8.3 & 2.0.3 & 5&  2012-11& 2015-2\\
			snakeyaml & 1.3 & 1.11 & 8 &  2009-7 &2012-9\\
			\hline
	\end{tabular}
	\vspace{1cm}
\end{table}
%			antlr   & 3.0 & 3.5.2 & 7& 2007-5 & 2014-3\\ %
%			collections & 3.2.1 & 4.0 & 1& 2008-4 & 2013-11\\
%			cm-lang & 3.0.0 & 3.3.2 & 4&  2011-7 &2014-4\\
%			groovy & 1.0  & 1.8.8 & 9 & 2007-1 &2012-9\\
%			gson & 1.0 & 2.3 & 13 & 2008-5& 2014-8\\
%			guice & 1.0  & 3.0 & 2& 2007-3& 2011-3\\
%			hadoop & 2.0.0 &2.5  & 5& 2012-5 &2014-8\\
%			hibernate & 3.2 & 3.6.10 & 4& 2006-10& 2012-2\\
%			jfreechart & 1.0.14 & 1.0.19 & 5 & 2011-11& 2014-7\\
%			quartz & 1.7.0 & 2.2 & 8&  2010-2& 2013-9\\
%			slf4j & 1.1.0 & 1.7.10 & 11&  2006-12& 2015-1\\
%						cli & 1.1 & 1.2 & 1 & 2007-7 & 2009-3\\


%also tried to build and to run unit testing (using the version's own test code) on the versions, and
%Typically, ,minor versions corresponds to , and consecutive pairs (e.g., Java6u32 and Java6u40) of minor versions are not maintained simultaneously. 


%According to Semantic Versioning~\cite{semver}, \textit{Major Versions} as we consider Regarding mature versions, we observe that they mainly fall into two levels. The first level of versions involve major changes in software features and usages, and we refer to them as . Typically, , and consecutive pairs (e.g., Java 6 and Java 7) of major versions are maintained simultaneously by software developers . By contrast,  the second level of versions are mainly for bug fixes and minor changes inside a major version, and we refer to them as \textit{Minor Versions}. 



%Since the code difference between different levels of version pairs varies, the selection of version pairs may have a large impact to our study results. 




\subsubsection{Detection of BBIs}
\label{subsec:incompDetect}
%To detect signature incompatibilities, we simply compare the class / interface / method / field signatures, as well as inheritance hierarchies between a pair of versions, and calculate the difference. Note that, for simplicity and to avoid repetition, when counting member incompatibilities, we do not add the members that are removed due to class or hierarchy incompatibilities (i.e., removing a class results in removal of all its members, and removing an ancestor of a class results in removal of all the members inherited from the ancestor). Also, for a changed or removed member in a class, we count it for one member incompatibility in the class where the member is actually changed or removed, no matter how many sub-classes of the class are affected. 

Not all test cases in the old version compile with the source code of the new version (typically because they suffer from signature incompatibilities). Therefore, to detect behavioral incompatibilities, for each version pair, we automatically recompiled the test code of previous version with the source code of the new version, and iteratively remove the test cases that do not compile with the new version of source code, until all test cases can be compiled successfully. Finally, we executed all the remaining test cases and collected test failures and errors. Specifically, test compilation errors appear in 37 versions from 12 subjects (except for BeanUtils, FileUpload, and Codec), and in total 2,590 of 57,208 test cases (4.5\%) are removed due to compilations errors. 

Since one BBI may causes multiple test failures and errors, we further manually inspected these test failures and errors and grouped them into BBIs. For each test error/failure, we extracted the error messages, the version diff of the failed test code, and the version diff of the test class and the source class being tested. Then, we categorize multiple test errors/failures as one BBI if the test errors/failures are caused by the same API method and result in the same error message, exception, or wrong value of the same output/side effect. Note that, among 296 backward incompatibilities, library developers have revised test cases (e.g., changing test oracles or ways of invocation) for 267 incompatibilities, and deleted test cases for the other 11. For the rest 18 incompatibilities, change was made in other places of the test suite such as revising the setting up code or upgrading referred libraries. We found that revised test cases can help a lot in understanding the behaviors of BBIs. This categorization was done by first 2 authors separately, with the third author as a judge for conflicts. %Subjectivity is unavoidable in such an exploratory process, so we provide all test failures/errors and detailed categorization information on the project website, which supports replicate studies.

 
%\textbf{Clustering of Backward Incompatibility Groups.}
%To cluster test failures and errors in to incompatibility groups, we leverage the observation th at, if two test methods do not refer to the same set of methods in the source code of the software library, their failure must not be caused by the same backward incompatibility. It should be noted that, it is still possible that the behaviors of two methods in the software library change due to a same root cause. However, they are typically considered as two backward incompatibilities of the software library because two different APIs are affected. Therefore, in our algorithm, we first clustered all test cases in one test class to a cluster. The reason is that, they may fail due to a same error in the setup method of the test class. After that, for each pair of test classes, if they invoke a same method in the source code of the library, we deem them as interfered. Then, we clustered the test cases based on the closure of the interference relationship. For OpenJDK, there are two packages that are very commonly used: \CodeIn{java.lang} and \CodeIn{java.util}, so we ruled out methods in these packages when calculating interference, and manually inspected test failures / errors in test cases of these two packages. Note that the way we clustered backward incompatibilities is also conservative, which is consistent with the detection process. Therefore, we can guarantee that all of the identified incompatibility groups are truly incompatibility groups, and the number we report in this paper is an under-estimation of the number of behavioral backward incompatibilities. 


%simply check whether two test cases (failed or with error) 

\subsection{BBIs in Popular Libraries}
\label{subsubsec:prevIncomp}

To answer \textbf{RQ1}, we present the detected test failures / errors from software-library consecutive version pairs in Table~\ref{table:incompatibilities}. The first column of the table presents the subject name. Columns 2-3 present the total number of test failures detected in all version pairs of a specific subject (abbreviated as $T.$), and the number of versions where test failures are detected (denoted as $I$) divided by all version pairs of the subject (denoted as $A$). Columns 4-5 and 6-7 present similar data for test errors and BBIs (after grouping). Note that a test failure is raised when an assertion in the test case fails, while a test error is raised when the test case throws an unhandled exception or fails to complete. We also carefully checked the corresponding release notes, API documents, and migration guides (for Android) of the corresponding version pairs, and present the results in Column 8 of Table~\ref{table:incompatibilities}.


\begin{table}
	\center 
	\caption{\label{table:incompatibilities} BBIs in Software-Library Version Pairs}
		
		
	\begin{tabular}{|p{2.2cm}|R{1cm}|R{1.2cm}|R{1.15cm}|R{1cm}|R{1.1cm}|R{1cm}|R{1cm}|R{1.1cm}|R{1cm}|}
		\hline 
		Subject & \multicolumn{2}{c|}{Failure} & \multicolumn{2}{c|}{Error} & \multicolumn{3}{c|}{B-Incomp.} \\ 		
		\cline{2-8}
		        & T. & I/A  & T. & I/A  & T. & I/A & Doc. \\
        \hline
		OpenJDK & 203  & 2/2 & 15   & 2/2 & 35   & 2/2 & 13\\
		Android & 112  & 2/2 &  11  & 2/2 & 56  & 2/2 & 20\\ % 44 3 7 // 68 8 34
		log4j  &  21  & 2/2  & 0  & 0/2 & 4 & 2/2 &1\\			
		maven  & 14  & 3/4 & 226  & 4/4 & 19 & 4/4 &3\\		
		bukkit    & 15  & 2/6& 31  & 3/6 & 7  &4/6 &0\\		
		beanutils &  0 &  0/1 & 0  & 0/1 & 0  & 0/1 &0\\
		codec    & 4  &  1/1  & 6  & 1/1 & 6 & 1/1 &3\\
		fileupload & 0  & 0/3 & 12  & 2/3 & 2   & 2/3 & 0\\
		commons-io   &  4  & 1/4 & 2  & 1/4 & 3  & 2/4 & 0\\
		ela. Search  & 36   & 4/7 & 98  & 3/7 & 24  & 4/7 & 5\\
		http-core & 60  & 5/6 & 15  & 4/6 & 32  & 5/6 & 10\\		
		jodatime & 15   & 5/7 & 6  & 2/7 & 17  & 5/7 & 7\\
		jsoup  & 54  & 9/10 & 2   & 1/10  &  36  & 9/10 & 3\\
		neo4j   & 3   & 2/5 & 7  & 1/5 & 6 & 2/5 & 0\\
		snakeyaml & 108  & 8/8 & 14  & 4/8& 49& 8/8 & 17 \\
		\hline
		\NameEntry{2}{\textbf{Tot.}} & \NameEntry{2}{\textbf{649}} & \textbf{46/} & \NameEntry{2}{\textbf{445}}& \textbf{28/} &\NameEntry{2}{\textbf{296}} & \textbf{52/} & \NameEntry{2}{\textbf{82}}\\
		              &      & \textbf{68}  &      &  \textbf{68} &     & \textbf{68}&\\
		\hline
	\end{tabular}
\end{table}


From Table~\ref{table:incompatibilities}, we make our observation as follows. Considering that cross-version testing may generate an under approximation of the number of BBIs, the prevalence of BBIs may be much higher than what is shown in the table.


\medskip\vspace{+0.05cm}
\noindent\begin{tabular}{|p{16cm}|}
	\hline
	\textbf{Observation 1:} BBIs between version pairs are prevalent among Java software libraries. We detect 296 BBIs in 14 of 15 subjects (93.3\%), and 52 of 68 version pairs (76.5\%). Averagely each version pair suffers from 4.4 BBIs, and only 82 of the 296 BBIs are documented\\
	\hline
\end{tabular}
\medskip\vspace{+0.2cm}

%Second, on average, we detected 9.5 test failures, 6.5 test errors, and 4.4 BBIs for each version pair, showing that one version pair typically suffers from multiple BBIs. 




%\textbf{Documentation of BBIs.} 
%\begin{table}
	\center 
	\caption{\label{table:test-doc} Average Test Compilation Failures and Documented Incompatibilities cross Version Pairs}
	\begin{tabular}{|p{0.5cm}|r|r||r|r|}
		\hline 
		Sbj. & \multicolumn{2}{c||}{Comp. Status} & \multicolumn{2}{c|}{Documentation}  \\ 		
		\cline{2-5}
		        & \#No Comp. Error & \#All & \#Doc. BBI & \#All BBI \\
        \hline
		JDK & 5,940 & 6,212 & 14 & 35  \\
		and & 13,788 & 14,010 & 24 &  56 \\ % 44 3 7 // 68 8 34
		log  & 2,107 & 2,215 & 1  & 4   \\			
		mav  & 2,536 & 2,598 & 3 & 19 \\		
		buk    & 1,124 & 1,139  & 0& 7 \\		
		bea &  1,263 & 1,263 & 0 & 0  \\
		cod    & 408 & 408 &  1  & 6  \\
		fil & 25 & 25 & 0 & 2 \\
		cio   &  744  & 767 & 1 & 3 \\
		ela  & 14,347  & 16,220 & 5 & 24 \\
		htt & 5,726 & 6,322 & 3 & 32 \\		
		jod & 15  & 2.2 & 5 & 17 \\
		jso  & 54 & 5.4 & 5 & 36  \\
		neo   & 3 &  0.6  & 0 & 6 \\
		sna & 108 &13.5 & 8 & 49  \\
		\hline
		\textbf{Tot.} & \textbf{649} & \textbf{9.5} & \textbf{70} & \textbf{296}\\
		\hline
	\end{tabular}
	\vspace{1cm}
\end{table}



\textbf{Distribution of BBIs between / within major versions.} Beyond the overall status of backward incompatibilities between consecutive version pairs of software libraries, we further studied the difference between major and minor version pairs. The results are presented in Table~\ref{table:incompVersions}.
From Table~\ref{table:incompVersions}, we have the observation as follows. 

\medskip\vspace{+0.05cm}
\noindent\begin{tabular}{|p{16cm}|}
	\hline
	\textbf{Observation 2:} Major version pairs and minor version pairs suffered from 4.7 and 3.5 backward incompatibilities on average, respectively, and 76\% of both types of version pairs 
	are backward incompatible. Since BBIs are still prevalent within a major version, continuous minor version updates are still not safe, although they are slightly safer than major version upgrades. \\
	\hline
\end{tabular}
\medskip\vspace{+0.05cm}


\begin{table}
	\center 
	\caption{\label{table:incompVersions} Distribution of BBIs in Different Version Pairs}
		
		
	\begin{tabular}{|l|r|r|r|r|r|r|}
		\hline 
        Subject& 
        \multicolumn{2}{c|}{Total} & \multicolumn{2}{c|}{Average} & \multicolumn{2}{c|}{Incomp. V / All V}\\
        \cline{2-7}
                & Mj. & Mn. & Mj.  & Mn. & Mj. & Mn.        \\
        \hline
		JDK &  23 & 12 & 23 & 12   &  1/1 & 1/1  \\
		Android & 56 & N/A & 28 & N/A  & 2/2 &  N/A \\ % 44 3 7 // 68 8 34
		log4j  & 3  & 1 & 3  &  1   &  1/1 & 1/1  \\
		maven  &  10  & 9  & 5 & 4.5  & 2/2  & 2/2 \\
		bukkit & 3 & 4 & 0.8 & 2  & 3/4  & 1/2 \\
		beanutils & N/A & 0 & N/A & 0  & N/A &  0/1 \\
		codec & 6 &  N/A & 6 &  N/A &  1/1 & N/A \\
		fileupload & 0 & 2 & 0 & 1  & 0/1  & 2/2  \\
		commons-io  &  3 & N/A  & 0.8 & N/A &  2/4 & N/A  \\
		ela.Search  & 0  & 24 & 0 & 6  & 0/3 & 4/4 \\
		http-core &15  & 17 & 5 & 5.7 & 2/3 & 3/3 \\		
		jodatime  & 17 & N/A & 2.4 & N/A  &5/7  & N/A \\
		jsoup  & 31 & 5 & 4.4 & 1.2 &  7/7 & 3/4 \\
		neo4j  &  6 & 0 &  2  & 0  & 2/3 & 0/2  \\
		snakeyaml & 49 & N/A &4.9 & N/A & 10/10  & N/A \\
		\hline
		\textbf{Total} & \textbf{222} & \textbf{74} & \textbf{4.7} & \textbf{3.5} & \textbf{36/47}  & \textbf{16/21} \\
		\hline
	\end{tabular}
	\vspace{0.5cm}	
\end{table}



%To rule out signature-level incompatibilities, we delete the test cases that do not compile in our cross-version testing. Since major version pairs have more signature-level incompatibilities, less test cases are being used in cross-version testing. Therefore, the similar numbers in the table may not imply similar severity of backward incompatibility. However, the table does show that there are many backward incompatibilities between minor version pairs, which is not good news, because minor version pairs are typically supposed to be used for patching and minor changes, and are thus expected to be backward compatible. 

\subsection{Categorization of BBIs}
\label{subsec:cats}

To answer \textbf{RQ3}, we categorize BBIs according to their incompatible behaviors, invocation conditions, and the reasons why library developers brought them in. 

For the categorization of incompatibilities from 3 aspects (behaviors, invocation constraints, and reasons), it is exploratory so we do not have a criterion available beforehand. We predefined high-level categories (e.g. return value change as a high-level category of incompatible behaviors). Then, first 2 authors went through the incompatibilities separately and classify them to the categories. They also annotate each BBI with labels made up by themselves. Then, all authors discussed and merged labels with similar meanings, and removed too-narrow labels to get the final set of finer-grained categories. If we found a finer-grained category cannot be put into predefined high-level categories (e.g., Environment in Figure~\ref{figure:condition}), we made them separate high-level categories. Due to the complexity of BBIs, the categorization process is not easy, especially when the BBI involves multiple API methods. 

Consider Example~\ref{example:Method}, which is a test code sample from Jsoup 1.7.1. In the test case, an HTML document object is generated from HTML text, and then the document is printed out using \CodeIn{doc.body().html()} after setting char set to ascii and turn on escape mode. In Jsoup 1.7.3, the translation of some special characters for escape under ascii char set becomes different, so the printed HTML text will be changed. After inspection, we found that the change is made in method \CodeIn{html()}. Therefore, until \CodeIn{html()} is called, the memory status remains the same for both versions. In such a scenario, we determine that \CodeIn{html()} is the API method for the BBI, and the four API methods called before it are invocation constraint of the BBI. 

\begin{example}
	\begin{verbatim}
Document doc = 
    Jsoup.parse("<p title=p> & < > ...");
doc.outputSettings().charset("ascii");
doc.outputSettings().escapeMode();
assertEquals("...", doc.body().html());
	\end{verbatim}
	\caption{Identification of BBI-Related API Method} 
	\label{example:Method} 
\end{example}

\subsubsection{Incompatible Behaviors}

From the 296 BBIs, we identified the following major categories of incompatible behaviors. 

\textbf{Exceptions and Crashes} indicates that, in the new version of the software library, an API method throws exceptions in a different way. This category contains 4 sub-categories. \textit{New Exception} indicates that the API method throws an exception in the new version but not in the old version. \textit{Different Exception} indicates that the API method throws exceptions both versions, but the exceptions are different. \textit{No Exception} indicates that, the API method throws an exception in the old version, but not in the new version. \textit{Infinite Loop} indicates infinite loop in the new version. 

\textbf{Return Variable Change} indicates that, the return value of an API method changes in the new version under certain usage scenario and input. Specifically, we divide this category into four sub-categories. \textit{Value Change} indicates that a primitive value (e.g., integer, boolean, String) is changed. The BBI in Example~\ref{example:Method} belongs to this category. \textit{Field Change} indicates a field of the return object is changed. \textit{Type Change} indicates that the actual type of the return value is changed, although the signature itself remain unchanged. This typically happens when the return type in the API method signature has many subtypes (e.g., \CodeIn{java.lang.Object}). \textit{Structure Change} indicates that, no primitive values in the return object is changed, but the object is organized differently (i.e., values of reference-type fields or sub-fields of the return object are changed). 

\textbf{Other Effects} indicates that an API method causes a different side effect on other parts of the software itself or the operating system, such as value changes of other variables in \textit{Memory}, changes of the \textit{GUI}, and \textit{File System}. 

The distribution of 296 BBIs is presented in Figure~\ref{figure:behavior}. From the figure, we can see that the categories of \textit{Return Variable Change} and \textit{Exception and Crash} account for 162 and 105 BBIs, respectively, and they combined to account for more than 90\% of the BBIs. The two major subcategories for \textit{Return Variable Change} are changes of primitive return value or field value changes of object-type return values, which account for more than 90\% of the category. Such a distribution implies either most BBIs cause exceptions / simple return value changes, or such BBIs are more likely to be detected by regression testing. We will try to answer this question to some extent with our field bug study, but in either case, the following observation is true. 

\medskip\vspace{+0.05cm}
\noindent\begin{tabular}{|p{16cm}|}
	\hline
	\textbf{Observation 3:} Most BBIs detected by regression testing will cause either exceptions or value changes of the return variable or its fields, while side effects and environment effects are seldom detected.\\
	\hline
\end{tabular}
\medskip

%Another observation is that, incompatibilities detected in JDK and Android distribute more averagely amount all categories. The reason of the two observations may be, unhandled exceptions and value changes are more easily to be caught by unit testing. Since the test suites of JDK and Android are more comprehensive, they are able to detect incompatibilities in more various categories. 

%\begin{table}
 
	\caption{\label{table:Behave} Incompatible Behaviors}
	\begin{tabular}{|l|r|r|r|r|r|r|r|r|r|r|}
		\hline
		&  \multicolumn{4}{c|}{Exceptions}          &    \multicolumn{3}{c|}{Ret. Val. Change}  & \multicolumn{3}{c|}{Other Effects}\\
\cline{2-4} \cline{6-11}
Sbj. & UE & NE & DE & IL & VC & TC & SC & ME & UI & FS\\
\hline
JDK & 15 & 2 & 0 & 0 & 11 & 4 & 1 & 0 & 2 & 0\\
and & 9 & 6 & 2 & 3 & 19 & 1 & 3 & 3 & 6 & 4\\
log & 0 & 0 & 0 & 0 & 2 & 0 & 0 & 0 & 0 & 2\\
mav & 13 & 0 & 0 & 0 & 5 & 0 & 1 & 0 & 0 & 0\\
buk & 5 & 0 & 0 & 0 & 2 & 0 & 0 & 0 & 0 & 0\\
bea & 0 & 0 & 0 & 0 & 0 & 0 & 0 & 0 & 0 & 0\\
cod & 4 & 0 & 0 & 0 & 2 & 0 & 0 & 0 & 0 & 0\\
fil & 2 & 0 & 0 & 0 & 0 & 0 & 0 & 0 & 0 & 0\\
cio & 0 & 0 & 2 & 0 & 1 & 0 & 0 & 0 & 0 & 0\\
ela & 9 & 2 & 0 & 0 & 11 & 1 & 0 & 0 & 0 & 0\\
htt & 5 & 2 & 1 & 0 & 12 & 0 & 0 & 12 & 0 & 0\\
jod & 5 & 0 & 0 & 0 & 12 & 0 & 0 & 0 & 0 & 0\\
jso & 2 & 0 & 0 & 0 & 34 & 0 & 0 & 0 & 0 & 0\\
neo & 4 & 0 & 0 & 0 & 2 & 0 & 0 & 0 & 0 & 0\\
sna & 8 & 4 & 0 & 0 & 34 & 2 & 1 & 0 & 0 & 0\\
\hline
\textbf{Tot.} & \textbf{81} & \textbf{16} & \textbf{5} & \textbf{3} & \textbf{147} & \textbf{8} & \textbf{6} & \textbf{15} & \textbf{8} & \textbf{6}\\
\hline
	\end{tabular}	
	\vspace{1cm}
\end{table}
\begin{figure}
	\centering
	\includegraphics[width=0.4\textwidth]{backward/figs/Behavior.pdf}
	\caption{BBI Distribution on Incompatible Behaviors}
	\label{figure:behavior}
	\vspace{0.3cm}
\end{figure}

\subsubsection{Invocation Constraints}

We further investigated the conditions under which BBIs can be invoked, and identified the following five major types of such conditions. 

\textbf{Always} indicates that the BBI always happens as long as the corresponding API method is invoked.

%However, it should be noted that such incompatibilities may not be easily detected in the client software, because the relevant API method may not easily invoked, and the incompatible behavior (e.g., changed return value) may be overwritten by other code and thus cannot be observed under certain conditions. %Such incompatibilities can be easily detected by regression unit testing, so they are more likely to be intended behavioral changes.

\textbf{Error} indicates that the BBI happens only when an error happens. An example is the change of error message when a network error is invoked. 

\textbf{Environment} indicates that the BBI happens only under certain environments of the application (e.g., operating systems, language settings). 

\textbf{Multiple APIs} indicates that a number of other API methods must be invoked before the backward incompatible API method to invoke the BBI. Example~\ref{example:Method} belongs to this category. 

\textbf{Input} indicates that the BBI happens when a certain input value is fed into the corresponding API method. Specifically, we divide this category into five sub-categories. \textit{Trivial Value} indicates a null pointer or an empty string / list as the input. \textit{String Format} indicates that strings with specific structure as the input. \textit{Special Field} indicates that objects with specific values at a certain field as the input. \textit{Special Value} indicates that certain primitive values (not including strings) as the input. \textbf{Special Type} indicates the argument of the API method must be of a specific subtype of its parameter type.

The distribution of 296 backward incompatibilities in the above categories is presented in Figure~\ref{figure:condition}. We can observe that, the top 3 categories of invocation conditions are \textit{Always}, \textit{Specific Value}, and \textit{String Format}. The commonality among the 3 categories of BBIs is that they can be invoked with one API method invocation. By contrast, the BBIs requiring multiple API methods to invoke are not common in those detected by regression testing. 


%253 of 296 (85.5\%) detected BBIs alway happen whenever the API method is invoked, or happen under certain inputs. In the \textit{input} category, 129 BBIs are invoked by simple input values, while the rest 32 are fields of complicated inputs. 



%The dominance of category \textit{Always (\textbf{AL})} may be because they are easier to detect.

%Also, we observed that the distribution of categories in some projects are uneven. For example, JSoup, as the most popular HTML parser with 36 incompatibilities in total, contributes 30 of the 43 incompatibilities in the category of  \textit{String Format (\textbf{SF})}. Actually, while incompatibilities are distributed evenly in general libraries like JDK and Android. For library of narrower usage such as JSoup, Elastic Search and SnakeYaml (a data presentation library), more than 70\% of incompatibilities concentrate in \CodeIn{Jsoup.parse()}, \CodeIn{Client.prepareSearch()} and \CodeIn{Yaml.dump()}, the most widely used APIs in the libraries. 

%\begin{table}
	\center 
	\caption{\label{table:Cond} Invocation Conditions of Incompatibilities}
	\begin{tabular}{|l|r|r|r|r|r|r|r|r|r|}
		\hline
		&        &                &             &            &              & \multicolumn{4}{c|}{Input}\\
		\cline{7-10}
Subject & AL & ER & EV & ST & MA & TR & SF & FI & SV\\
\hline
JDK & 2 & 2 & 2 & 0 & 6 & 2 & 2 & 6 & 13\\
and & 18 & 4 & 2 & 1 & 4 & 11 & 3 & 4 & 9\\
log & 2 & 0 & 0 & 1 & 0 & 0 & 0 & 0 & 1\\
mav & 13 & 0 & 0 & 0 & 0 & 0 & 6 & 0 & 0\\
buk & 5 & 0 & 0 & 0 & 0 & 1 & 0 & 0 & 1\\
bea & 0 & 0 & 0 & 0 & 0 & 0 & 0 & 0 & 0\\
cod & 2 & 0 & 0 & 0 & 0 & 0 & 0 & 0 & 4\\
fil & 2 & 0 & 0 & 0 & 0 & 0 & 0 & 0 & 0\\
cio & 1 & 2 & 0 & 0 & 0 & 0 & 0 & 0 & 0\\
ela & 6 & 2 & 0 & 2 & 4 & 2 & 1 & 7 & 0\\
htt & 8 & 4 & 0 & 0 & 3 & 3 & 0 & 7 & 7\\
jod & 5 & 1 & 0 & 0 & 0 & 8 & 0 & 0 & 3\\
jso & 0 & 1 & 0 & 0 & 2 & 1 & 30 & 0 & 2\\
neo & 3 & 0 & 0 & 0 & 0 & 1 & 0 & 2 & 0\\
sna & 25 & 2 & 0 & 6 & 2 & 2 & 1 & 6 & 5\\
\hline
\textbf{Total} & \textbf{92} & \textbf{18} & \textbf{4} & \textbf{10} & \textbf{21} & \textbf{31} & \textbf{43} & \textbf{32} & \textbf{45}\\
\hline
	\end{tabular}
		\vspace{1cm}
\end{table}
\begin{figure}
	\centering
	\includegraphics[width=0.4\textwidth]{backward/figs/Condition.pdf}
	
	\caption{BBI Distribution on Invoking Conditions}
	
	\label{figure:condition}
	\vspace{0.3cm}
\end{figure}


\subsubsection{Reasons for Bringing in BBIs}

Since the library developers revised the test cases to accommodate the changes, we can find out the reasons and the purposes of the behavior change from the revised test code. We identified the following 3 major categories of reasons, which contains 13 sub-categories. 

\textbf{Usage Change} indicates that the library developers expect to keep the normal behavior of the library, but also expect client developers to change their usage pattern. This category has 4 sub-categories, which are \textit{API Pattern Change} indicating changes on expected API sequences, \textit{Enforce Rules} indicating rejecting of some poor input, \textit{Enable Poor Inputs} indicating the support to some poor input, and \textit{Input Format} indicating the change on the format of string type inputs. 

\textbf{Better Output} indicates that the library developers expect to change the normal behavior of the API, while not expecting client developers to change their input. This category also has 5 sub-categories, which are \textit{More Reasonable Output/Effect} indicating behavior change of an API method under a normal input to make it more reasonable\footnote{For example, in log4j 2.0.2, the time stamp on the log is changed from when the log is initialized to when the time stamp line is printed. }, \textit{Change Default Setting} indicating changing of the default setting (often a constant such as the default screen size), \textit{Error Message} indicating change of reported errors, \textit{Explicit Report} indicating explicitly throwing exceptions for errors, and \textit{Output Format} indicating format change of string output. 

\textbf{Other} indicates reasons not in the above 2 categories, including \textit{Exposure of Internal Structure Change}, \textit{Signature Change exposed with Reflection}, and \textit{Upgrade Library}, in which the first two categories are regression faults.  

The distribution of 296 backward incompatibilities in the above categories is presented in Figure~\ref{figure:reason}. From the table, we can see the top 3 reasons for bringing incompatible behaviors are \textit{More Reasonable Output/Effect}, \textit{Output Format}, and \textit{Enforce Rules}. We also find that, \textit{Enable Poor Inputs}, the opposite of \textit{Enforce Rules}, is the 4th popular reasons. Such contradiction reflects hesitation of library developers on whether responsibilities (e.g., input validation checking) should be at the library side or client side. Also, we did observe library developers moving back-and-forth on some incompatibilities. For example, in SnakeYaml, whether the dump output should include class name of the data value is changed 3 times through versions 1.3 to 1.6. Furthermore, although \textit{More Reasonable Output/Effect} is the largest sub-category in \textit{Better Output}, 93 of 165 BBIs in the category are not semantic changes, but presentation changes (e.g., error message, explicit exceptions, output formats). The 17 BBIs on change of default setting also related to client developers' preference. To sum up, we have the following observation. 

\medskip\vspace{+0.05cm}
\noindent\begin{tabular}{|p{16cm}|}
	\hline
	\textbf{Observation 4:} Developers bring in a large portion of BBIs because they want to allow more or less inputs, or change the output presentation or option. This implies that the root cause of most BBIs is the different requirement of client developers, so the BBIs can be avoided if library developers understand requirement better (e.g., through surveys or code statistics on client projects), or have design for dual support.
	\\
	\hline
\end{tabular}
\medskip

%\begin{table}
\centering
	\caption{\label{table:reason} Reasons for Bringing in Incompatibilities}
	\begin{tabular}{|p{0.5cm}|R{0.5cm}|R{0.5cm}|R{0.5cm}|R{0.5cm}|R{0.5cm}|R{0.5cm}|R{0.5cm}|R{0.5cm}|R{0.5cm}|R{0.5cm}|R{0.5cm}|R{0.5cm}|R{0.5cm}|}
		\hline
		&  \multicolumn{5}{c|}{Usage Change}      &   \multicolumn{5}{c|}{Behavior Change}  & \multicolumn{3}{c|}{Other}\\
		\cline{2-14}
Sbj & AP & ER & EL & RE & IF & RB & CD & EM & DE & OF & IS & SR & UL\\
\hline
jdk & 6 & 13 & 3 & 1 & 0 & 7 & 2 & 0 & 1 & 1 & 1 & 0 & 0\\
and & 4 & 6 & 8 & 1 & 0 & 14 & 8 & 0 & 3 & 7 & 2 & 3 & 0\\
log & 2 & 1 & 0 & 0 & 0 & 1 & 0 & 0 & 0 & 0 & 0 & 0 & 0\\
mav & 0 & 0 & 1 & 1 & 0 & 2 & 0 & 1 & 0 & 0 & 1 & 1 & 12\\
buk & 0 & 1 & 0 & 0 & 0 & 0 & 0 & 0 & 0 & 2 & 0 & 4 & 0\\
bea & 0 & 0 & 0 & 0 & 0 & 0 & 0 & 0 & 0 & 0 & 0 & 0 & 0\\
cod & 0 & 4 & 0 & 0 & 0 & 2 & 0 & 0 & 0 & 0 & 0 & 0 & 0\\
fil & 0 & 0 & 0 & 0 & 1 & 0 & 0 & 0 & 0 & 0 & 0 & 0 & 1\\
cio & 0 & 0 & 0 & 0 & 0 & 1 & 0 & 0 & 2 & 0 & 0 & 0 & 0\\
ela & 1 & 2 & 3 & 5 & 0 & 3 & 1 & 4 & 0 & 2 & 1 & 2 & 0\\
htt & 2 & 4 & 5 & 3 & 1 & 10 & 2 & 0 & 1 & 4 & 0 & 0 & 0\\
jod & 0 & 3 & 1 & 6 & 1 & 0 & 1 & 0 & 1 & 3 & 1 & 0 & 0\\
jso & 0 & 2 & 8 & 0 & 3 & 13 & 0 & 0 & 0 & 10 & 0 & 0 & 0\\
neo & 0 & 3 & 1 & 0 & 0 & 1 & 1 & 0 & 0 & 0 & 0 & 0 & 0\\
sna & 2 & 2 & 3 & 2 & 0 & 1 & 2 & 16 & 1 & 15 & 3 & 2 & 0\\
\hline
\textbf{Tot.} & \textbf{17} & \textbf{41} & \textbf{33} & \textbf{19} & \textbf{6} & \textbf{55} & \textbf{17} & \textbf{21} & \textbf{9} & \textbf{44} & \textbf{9} & \textbf{12} & \textbf{13}\\
\hline
	\end{tabular}	
	\vspace{1cm}
\end{table}

\begin{figure}
	\centering
	\includegraphics[width=0.4\textwidth]{backward/figs/Reason.pdf}
	
	\caption{BBI Distribution on Reasons}
	
	\label{figure:reason}
	\vspace{0.3cm}
\end{figure}


\section{Real-world Bug Study}
\label{sec:studybug}
In this section, to answer the \textbf{RQ4} and \textbf{RQ5}, we present the study result on real world bug reports related to BBIs, and explore how BBIs are affecting the client software developers. The basic information of the collected bug reports are presented in Table~\ref{table:basicbug}. 

%In the following subsections, we first identify bug reports that are related to signature incompatibilities, and cross-software duplicate bug reports. Then, we categorize bug-inducing behavior incompatibilities and map them back to categories in our library study, and study their documentation status. Finally, we study how these bug are fixed or resolved by library developers and client developers. 

\subsection{Collection of Bug Reports}
To collect bug reports caused by BBIs. We searched two large on-line open bug repositories: JIRA and GitHub, on their project-specific bug report search engine. Specifically, we used as keywords the combination of terms related to software upgrading (e.g., ``upgrade'', ``update'', ``version''), and the names of the software libraries listed in Table~\ref{table:basicInfo}. From the collected bug reports, we \textit{randomly} selected 500 bug reports, and carefully inspected these bug reports. During the inspection, we read the developers' comments and other references from the bug report to check whether they are confirmed by the developers to be caused by BBIs, and retain all the bug reports that are caused by BBIs. 

We collected only bugs that are closed before Jan. 1st 2015 and never re-opened after that. We believe that these bugs are not likely to be re-opened so their status should be confirmed. We collect both bugs that are closed and fixed and bugs that are closed but not fixed due to developers' decision. 

The reason is that, BBIs bugs are related to both software libraries and client software, so they can be fixed either at the library side or at the client side. Also, there are cases that a BBI bug is never fixed because the software library developers refuse to revert their changes, and the client developers did not find a way to work around it. In such cases, the developers may choose to not to support the new version of software library (not likely for runtime libraries such as OpenJDK and Android because not supporting the updated platform will cause dysfunction in client software), or have the users to tolerate the bug if the bug is relatively minor. 

With the process above, we collected 126 bugs, and divided these bugs into two groups: \textit{library bugs} that are submitted to software library projects that have BBIs, and \textit{client bugs} that are submitted to software client projects because they triggers BBIs of their software libraries. The breakdown of collected bugs is shown in Table~\ref{table:basicbug}.

\begin{table}
	\center
	\caption{\label{table:basicbug} Basic Information of Bugs}
				
	\begin{tabular}{|l|r|r|r|}
		\hline
		Subject & Library Bugs& Client Bugs & Total\\ 
		\hline
		Java SDK &        8      &     10   & 18   \\
		Android &         13     &     64   & 77  \\
		Other   &         29     &    2    & 31  \\
		\hline
			Total   & 50             &  76     & 126     \\
		\hline
	\end{tabular}
	\vspace{+0.3cm}			
\end{table}

From the table, we can observe that, as we used a random selection of bugs, the majority of selected bug reports are from Android and Java SE. The reasons are two fold. First, Java SE and Android are much popular than other software libraries studied. Second, Java SE and Android are both runtime platforms, so that client software developers do not have control on which version of JVM and Android system their software will be executed on. Therefore, BBIs may be revealed after a Java or Android update at the users' side, and get reported to the client software developers. This also explains another observation that, the majority of bugs of Android and Java SE are client bugs, while the majority of bugs of other libraries are library bugs, because Android and Java BBIs are more likely to be reported by end users, while BBIs in other libraries are more likely to be reported by client software developers to library software as library bugs. Sampling from unbalanced bug sets is difficult. Random sampling will be ruled by dominating classes (e.g., JVM and Android). Giving quota to classes, samples will not reflect the actual data distribution, and proper quota size needs to be determined. We chose random sampling in our study to see the actual impact of BBIs in the field. 

%Obviously, Java SE is used in all Java projects. Furthermore, in our previous study on API popularity among 10,000 random Github projects~\cite{QSICStudy}, Android libraries are used in 34\% of the 10,000 projects , while Apache Http, which ranks the 3rd after Java SE and Android, is used in only 10\% of the projects. By contrast, most of the other libraries in our study are packaged with the client software. Thus, client software developers are able to test the backward compatibility of a new software-library version, and work around the backward incompatibilities before the software is released to the users. 

%\subsubsection{Selecting Test Errors / Failures for Manual Inspection}

%When selecting test errors and failures for manual inspection, we expect our subset to be selected random, as well as representative enough to cover all the studied subject libraries. Therefore, we use the strategy of senate-house election. Specifically, we first randomly selected 1 test error / failure from each subject library, and then we randomly select 100 test errors / failures from all the rest test errors and failures. So, finally we form a selected subset with 127 test failure and errors, from which we identified 1xx behavioral incompatibilities with manual inspection. 

\subsection{Study on Bug-Inducing BBIs}

In this subsection, we categorize the bug-inducing BBIs and compare them to the BBI distribution in our library study. Specifically, among the 126 bugs, we find 13 bugs (12 client bugs from Android and 1 library bug from http-core) that can be directly mapped to the BBIs found in our library study. This shows that the test cases in regression testing are far from sufficient for identifying all bug-inducing BBIs, but it also implies that, if the regression testing results can be well documented or conveyed to client developers in better ways, some bugs can be avoided. 


%\subsubsection{Signature Incompatibilities}

%Since the bug reports are randomly collected, some of the bug reports are related to signature-level incompatibilities. As we mentioned in Section~\ref{sec:intro}, since signature-level incompatibilities fail compilation, they are not likely to result in real world bugs. 

%Among the 144 bug reports, we do find 18 of then related to signature-level incompatibilities. Specifically, 6 of the 18 bugs are library bugs, which are reported by client developers to request of the recovery of a removed API method. Other 6 of the 18 bugs are runtime errors from client software running on Android, Java, and Bukkit. Because these libraries serve as runtime environments, when automatic updates happen, the software running on them will throw exceptions such as ``NoSuchFieldException'' or ``UnSupportedOperationException''. 

%The rest 6 of the 18 bug reports are more interesting. As for these bugs, the developers of relevant client software use reflections to invoke the API methods with signature changes, or use ``instanceof'' operations, or downcasts on the classes whose inheritance hierarchy has changed. Thus the signature incompatibilities become latent, and cannot be detected by compilers. It should be noted that, 4 of the 6 bug reports are related to Android. Reflection is encouraged in Android to address backward incompatibilities (i.e., call different API methods for different Android versions), so it seems that reflection serves as a double-edge sword here. 

%\framebox{\parbox{\dimexpr\linewidth-2\fboxsep-2\fboxrule}{\textbf{Finding 1}: An important group of runtime bugs related to signature incompatibilities are due to invoking library methods with reflection or checking class hierarchy of the library.}}Besides the 18 bug reports discussed above, the rest 126 bug reports are all related to behavioral incompatibilities. 


%\subsubsection{Behavioral Incompatibilities}

Before we perform more in-depth investigation, we first manually scanned the bug reports to detect \textit{cross-software duplicate bug reports}. Duplicate bug reports inside a project are typically labeled and we did not select them when collecting our bug report set. However, different client software may fail due to a same BBI of a same library, so these client bug reports are ``duplicate'' with each other. After the duplicate-bug-report detection, we identified 112 BBIs as presented in Table~\ref{table:basicIncomp}. 

\begin{table}
	\center 
	\caption{\label{table:basicIncomp} Bug-Causing BBIs}
	
	\begin{tabular}{|l|r|r|r|}
		\hline
		Subject & Cause Library Bugs& Cause Client Bugs & Total\\ 
		\hline
		JDK &        8      &     10   & 18   \\
		Android &         13     &     51   & 64  \\
		Other   &         29     &    1    & 30  \\
		\hline
		Total   & 50             &  62      & 112     \\
		\hline
	\end{tabular}
	\vspace{+0.3cm}		
\end{table}

\subsubsection{Incompatible Behaviors}

\begin{example}
	\begin{verbatim}
In Android 4.4, SMS apps are no longer able 
to send SMS to the SMS provider (rejected 
silently) in the Android system, unless they 
are reset to receive a broadcast 
SMS_DELIVER_ACTION. 	
 \end{verbatim}
	\caption{Bug-408: Be able to configure as a default 
		SMS app in KitKat (\small{from WhisperSystems/TextSecure})} 
	\label{example:sys}
\end{example}

The breakdown of bug-inducing BBIs per incompatible behaviors is presented in Table~\ref{table:behavior}. Note that, in our bug study, we use the same categories as defined in Section~\ref{subsec:cats}. For incompatible behaviors, we found 2 BBIs that cannot be put into any defined categories, so we add a new category \textit{Sys. Event}, which indicates changes in system events (Example~\ref{example:sys}).

\begin{table}
	\center
	\caption{\label{table:behavior} Categorization of Incompatible Behaviors}
			
	\begin{tabular}{|l|l|r|r|r|r|r|r|r|}
		\hline
		\multicolumn{2}{|l|}{Behavior} & \multicolumn{2}{|c|}{Android} & \multicolumn{2}{|c|}{JDK} & \multicolumn{2}{|c|}{Other} & All\\
		\cline{3-8}
		\multicolumn{2}{|l|}{}      & L & C & L & C & L & C& \\
		\hline
\NameEntry{2}{Exception and Crash}&New Exception &2 &  21&5&5&13&1 & 47\\ 
		&Infinite Loop &0 & 0  & 1&0& 0&  0& 1 \\ 
		\hline
		\NameEntry{4}{Ret. Var. Change}& Value Change &1 & 0& 0&1&2&0&4 \\
		& Field Change & 1 & 2 & 1 & 2 & 7 & 0 & 13\\
		& Type Change&1     & 2 &0 & 0 & 1 & 0 & 4  \\
		& Structure Change&0    & 0 & 0&1&3   & 0 & 4 \\
		\hline
		\NameEntry{4}{Other Effects}& Memory&3 & 4  &0&0&1 & 0& 8 \\         
		&  GUI     &5     &19 & 0&1 &1  &  0 & 26 \\
		& File Sys.   &0     &  2 & 0 &0 & 1&  0 &3  \\
		& Sys. Event    & 0    &  1 & 1 & 0& 0&  0 &2  \\		
		\hline 
	\end{tabular}	
\end{table}

In the table, Columns 2-3 present the number of library bugs (denoted as L) and client bugs (denoted as C) from Android. Columns 4-7 present similar data for Java and Other subjects. Column 8 presents the total of each line. Since Android and JDK dominates our bug report study set, in the rest of the paper, when comparing bug study results and library study results, we provide both overall library study results, and library study results for JDK and Android combined. From Table~\ref{table:behavior}, we have the following observations. 

First, \textit{New Exception} and \textit{Infinite Loop} account for 48 of the 112 (43\%) BBIs, which is higher than their proportion in the library study (28\% overall, 30\% for JDK+Android). These behaviors may be more likely to be reported as bugs due to their severity.

Second, \textit{Value Change} accounts for 4 BBIs (3\%), which is much smaller than its share in library study (29\% overall, 18\% for JDK+ Android). But \textit{Field Change} accounts for 13 BBIs, which is much more than \textit{Value Change} despite its lower share in library study. 

Third, categories \textit{Different Exception} and \textit{No Exception} do not cause any bugs in our data set. This may be because client developers tend to avoid exceptions in their code so they are not affected. This also indicates that such changes are safer at the library side. 

\begin{example}

		\begin{verbatim}
	In Android 4.4, a method that update the 
	padding setting must be invoked before 
	showing the date information, otherwise, 
	part of the date information can not be seen.
		\end{verbatim}
	\caption{Bug-46:Bold ZeroTopPaddingTextView displays cut off on 4.4 (\small{from derekbrameyer/android-betterpickers})} 
	\label{example:ui} 
	
\end{example}
	
Fourth, GUI changes accounts for 26 BBIs (22.3\%), which is the second largest category in bug study, but its share in library study is only 3\% overall. The large number of GUI-change BBIs may be related to the large proportion of Android-related bugs in our bug set. Although these BBIs may be specific to Android framework, they are still important and representative because of the popularity of Android apps, and the existence of similar frameworks. We discovered that, most bug-inducing GUI BBIs are changing settings of UI controls and thus affecting presentation. Examples of the settings include the position to put notification bars (top or bottom), box widths, etc. Example~\ref{example:ui} presents a GUI-related bug. The bug report is caused by a BBI between Android 4.3 and Android 4.4 about the changed value of padding settings. It should be noted that, user interface bugs are not just decoration problems, and they may largely affect software usages (i.e., information cannot be seen, or buttons go outside the screen and cannot be clicked). In general, we have the observation as follows.\\



\medskip
\noindent\begin{tabular}{|p{16cm}|}
	\hline
	\textbf{Observation 5:} Distribution of BBIs in library study and field bug study is mismatched on the incompatibility behavior categories. New Exception, GUI Changes, and other side-effects account for a higher proportion in field bug study, while other categories of BBIs account for a lower proportion.\\
	\hline
\end{tabular}
\medskip
\vspace{+0.2cm}




%Since Android and JDK dominates our bug report study set, in the rest of the paper, when comparing bug study results and library study results, we provide both overall library study results, and library study results for JDK and Android combined. 



%this proportion of 45\% is much more than their proportion (28\% overall, 30\% JDK+Android) in the library study at Figure~\ref{figure:behavior}). This implies that, library developers should be very careful when bringing in incompatibilities in this category. Also, simpler and more scalable techniques that target at unhandled-exception-related changes in the software library may be quite effective in detecting these bug-inducing backward incompatibilities. 


%(50\% overall, 33\% JDK+Android). It is possible that such incompatibilities are relatively simple and are thus easier for client developers to handle or they are not breaking client code bad enough to result in bug reports. This also shows that unit testing, which does well in checking return values under simple inputs and context, is not sufficient for identifying bug-inducing incompatibilities. 





%Such settings are often changed to achieve better presentation of the Android System GUI, but they may also affect the GUI of Android apps. By contrast, the GUI BBIs discovered in our library study are mostly about UI control actions such as dialog not showing or cursor not moving. 

	



\subsubsection{Invocation Constraints}

\begin{table}
	\center
	\caption{\label{table:condition} Categorization of Invocation Conditions}
		
	\begin{tabular}{|l|l|r|r|r|r|r|r|r|}
		\hline
		\multicolumn{2}{|l|}{Conditions} & \multicolumn{2}{|c|}{Android} & \multicolumn{2}{|c|}{Java} & \multicolumn{2}{|c|}{Other} & All\\
		\cline{3-8}
		 \multicolumn{2}{|l|}{}      & L & C & L & C & L & C& \\
		\hline
		\multicolumn{2}{|l|}{Always} &5 &  14  &2  & 1   & 2 &0& 24 \\ 
		\hline
		\multicolumn{2}{|l|}{Environment}  &0 & 1   & 0 & 0   & 1 &0& 2 \\ 
		\hline
		\multicolumn{2}{|l|}{Special Type} &0 &  2  & 0 &  0  & 2 &0& 4 \\ 	
		\hline		
		\multicolumn{2}{|l|}{Multiple APIs} &4& 21 & 1 & 5& 7 &1& 39 \\ 	
		\hline		
		
		\NameEntry{4}{Input}& Trivial Value &0 &  0  & 1 &  0  & 2 &0& 3 \\ 
		& String Format &2 &   2 & 3 &   3 & 6 &0& 16 \\ 
	    & Specific Field &0 & 4   & 0 &  0  & 0 &0& 4 \\ 
		& Specific Value &2 &  7  & 1 &  1  & 9 &0& 20 \\ 
		\hline
	\end{tabular}
	\vspace{+0.3cm}			
\end{table}


The breakdown of bug reports according to BBI-invoking conditions is presented in Table~\ref{table:condition}. From the table, we have the following observations. 

First of all, 24 of 112 (21\%) BBIs always happen (compared to 31\% overall and 22\% in JDK+Android in the library study), causing at least 15 client-side bugs. Second, only 2 of the BBIs are related to the environment. This may be largely due to the platform independence of Java, so we doubt whether this conclusion can be generalized to other programming languages. 
Third, 39 BBIs (35\%) occur only after certain other API methods are invoked and thus belong to the category of \textit{Multiple API}, compared to 7\% overall and 11\% JDK+Android in the library study. This actually implies that a lot of BBIs happen under special usage scenarios which are not covered by the library-side test code. In such cases, client code may be a good source to extract suitable test cases for detecting BBIs in a software library. 
Fourth, no BBIs fall into the \textit{Error} category of invocation-condition, which shows that BBIs in this category are less likely to cause client bugs or the client bugs they cause are difficult to detect. In general, we have the observation as follows.\\

\noindent
\begin{tabular}{|p{16cm}|}
	\hline
	\textbf{Observation 6:} Distribution of BBIs in library study and field bug study is mismatched on invocation constraints. Multiple APIs account for a much higher proportion in field bug study, compared to its share in library study. This implies that regression testing may need to be strengthened for usage patterns involving multiple API methods. 	
%	these categories (e.g., automatic GUI oracle checking, including multiple API usage patterns from client code into regression test suite)
	\\
	\hline
\end{tabular}
\medskip
\vspace{+0.05cm}
%\subsubsection{Call Backs}
%One interesting finding in the 112 studied incompatibilities is that, 6 of them are related to call backs. All 6 incompatibilities are from Android bugs (4 client bugs and 2 library bugs). 

%One example of call-back-related incompatibilities is \textit{Bug-62100: ``WebViewClient.onPageFinished() called multiple times''} of Android system. The bug is a library bug about a backward incompatibility between Android 4.3 and Android 4.4. Specifically, in Android 4.3, this call back method is called only once when a web view with multiple frames is closed. However, in Android 4.4, the developers added a new method \CodeIn{didFinishLoad(...)}, and invoke it when closing each frame (the added method is shown in the code sample below). Note that this method invokes the call back method \CodeIn{onPageFinished(...)}, so in Android 4.4, the call back method is called multiple times when closing a web view with multiple frames. Such a  change may cause severe problem, if the client developers close some resources (closing a closed resource may cause exceptions) or change some global objects such as counters, in the call back. This bug has been reported to and fixed by Android developers. In the fix, the developer added line 4 to check whether the frame to be closed is the main frame, and invoke the call back method only if the frame is a main frame.
	%\vspace{-0.2cm}

%\begin{CodeOut}
%\begin{alltt}
%1 @Override
%2 public void didFinishLoad(long frameId, String validatedUrl, 
%3   boolean isMainFrame) \{
%4+    if (isMainFrame)
%5       AwContentsClient.this.onPageFinished(validatedUrl);
% 	 ...
% \} 
%\end{alltt}	
%\end{CodeOut} %
%	\vspace{-0.2cm}
%Call-back backward incompatibilities can easily happen, because library developers may simply change the way they are calling a certain method inside their code, without noticing that this method is being overridden by client developers. Also, call-back backward incompatibilities are difficult to avoid, because library developers often cannot make any assumptions on the content of the call back. 
		
\subsection{Documentation Study}
\label{subsubsec:docIncomp}
To answer the third research question, we further studied the documentation status of the bug-inducing BBIs. Since these BBIs are bug inducing, we predict that they may be more poorly documented then the BBIs detected from cross-version testing (34\% documented), and the results shown in Table~\ref{table:documentation} confirm our guess. In Table~\ref{table:documentation}, the first column presents the documentation status (and the place if documented). The rest columns are organized similar to Table~\ref{table:behavior}. 

\begin{table}
	\center
	\caption{\label{table:documentation} Documentation Status of BBIs}
	\begin{tabular}{|l|l|r|r|r|r|r|r|r|}
		\hline
		\multicolumn{2}{|l|}{Behavior} & \multicolumn{2}{|c|}{Android} & \multicolumn{2}{|c|}{Java} & \multicolumn{2}{|c|}{Other} & All\\
		\cline{3-8}
		\multicolumn{2}{|l|}{}      & L & C & L & C & L & C& \\
		\hline
		\multicolumn{2}{|l|}{No Doc.} &12&43&7&6&  29&  1  & 98  \\
		\hline
		\multirow{3}{*}{Doc.}&Release Notes &1   & 1&1&3& 0 &  0 & 6  \\
		&JavaDoc       & 0  &  3 &0&1& 0 &   0  & 4  \\
		&Migration Guide &  0  & 4   &0&0& 0& 0  &  4  \\
		\hline 
	\end{tabular}
	\vspace{+0.3cm}	
\end{table}

From Table~\ref{table:documentation}, we can see that bug-inducing behavioral BBIs are very poorly documented. Only 14 (13\%) BBIs are documented.  Also, the documented changes are relatively scattered, especially for Android (in release notes, JavaDocs, and Migration guides). Also, 8 client bugs of Android and 4 client bugs of Java are related to documented behavioral changes. This implies that we may need a better way than documentation to convey the information of behavioral change and remind client software developers about such changes. 
		
\subsection{Bug Resolution Study}
		

To answer \textbf{RQ5}, we further studied how the real world bugs related to BBIs are resolved (note that they may be not fixed). Cross-software duplicate bugs may be fixed differently in different client software, so we view them as separate bugs in this subsection. 

\subsubsection{Resolution of Library Bugs}

The breakdown of bugs according to how they are resolved is shown in Table~\ref{table:libfix}. The first two columns present the types of resolution. If a library bug is fixed, we check whether it is fixed by a simple revert of the previous change, a patch of the previous change, or library developers decided to support both the previous behavior and the new behavior (typically by adding a parameter, and set either the previous behavior or the new behavior as default). If a library bug is not fixed, we study how library developers response to the bug report, and check whether it is intended behavior, or the developer is reporting a behavioral change on internal APIs which should not be used by client developers. 

\begin{table}
	\center
	\caption{\label{table:libfix} Resolution of Library Bugs}
			
	\begin{tabular}{|l|l|r|r|r|r|}
		\hline
		\multicolumn{2}{|l|}{Resolution} & Android & Java & Other & All\\
		\hline
		\NameEntry{3}{Fixed}& Reverted & 1   & 0 & 2    &   3\\
		& Patched&  6 &  4  &  16  &   26 \\
		& Double Support& 0  &  0  & 1   & 1   \\
		\hline
		\NameEntry{3}{Not Fixed}& Intended& 5  & 2 & 10  &  17 \\         
		& Discouraged     & 1  &2  &  0  & 3  \\		\hline 				
	\end{tabular}	
	\vspace{+0.3cm}	
\end{table}

From Table~\ref{table:libfix}, we have the following observations. 
First, 20 of 50 library bugs are not fixed. The major reason is that the behavior change described in the bug report is intentional. It should be noted that, since these behaviors are reported as library bugs, they may already cause some bugs or at least test failures at the client side, although the client bug may not be reported. Second, among the 30 bugs that are fixed, most of them are patched, which shows that the many bug-inducing BBIs are caused by side effect of other productive changes. 

\subsubsection{Resolution of Client Bugs}
The breakdown of client bugs according to how they are resolved is shown in Table~\ref{table:clifix}. The first two columns present the types of resolution. If a client bug is fixed, we check whether it is fixed by (1) changing the incompatible API method to another one; (2) changing the arguments of the incompatible API method to other constants, variable, or expressions; (3) adding an API invocation to set a certain internal-state field before or after the invocation of the incompatible API method; (4) converting the return value of the incompatible API invocation to the original value; (5) a global structural code change; (6) updating libraries; (7) changing configuration of software; or (8) bypassing the incompatibility behavior by skipping software features (see Example~\ref{example:libfix}).

\medskip
\begin{example}	
\begin{verbatim}
The cause of the bug is that, "ResourceNotFound 
Exception" is thrown in Android 5.0 when an 
"overall scroll glow drawer" is requested. 
In the fix, the client developers simply catch 
the exception without doing anything with it. 
Therefore, the request of "overall scroll glow 
drawer" is actually bypassed. 
\end{verbatim}
\caption{Bug-969: Android 5.0 crash when trying to open the app (\small{from open-keychain/open-keychain})} 
	\label{example:libfix}
\end{example}


%An example of bypassing is the resolution of \textit{}. 

For the client bugs that are not fixed, we discovered two resolutions. The first resolution is that the client developer simply decided to wait until a new version library is released. One reason of such resolution is that, the BBI is caused by a regression bug, so the client developer waits for the library developers to release a bug-free version. Another reason (and the major reason in our study) is that, the BBI affects a third-party library that the client developers are relying on. Since the client developers cannot change the code of the third-party library (sometime they even do not have access to the source code), they are not able to resolve the BBI, and have to wait for the new version of the third party library. The second solution is that, the client developer simply tolerate the behavior change (if the BBI does not cause crashes). They may simply ask their users to get used to the new behavior such as a UI change, or transfer the BBI to downstream developers. 

\begin{table}
	\center
	\caption{\label{table:clifix} Resolution of Client Bugs}
	\begin{tabular}{|l|l|r|r|r|r|}
		\hline
		\multicolumn{2}{|l|}{Resolution} & Android & Java & Other & All\\
		\hline
		\NameEntry{8}{Fixed}& Change API & 1   &  2 &   0  &  3  \\
		& Change Input& 13  & 0   &  1  & 14 \\
		& Add Set Field& 17   & 1   &   0  &  18  \\
		& Return Convert& 6  & 0   &  0  &  6  \\
		& Structural&  8 &   5 &   0  &  13  \\
		& Config&    2 &   0 &  0  & 2   \\
		& Lib. Update&  2   &  0  &  0     &  2  \\ 
		& Bypass&   4  &  0  & 0  &  4  \\ 
		
		\hline
		\NameEntry{2}{Not Fixed}& Wait Lib. Fix& 4  & 2 &  0  & 6  \\         
		& Tolerate  & 7  & 0 &  1  &  8 \\		
		\hline 
	\end{tabular}	
	\vspace{+0.3cm}		
\end{table}

Table~\ref{table:libfix} shows that 14 client bugs are not fixed. Note that, we find that most developers are willing to and have tried to fix the bugs, but BBI-related bugs are more difficult to fix, because they typically involve code written by other people. Regarding the fixed client bugs, we have the observation as follows. \\


\noindent\begin{tabular}{|p{16cm}|}
	\hline
	\textbf{Observation 7:} 41 of the 62 fixed client bugs are fixed through small changes including changing API, changing input value, add an API to set field, or convert the return value to the original value. \\
	\hline
\end{tabular}
\\

We present an example of client bug fixing with ``Add an API to set field'' in Example~\ref{example:fix}. The corresponding BBI is that, for Android 5.0, if the developer wants to start a service with an intent, the type of the intent must be explicitly set with the method \CodeIn{setClass(...)}, and the method \CodeIn{startService(...)} will check the \CodeIn{class} field of the intent and throw exceptions if the value is null. %For this example, if we can summarize the addition of field checking in the library code, it will be not difficult to generate the patch code based on the summary. 

\begin{example}
	\begin{verbatim}
 Intent intent = new Intent(
 Actions.NOTIFICATION_SET_FOR_DEVICE); 
+intent.setClass(context, 
     NotificationIntentService.class);
 intent.putExtra(BundleExtraKeys.DEVICE_NAME
     , deviceName);          
 ...
 context.startService(intent); 
	\end{verbatim}
	\caption{Fix: Bug-812: Lollipop notification settings won't work \small{(from klassm/andFHEM)}} 
	\label{example:fix}
\end{example}

		

\subsubsection{Reporting to Library Developers. } In our study, we further studied whether client developers would like to report their bugs to the library developers. Among the 76 client bug reports, we find that the symptom is reported to library developers in only 6 bug reports. In most of the cases, the developers simply search through the Internet to find a workaround. Also, for the 6 reported bugs, only 3 are fixed by the library developers, while the other 3 are rejected because the corresponding BBIs are intended behaviors. 


\section{Discussion}
\label{sec:discuss}
In this section, we discuss the lessons learned, limitations and the threats to our study, and regression faults.
%\subsection{Summary of Major Findings}

%\textbf{At the end of our study, we highlight our major findings in two studies as follows.}

%\begin{enumerate}	

	
%	\item Behavioral backward incompatibilities are prevalent in library evolution. They appear frequently in both major and minor versions, and cause many real-world bugs. 

%	\item The top reasons of bringing in behavioral backward incompatibilities are making behavior more reasonable, enhancing string output format, enforcing input rules, and enable lousy inputs. 

%	\item There is a mismatch between distribution of incompatibilities in our library study and bug study, implying that some incompatibilities are safer or easier to detect, while others are more destructive or difficult to detect. 

%	\item Half of the 12 runtime errors caused by signature incompatibilities are related to the usage of reflection or class-hierarchy checking such as downcast. 
%	\vspace{-0.1cm}
	
%	\item The most common backward incompatible behaviors are unhandled exception and GUI changes, which account for 42\% (47 of 112) and 23\% (26 of 112) of backward incompatibilities. 
%	\vspace{-0.1cm}
	
%	\item As a backward incompatible behavior, change of return value accounts for only 15\% (17 of 112) backward incompatibilities. 
%	\vspace{-0.1cm}
	
%	\item 39 of 112 backward incompatibilities can be revealed only when certain other APIs are invoked before or after the backward incompatible API. 
%	\vspace{-0.1cm}
	
%	\item Most backward incompatibilities (98 of 112) causing the bug reports we studied are not documented at all. 

%	\item Most client bugs are fixed by client developers, and 65\% fixed bugs (49 of 75) are fixed with simple changes such as replacing the input arguments, converting the return values, and add an invocation to a certain field-setting API method before or after the backward incompatible API invocation.

	%
%\end{enumerate}
		

\subsection{Lessons Learned}
		
The final goal of our study is to find actionable goals for library / client developers to avoid or resolve bugs caused by behavioral backward incompatibilities. 


\subsubsection{Avoidance of BBI Bugs}
\textbf{Enforcing Old Tests on Release.} In our study, we are surprised by the large number of BBIs found with simple cross-version testing. Note that all tests come with the project and developers are supposed to run them every time they build the code. Our study shows that, most tests detecting BBIs are changed accordingly or deleted when BBIs were brought in, so they can never detect the BBIs. Therefore, we suggest to enforce testing with old tests when releasing a new version. Such a feature can be added to IDEs or version control systems. It is unnecessary that all old tests pass but developers should provide document or workaround for failed tests, especially on minor version releases. 

\textbf{Augmenting Regression Tests.} Our results in Figure~\ref{figure:behavior} shows that, although 105 BBIs cause different exception / crash status and 86 BBIs cause primitive return value changes, the remaining 105 of 296 BBIs (36\%) cause memory state change (e.g., certain field of the returned object) or other side effects (e.g., GUI, file systems). Such BBIs can be revealed only with extra API calls or side-effect checking. Furthermore, Table~\ref{figure:behavior} shows that such BBIs cause 60 of 112 (54\%) studied BBI-related bugs. Some side effects, such as file and UI change can be difficult to detect using normal assertions. Augmented regression tests with more advanced memory revealing assertions will detect more bug-inducing BBIs, and automatic test augmentation techniques such as Ostra~\cite{OOPSLA06} may be helpful. 



%(1) Library developers should perform cross-version testing when they release a new version. 

%Our first study finds averagely 3 behavioral backward incompatibilities with basic cross-version testing. This is surprising because all the test code we use comes with the source code and the test code are executed each time the project is built. The major reason for the missing 


\textbf{BBI Recommendation.} Our study reveals mismatches between the distribution of BBIs in various categories and the corresponding distribution of BBI-related bugs, which shows that certain categories of BBIs are more likely to result in bugs. For example, GUI changes as incompatible behaviors and multiple APIs as invocation conditions has a much higher population in the studied bug-related BBIs, and none of studied bugs are caused by BBIs related to different exceptions or changed error messages. Also, APIs that are frequently used, once have BBIs, are more likely to result in multiple client-side bugs, which is also observed in our study. Furthermore, our study also observed contradictory reasons of behavioral changes (e.g., allowing lousy input and enforce input rules) and several reverted or double-supported BBIs, which shows that developers are not always clear about the consequences of involving BBIs. Therefore, based on the category and API-usage frequency information (together with other factors such as major or minor version released, development status, etc.), a recommendation system can be helpful for developers when they make decisions on involving a BBI in a release. 

\textbf{Test Change Tracking.} Our study shows that, for 267 of 296 BBIs (90.2\%), developers change their test code to accommodate the behavior change. Therefore, change of test code on public APIs can be a sign of BBIs, and tracking test code changes may provide more information about the happening and evolution of BBIs. 




\subsubsection{Detection and Resolution of BBI-Related Bugs}
\textbf{BBI Notification.} Our study shows that the documentation status of behavioral incompatibilities is very poor. Even when a behavioral change is documented, there are still many relevant client bugs. We believe that, advanced techniques on documentation of behavioral incompatibilities is in a great need and will help reduce many bugs related to backward incompatibilities. The technique should be able to directly check the client code (e.g., finding code clones of a failed library-side test case) and raise warnings about potential relevant behavioral backward incompatibilities. 

\textbf{Advanced GUI Testing.} We find that, GUI behavior change is one of the major cause of BBI-related client bugs. Many of such bugs cannot be easily detected by normal assertions, such as the bugs on text invisibility due to color and size change of UI controls (Example~\ref{example:ui}). Some BBI bugs can be detected only with human eyes. Automatic oracle checking is straightforward for unit regression testing, but hard for GUI regression testing. This calls for more advanced user-interface checking techniques to support automatic regression testing of GUI applications. 

\textbf{Test Code Reference.} Since developers change their test code to accommodate BBIs, the test code change can be used as examples of how to work around BBIs. When using certain API methods, client developers may consider watching the library-side test-code changes, or searching for relevant test-code changes for workarounds when resolving the BBIs from client side. The resource of test code changes can also be used in documentation, BBI notifications, or automatic BBI resolution tools for the client side. 

\textbf{Automatic Fixing of Client Bugs.} Our study shows that, 67\% bugs (41 of 62) fixed from client are based on simple changes such as replacing the input arguments, converting the return values, and add an invocation to a certain field-setting API method before or after the BBI API invocation. Consider Example 5, where an Intent object's field needs to be set before it is used. In many such BBIs, a validation is added in library code to check the field (e.g., checking class field of Intent for accessible classes) and an exception is thrown there. The existence of such patterns shows possibilities that many BBI-related bugs can be fixed automatically by adding new change rules to automatic bug fixing tools.
\vspace{-3ex}
\subsection{Limitations and Threats}
\label{subsec:limit}
\textbf{Limitations.} First, we use cross-version testing to detect BBIs in software libraries. This result in an under-estimation of the number of BBIs between version pairs. Also, since we require all test cases pass in their original versions, the detected BBIs are biased to the intended behavioral changes (since the relevant test cases are already fixed). However, the major goal of our study on regression testing is to show the prevalence of BBIs, and we believe the above mentioned limitations do not affect our conclusion. Second, when studying BBI-related bugs, we searched the bug repositories with keywords such as ``update''. Bugs related to BBIs do not have obvious keywords, such as ``deadlock'' for concurrency bugs. In particular, some BBI-related bugs may stay in the software for a long time, and the client developers may not realize the root cause of the bug even after it is fixed. Thus, our selected bugs may be biased to those bugs that are easily found to be relevant to BBIs. 


%As an early step towards better understanding of behavioral backward incompatibilities, our study has a number of limitations. 




%Third, in our study on bug reports, we largely depend on the comments and code commits of developers to determine whether a bug is fixed or not, and the fix locations. So developers' mistakes may affect the precision of our results. 

%Fourth, in our study of documentation status, we carefully checked release notes, migration guides, and API JavaDocs. However, it is still possible BBIs are documented elsewhere, such as in bug repositories. Thus, we may miss documented BBIs. However, the places we check are also the places client developers may refer to. If the BBIs are documented somewhere hard to find, it is still not well documented. 

\textbf{Threats to Validity.} The major threats to internal validity of our study is the potential errors and mistakes in the process of building software and performing regression testing, studying the bugs, and doing the statistics. To reduce this threat, we carefully wrote all the tools we used, and manually checked the results for correctness. The major threats to external validity is that, our conclusion may hold for only Java software libraries, and the libraries under study. Furthermore, our conclusion may hold for only the bugs studied. Since our bug dataset is dominated by Android and Java, some conclusions (e.g., about GUI) may be specific to these libraries. To reduce this threat, we chose the most popular Java software libraries, as well as randomly chose the bugs to be studied. 


\subsection{Detailed Classification of BBIs}

\textbf{Intention of Behavior Changes.} In our paper, we do not differentiate regression bugs from other BBIs and treat them exactly the same. Our second study shows that, both regression bugs and intentional BBIs cause real-world client bugs. Also, it is very hard to tell whether a behavioral change is intentional because an intentional behavioral change can have unexpected side effects. 

%In our first study, developers updated test cases accordingly for 267 of 296 incompatibilities. These incompatibilities are more likely to be intentional because changes on test code show that developers are aware of the changes, and we have manually found reasons of the BBIs. In our second study, we can infer whether a bug is a regression fault from how they are fixed. For library bugs, according to Table~\ref{table:libfix}, 20 of 50 bugs that are not fixed or resolved with supporting both behaviors should not be regression faults, while others may be regression faults. For client bugs, according to Table~\ref{table:clifix}, only 9 bugs are either fixed with library update or waiting for library fixes. However, this number may not be very informative for client bugs because some client bugs may be temporarily fixed by workarounds and will be fixed with library update in future.


%Though there are still other possibilities such as developer temporarily disabling test cases for the software to build successfully. 

\textbf{Contract-based Classification of Behaviors.} In our paper, we manually categorized incompatible behaviors and invocation constraints in an intuitive way. In future, we plan to apply analysis tools to categorize incompatibilities in a more formal manner. Specifically, we plan to categorize incompatibilities to precondition violations where the new upgraded function requires more from its inputs than the old one(e.g., an non-null input is required) and postcondition violations where the return values of the new function do not subsume the old ones, or the new function throws different exceptions or has different side effects. 


%For two consecutive software versions, regression faults are bugs in the later version that are caused by the changes between the two versions. Since regression bugs also cause behavioral changes between two versions, they can be viewed as a category of behavioral backward incompatibilities that are unintentionally brought in by library developers. 

%In our first study on cross-version testing, we believe that most of detected test failures and errors are NOT regression bugs. The reason has been mentioned in Section~\ref{subsec:limit} (we require all test cases pass in their original versions). This fact implies that the library developers intentionally changed the test code in the new version to avoid test failures / errors. 

%



%This validation (may be located from stack trace of the thrown exception) can provide information on which field to set, and constraint for the value to be set. With such information, automatic fixing tool may exhaust accessible values in the scope to generate fix candidates for further validation with test suite.

%(1) Automatic repair techniques may be helpful for resolving client-side bugs

%(2) Library test cases can be a source to find resolution of client-side bugs


%Generally, findings in our paper show that library developers and client developers need to be more involved to each other's side to reduce behavioral backward incompatibilities and relevant bugs. 


%\textbf{Prevalence of Behavioral Incompatibilities.} Our study shows that behavioral incompatibilities are very common in popular Java software libraries. Although cross-version testing is able to reveal only a small portion of potential backward incompatibilities, we still detected 1094 test failures and errors, as well as identified 296 incompatibility groups. We also find that behavioral incompatibilities do cause lots of real bugs in the real world. 

%\textbf{For Library Developers:} Our study shows that library developers of all popular libraries are not documenting cross-version test failures. A potential reason is that, it has been common practice to build and test at code commit time. So library developers tend to change the test code as their library evolves. An enforcement of cross-version testing at release time would be helpful for documenting incompatibilities. Also, based on such information, library developers could better distribute their changes in major and minor versions. Furthermore, according to our findings, some behavioral changes, such as changing exceptions, are safer, while others, such as change API usage patterns and GUI settings, are more destructive. It would also be desirable for library developers (with proper tool support) to check how the changed API methods are used in client side, and test library code with those code. 





%Our study shows that, the invocation conditions of behavioral incompatibilities vary, and the invocation of other APIs is one of the major condition. This implies that testing an API method together with other methods that may change relevant internal memory state may benefit the detection of behavioral incompatibilities. We also find that, user interface change is one of the major symptom of behavioral incompatibilities. This calls for automatic user-interface checking techniques to support automatic oracle in regression testing of GUI applications. 

%\textbf{For Client Developers:} When upgrading library, client developers should check carefully on newly thrown exceptions (may be from regression test results of library code), 
%latent reference of APIs such as reflections and call-backs, as well as GUI components potentially affected by the upgraded library. Changed library test cases (if available) can be a good source to find solution or workarounds of client-side incompatibility bugs. 


%\textbf{For Researchers:} Our study shows that the documentation status of behavioral incompatibilities is very poor. Even when a behavioral change is documented, there are still many relevant client bugs. We believe that, advanced techniques on documentation of behavioral incompatibilities is in a great need and will help reduce many bugs related to backward incompatibilities. The technique should be able to directly check the client code (e.g., finding code clones of a failed library-side test case) and raise warnings about potential relevant behavioral backward incompatibilities. 



%Our study shows that the documentation status of behavioral incompatibilities is very poor. Even if a behavioral change is documented, there are still many relevant client bugs. We believe that, advanced techniques on documentation of behavioral incompatibilities is in a great need and will help reduce many bugs related to backward incompatibilities. The technique should be able to document various factors of behavioral changes such as the APIs that help to invoke behavioral incompatibilities, and the changes on the user interface. 

%\textbf{Resolution of Behavioral Incompatibilities.} Our study shows that, there are a number of simple change patterns for fixing bugs related to behavioral incompatibilities. Specifically, a lot of bugs are fixed through direct adjustment or replacement of the input values, or conversion of the return values. Also, we find that many bugs are fixed by directly setting proper values to a field whose value is changed or checked due to the behavioral change of the API method. The existence of such patterns shows possibilities that many bugs related to behavioral incompatibilities can be fixed automatically. 




	\vspace{-0.2cm}
\section{Related Work}
\label{sec:related}
	\vspace{-0.1cm}
	
\textbf{Performance Testing and Faults.} Previous work focuses on generating performance test infrastructures and test cases, such as automated performance benchmarking~\cite{KALIBERA}, model-based performance testing framework for workloads~\cite{BARNA11}, using genetic algorithms to expose performance regressions~\cite{LUO16}, learning-based performance testing~\cite{GRECHANIK12}, symbolic-execution-based load-test generation~\cite{ZHANG11}, probabilistic symbolic execution~\cite{Chen2016}, and profiling-based test generation to reach performance bottlenecks~\cite{luoinput2016}. Pradel et al.~\cite{PradelISSTA2014} propose  an approach to support generation of multi-threaded tests based on single-threaded tests. Kwon et al.~\cite{ATC2013} propose an approach to predict execution time of a given input for Android apps. Bound analyses~\cite{SPEED} try to statically estimate the upper bound of loop iterations regarding input sizes, but they cannot be directly applied as the size of collection variables under a  certain test can be difficult to determine. Most recently, Padhye and Sen~\cite{PadhyeICSE2017} propose an  approach to identify collection traversals in program code; such approach has the potential to be used for execution-time prediction. In contrast to such previous work, our approach focuses on prioritizing existing performance test cases. The most related work in this direction is done by Huang et al.~\cite{huang2014performance}, whose differences with our approach are elaborated in Section~\ref{sec:intro}. 

Another related area is research on performance faults, including studies on performance faults~\cite{JIN12, PerfBugStudy}, static performance-fault detection \cite{Nistor14, JOVIC11, KILLIAN10, YAN12}, debugging of known performance faults \cite{SHEN05,HAN12,LEUNG07,AGUILERA03}, automatic patches of performance faults \cite{Nistor15}, and analysis of performance-testing results~\cite{FOO11,FOO10}. 


%The default approach for regression testing is to retest all test cases after releasing a new version, which is an expensive proposition. To solve this problem, there are good collection of industry case studies and research effort on performance regression testing in software systems. 

\noindent\textbf{Test Prioritization and Impact Analysis.} Test prioritization is a well explored area in regression testing to reduce test cost~\cite{HARROLD93,BLACK04,ZHONG06} or to detect functional faults earlier~\cite{ELBAUM00,KIM02,LI07}. Mocking~\cite{MockStudy} is another approach to reduce test cost, but it does not work for performance testing as mocked methods do not have normal execution time. Another related area is test selection or reduction~\cite{ROTHERMEL97,CHEN94,Hao2009} which sets a threshold or other criteria to select/remove part of the test cases. Most of the proposed efforts are based on some coverage criterion for test cases, and/or impact analysis of code commits. The impact analysis falls into three categories: static change impact analysis~\cite{TURVERref,Arnoldref96,Wang2010ASE}, dynamic impact analysis
~\cite{LAW03,ORSO11,APIWATTANAPON05}, and version-history-based impact analysis~\cite{ZIMMERMANN04,SHERRIFF08,MengHima}. Our approach leverages a similar strategy to rank performance tests according to the change impact on them. However, we propose specific techniques to estimate performance impacts, such as collection-loop correlation and performance impact analysis. 

%Functional regression testing is a well explored area to reduce testing cost by test case selection based on test case property andcode modification(), test suite reduction by removing redundancy in test suite () and test cases prioritization orders test case execution in a way to hope (). Different from these work, our goal is to reduce performance regression testing overhead via test suite prioritization based on change impact analysis whether an operation is expensive or lies in hot path. 

%\noindent\textbf{Impact Analysis.} The evolution of software systems and ongoing changes demand for explicit means to assess the impact of a change on existing artifacts and concepts. Thus, software change impact analysis is in the focus of researchers in software engineering. The important difference is that Our proposed method focus on the performance test suite prioritization  via performance impact implication of change.

%\section{Future Works}
\label{sec:future}
In the future, we plan to further explore the following research directions. 

First of all, our study focuses on Java software libraries, so our conclusion may not be generalized to other programming languages. Therefore, we plan to conduct similar studies on software libraries written in other languages, especially non-object-oriented languages to confirm or extend our conclusion. We also plan to inspect more backward-incompatibility-related bug reports. 

Second, as we mentioned in Section~\ref{sec:studylib}, regression testing with developers' test suite may find only a small portion behavioral incompatibilities, and thus results in a very course underestimation of the number of behavioral incompatibilities. In the future, we plan to leverage automatic test generation and more advanced automatic test oracles to better detect behavioral backward incompatibilities. 

Third, due to the difference in the popularity of API methods, the potential influence of backward incompatibilities varies. A backward incompatibility is more important if the relevant API method is used (directly or indirectly) more widely. We plan to further study the influence of behavioral incompatibilities and signature incompatibilities. 

Fourth, in our study, we find a number of challenges and research opportunities including behavioral incompatibilities related to reflections, call backs, GUI, and execution environments, better documentation of behavioral incompatibilities, etc. We plan to address some of these challenges in the future. 
\section{Conclusion}
\label{sec:conclusion}
In this paper, we present a study on behavioral backward incompatibilities based on regression testing of 68 version pairs of 15 Java software libraries, and inspection of 126 real world bugs. From our study, we find that behavioral backward incompatibilities are prevalent among Java software libraries, and caused most of real-world backward-incompatibility bugs. Furthermore, many of the behavioral backward incompatibilities are intentional, but are rarely well documented. We also categorized behavioral backward incompatibilities according to incompatible behaviors and invocation conditions, and found category mismatches between the the BBIs detected in regression testing and the BBIs causing bugs. 

%Finally, we found that most BBI-related bugs are fixed by client developers with relatively simple patches.

%compare those detected in regression testing with those causing real-world bugs in the field. 



%behavioral backward incompatibilities , 

%and found that, unhandled exception and GUI changes are the most common incompatible behaviors, and most client software bugs caused by backward incompatibility of software libraries are fixed with simple changes to the code around the backward incompatible API invocation. 
\balance



\chapter{Lesson Learned}
From our large scale empirical study on the usage of mocking frameworks in software testing, we have learned that mocking frameworks are widely used in practice, and a small portion of dependencies are mocked - developer most likely to mock network, database and time consuming services API. The reason of mocking is that software dependencies such as web services and databases are very slow when they are invoked. Thus involving such dependencies in testing will slow down the whole testing process,
which may be fine for system testing, but not acceptable in unit testing and regression testing which are typically performed whenever a change is committed.\\

Regression performance testing is an important but time/resource consuming process. Developers need to detect performance regressions
as early as possible to reduce their negative impact and fixing cost. But in the evolution of a code it is very challenging 
that a code commit may include any type and scope of code changes, from one line revision, to feature addition and interface revision. 
A code commit may contain newly added code, especially new loops. No execution information of such code is available, but given that loops can have high impact
on performance, there is a strong need of estimating the code commits execution time and frequency. Even if the execution time of changed code in a code commit has little impact on performance, the code commit may include changes on collection variables, eventually
affecting the performance of unchanged code. From our study, we can conclude that it better to replace mocking with real code when code change performance impact is high.\\

Our finding from BBIs study shows that BBIs are prevalent among Java software libraries and majority of BBI bugs
are not documented. There is little difference between major version pairs and minor version pairs suffered from backward incompatibilities on average,
so they are equally important for BBI testing. Furthermore, majority client bugs are fixed through small changes including changing
API, changing input value, add an API to set field, or convert the return value to the original value which signifies that majority of BBI can be
fixed automatically. Cross version testing is an important way to find BBI bugs bolster that developer should be careful when to mock and when not. 





\chapter{Future Directions}

\section{Prioritization Regression Tests}
The challenge of static analysis without profiling and input may not may not accurately assess the risk of sophisticated performance
regression issues such as resource contention, caching effect. Improve the accuracy our cost model include context sensitive
profiling information. Our current work consider single threaded program because of thread context switch,
it is hard to capture execution time accurately. It would be better to consider thread contention as a feedback to
our model to handle multi threaded program.\\ 

To provide more information about the changes on API methods, there have also been research efforts trying to summarize changes between two consecutive versions of a software library. On the signature level, Wu et al.~\cite{Wu:AUCA} proposed AUCA, an auditor for API changes, that reports a large variety of signature-level changes of APIs. Moreno et al.~\cite{Moreno:ARENA} proposed ARENA, an automatic tool to summarize software-library changes and generate release notes. There is a good research direction to combine with our approach to make better hybrid model.\\


On the behavior level, McCamant and Ernst~\cite{McCamant:FSEUpgrade,McCamant:ECoopUpgrade} proposed to represent behavior API methods with program invariants generated with Daikon. Person et al.~\cite{Person:FSEDiffSE,Person:PLDIIncreSE} proposed differential symbolic execution to summarize as symbolic expressions of inputs the semantic difference between two versions of a method. Lahiri et. al~\cite{Lahiri:CAVSymDiff} proposed SymDiff, a tool that leverages a modularized approach to check semantic equivalence of different code versions, and calculate program paths that can reveal code behavioral difference.\\

There is no research to prioritize regression tests based on backward incompatibility. Behavioral Backward incompatibility is our another research study where we categorize the behavioral bug beyond api signature change. It would be excellent research direction to prioritization regression tests based on backward incompatibility.

\section{Augmenting Regression Tests}
Software engineers use regression testing to validate software as it evolves. To do this cost-effectively, they often begin by running
existing test cases. Existing test cases, however, may not be adequate to validate the code or system behaviors that are present in a
new version of a system. Test suite augmentation techniques address this problem, by identifying where new test
cases are needed and then creating them with reuse of existing test cases for augmentation. This is because existing test cases provide a rich source of data on potential inputs and code reachability, and existing test cases are naturally available as a starting point in the regression testing context. To create effective test suite augmentation techniques we need to understand the influence of the foregoing factors. Based on such an understanding, It would be better to create augmentation
techniques that leverage test cases in a cost-effective manner. Input of test case are foreign element that effect the performance, how an existing test-generation tool to generate new test inputs to augment the existing test suite would be an excellent idea to go forward.\\

Our existing performance model can be extended easily by correlating input to collection or array. We can easily model the
input change and amplify performance change that depends on input. It will be good tool to tune application performance on
the average case and also for performance debugging. Behavioral bug like file change and UI layout change can be difficult to detect using normal assertions. Augmented regression tests with more advanced memory revealing code and assertions will detect more bug-inducing BBIs, and automatic test augmentation techniques such as Ostra~\cite{OOPSLA06} may be helpful. So augmented test suite would be better approach to expose more behavioral bug automatically.

\section{Logging}
 Inappropriate provisioning of resources may lead to unexpected performance bottlenecks or memory overflow. So there is an essential need to system monitoring. Monitoring a computer on which System Monitor is running can affect computer performance slightly. Therefore, either log the System Monitor data to another disk (or computer) so that it reduces the effect on the computer being monitored, or run System Monitor from a remote computer. Using the logging trace System can predict resource usage and integrating with our performance model would be able to predict more sophisticated performance impact. 

\appendix




\bibliographystyle{plain}
\nocite{*}
\bibliography{sampleThesis}



\end{document}
