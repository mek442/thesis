\section{Related}
\label{sec:related}

As far as we know, this paper is the first research effort to study the usage status of mocking frameworks and mock objects in software practice. There have been a number of existing research efforts on enhancing mocking frameworks or leveraging mock objects to improve automatic software testing. Freeman et al.~\cite{IEEEhowto:kopka}~\cite{Freeman} presented the basic process and concepts of using mocks objects in unit testing, as well as a mocking framework jMock. Galler et al.~\cite{Galler} proposed an approach to automatically generation of mocking objects satisfying certain preconditions to serve as test inputs in automatic test-case generation for Java unit testing. Taneja et al.~\cite{Taneja} proposed to automatically generate mock objects to simulate the behavior of database systems in the automatic test-case generation of database systems. Coelho et al.~\cite{Coelho} proposed an approach to generate mock agents in the unit testing of multi-agent software systems. Due to the necessity of mock objects in unit testing, researchers also proposed approaches to automatically generating mock objects along with unit test cases ~\cite{woda}~\cite{Pasternak}. For the existing test cases which are not using mock objects, Saff and Ernst~\cite{Saff} proposed an approach to automatically refactor such test cases by adding mock objects. Marri et al.~\cite{Marri} carried out an experiment to study the benefit of using mock objects to simulate file systems. All these research efforts are about automatic generation of mock objects and how to leverage mock objects in specific testing problems, while our work focuses on the current usage status of mocking frameworks and mock objects in real world software projects. 

There are also some existing studies investigating the status of testing practice, or the effectiveness of testing techniques on real-world open source projects. Singh et al.~\cite{singh2013empirical} empirically explored 20,000 open-source projects, and found that bigger projects have a higher probability to contain test cases and projects with more developers may have higher number of test cases. In addition, they also found that the number of test cases has a weak correlation with the number of bugs. Greiler et al.~\cite{greiler2012test} explored the current testing practices currently used for the specific plug-in systems. Pham et al.~\cite{pham2013creating} investigated how social coding sites influence testing behavior. They found several strategies that software developers and managers can use to positively influence the testing behavior. Fraser et al.~\cite{fraser2012sound} empirically evaluated automated test generation on 100 real-world Java projects. They found that high coverage is achievable on commonly used types of classes, and also identified future directions for further improve code coverage. In this paper, we aims to investigate the differences between manual tests and DSE-based tests for real-world systems.
