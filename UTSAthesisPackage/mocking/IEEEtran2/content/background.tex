\section{Background}
\label{sec:background}
In this section, we first introduce some basic background knowledge on mocking objects. Then, we introduce the mechanism of mocking frameworks and some major mocking frameworks.
\subsection{Mocking Objects}

Mocking objects are used in software testing to simulate software dependencies so that the testing process can be accelerated and the testing scope can be limited to the component under test (instead of going beyond the interface of dependencies and invoke potential bugs relevant to dependencies). To simulate real dependencies, mock objects typically have the same interface as the objects they mimic. Therefore, the client object remains unaware of whether it is using a real object or a mock object. 

%The major difference between mock object and other test doubles (i.e., objects simulating software dependencies, such as test stubs, fake objects) is that, on top of simulating software dependencies, mock objects also contain assertions of their own. Mock objects will examine the context of each invocation of their methods, on whether a method is invoked, whether several methods are invoked in correct order, and whether arguments of the methods are passed in correctly. 

For example, when testing an email client, software testers may not use the real email server which has slow response and random failures. By contrast, they may construct a fake email server object (a mock object) that returns whatever expected by the software testers. The expected return value can be either a simple success code for the testing of normal functioning of the email client, or a server error code for the testing of error handling of the email client. Such a fake email server helps to reduce testing time as well as to explore the handling of server errors that are very hard to invoke with a real server. However, when such a simplified email server is used, it is hard for testers to tell whether the email client has invoked the email server in a correct order (if multiple-step configuration of the server is required), or whether the client has passed correct arguments to the server. 

Therefore, the software testers may further have some code in the fake email server to check whether the client interacts with it correctly and passes correct arguments. With such a mock email server, software testers can perform quick and controllable testing of the email client without losing the thorough exploration of the interface between the email client and the email server.


\subsection{Mocking Frameworks}

As mentioned in the previous section, mocking objects are desirable for simulating software dependencies in software testing. However, since mock objects need to perform various checking of the invocations, it is usually nontrivial to write such mock objects, and thus the benefit of using mock objects can be undermined by the effort spent on writing mock objects. 

To solve this problem, in the past decade, people have developed various mocking frameworks to generate mock objects automatically. A code sample of using a mocking framework is presented below. With a mocking framework, software testers first create an empty mock object (Line 3). Then, for the created mock object, software testers can specify a number of invocation for the object to check, both on the order of the invocations and the correctness of the arguments (Line 6). After that, in the test code, testers run the component under test as usual but with created mock objects (Line 8). During the execution, if the specified invocation order or arguments are not satisfied, the mock object will throw a test failure so that the software testers know that something wrong happened (Line 9). 

\vspace{-0.3cm}
\begin{lstlisting}[language=Java]
1: @Test
2: public void testEmailClient(){
3:     Server sv = EasyMock.
            CreateMock(Server.class); 
4:     Client c = new Client(sv);         
5:     Email em = new Email(...)
        //record
6:     EasyMock.expect(sv.send(em)).
            andReturn(0);
        //replay
7:     EasyMock.replay(sv);
8:     c.send(em)
9:     EasyMock.verify(sv)
10:}
\end{lstlisting}
\vspace{-0.3cm}

For each major programming language, there have been a number of mocking frameworks developed, such as Mockito, EasyMock, and JMock for Java, pmock, mox for Python, and NMock, Moq for C\#. All of these mocking frameworks support the basic features of mocking including the creation of mock objects, the specification of invocation orders and arguments, and the verification of the specifications. The differences between these mocking frameworks are mainly on the design of APIs and supporting of advanced specifications such as checking the number of invocations of a certain method. 
