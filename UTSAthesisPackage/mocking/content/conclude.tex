\section{Conclusion}
\label{sec:concludemock}
In this paper, we direct a large scale empirical study on the usage of mocking frameworks in software testing. To perform the study, we collected 6,000 Java software projects from Github, and analyze the source code of the projects to answer a number of questions about the popularity of mocking frameworks, the usage of mocking frameworks, and the mocked classes. 
Our major findings include that mocking frameworks are widely used in practice, and a small portion of dependencies are mocked. This finding shows the requirement of more research on mocking frameworks, as well as on the reasons why developers choose to mock a class while not to mock the other. Furthermore, we find that a number of unique features in Mockito and EasyMock are widely used. This implies that it is possible to build better mocking frameworks by incorporating the most popular features of existing mocking frameworks. We also find that software developers tend to do more mocks on source code classes than library classes, which is not as we expected. 

In the future, we plan to extend our work in the following directions. First of all, we plan to direct similar studies on software projects written in programming languages other than Java to check whether our findings can be generalized. Second, we plan to develop techniques to help software developers and testers make decisions on whether or not a class should be mocked. Third, we plan to develop techniques to generate mock objects automatically with mocking frameworks in automatic unit testing. 